\chapter{Models of Steganography}
After the previous chapter introduced all necessary notions concerning
cryptography, this chapter deal with the formal definitions of
\emph{provably secure steganography}. Throughout this thesis, we will
use multiple different models of steganography, that mainly differ in
three aspects:
\begin{description}
\item[Efficiency:] The first formal definition of provably secure
  steganography was given by \citeauthor{hopper2009provably} in
  \cite{hopper2009provably}, the running time of a stegosystem was
  not necessarily efficient, \ie not bounded by a polynomial in the
  security parameter $\kappa$. While some subsequent works defined
  efficiency as a requirement (see \eg \cite{backes2005active,
    dedic2009lower}), \citeauthor{hopper2009provably} make use of the
  fact that their stegosystems may run for a long time to obtain their
  results. We thus distinguish between the original definition -- which
  we will call \myDefInformal{non-efficient stegosystems} -- and the
  updated notion \myDefInformal{efficient stegosystems}. 
\item[Applicability:] A typical problem that arises when one designs a stegosystem
  concerns their \emph{applicability}: On which kind of channels should
  or stegosystem work? One could for example design a stegosystem that
  works for a concrete channel where the documents are $200$ \acs{JPEG}
  pictures of size $600\times 600$ pixels that we know in
  beforehand. \myNote{This nomenclature is taken from
    \cite{liskiewicz2013grey}.}
Such a stegosystem is called a \myDefInformal{white-box
    stegosystem}, as the stegosystem has complete knowledge of the
  channel. Typically one wants to design more general stegosystems. For
  example, it might be appropriate to design a stegosystem that works
  for all channels that contain \acs{JPEG} pictures of size $600\times
  600$ pixels. As the stegosystem still has some knowledge about the
  documents, such a system is called a \myDefInformal{grey-box
    stegosystem}. The most general form of a stegosystem is a
  stegosystem that works \emph{on every channel} (containing
  sufficiently many documents). We call such a system a
  \myDefInformal{universal stegosystem} or a \myDefInformal{black-box
    stegosystem}. As we try to give as general results as possible in
  this thesis, we will develop grey-box or black-box stegosystems for
  our positive results and rule out white-box stegosystems for our
  negative results.
\item[Key-Symmetry:] As in the cryptographic setting, the stegoencoder
  needs a key $k$ to encode the message into the channel  and the
  stegodecoder also needs a key $k'$. If $k=k'$ we speak of a
  \myDefInformal{symmetric-key stegosystem} or \myDefInformal{secret-key
    stegosystem}. In contrast, if $k\neq k'$ and $k$ is publicly known
  and $k'$ is kept secret, we call such a system a
  \myDefInformal{public-key stegosystem}. Furthermore, we denote the
  publicly known key $k$ as $\pk$ (for public key) and the secret key
  $k'$ as $\sk$ (for secret key). Depending on the setting we will also
  analyze different security notions. 
\end{description}

To help the reader to keep track which of these $2\cdot 3\cdot 2=12$
configurations we currently use, the names of the chapters typically
contain all information about the notions used in the chapter. We will
also always give a short description about these aspects in the first
few sentences of the chapter.

\section{Unsuspicious Communication}
In order to formalize that the output of a secure stegosystem is
indistinguishable from unsuspicious communication, we first need a mean
to define this unsuspicious communication. We will do this via the
notion of a \emph{channel}. We will think of this unsuspicious
communication as the unidirectional transmission of \emph{documents}
from Alice to Bob and will model this as a probability distribution upon
those documents. This distribution indicates the probability that Alice
sends a certain document to Bob. There are two more things we need to
consider to make this model realistic and useful for us. First, the
probabilities may change over the time depending on the already sent
documents. If Alice sends Bob a postcard from the beach, it is quite
unlikely (though not impossible) that the next postcard that Bobs get
will come from the Antarctic. This change of the probability
distribution will be reflected by something we call the \emph{history}
-- the sequence of already transmitted documents. Second, larger
security parameters typically allow us to send larger messages. Hence,
the amount of information needed to hide those messages also grows. 
To hide those messages, there are to approaches to handle the need for
more information:
\begin{itemize}
\item In the first approach used by \citeauthor{hopper2009provably} in
  \cite{hopper2009provably}, it is assumed that the size of the
  documents is independent from the security parameter and thus treated
  as a constant. In order to have a large enough entropy to handle
  larger messages, \citeauthor{hopper2009provably} do not deal with
  single documents, but rather with sequences of documents of sufficient
  length. This model was critized by
  \citeauthor{lysyanskaya2006imperfect} in
  \cite{lysyanskaya2006imperfect} as one should only be able to look at
  the distribution with history $h$ containing the document $d$
  \emph{after} the document $d$ was transmitted to Bob especially if the
  the size of documents is very small.
\item In the second approach that we will use, we assume that the size
  of the document depends on the security parameter, \ie the entropy of
  a single document is high enough already. This approach is more
  general then the first one as we will simply interpret a sequence of
  constant-sized objects as a single document. This simplifies the
  analysis and our notation as we can always directly talk about
  documents and not about sequences.
\end{itemize}

\begin{example}
  Let us look at the example that Alice send Bob pictures from her
  holiday. Suppose that every picture is encoded in \acs{JPEG} and of
  size $600\times 600$ pixels. Denote the set of all such pictures by
  $\Pics$. Furthermore suppose that on security
  parameter $\kappa$, we want to embed messages of length $m(\kappa)$. 

  In the first approach, a document $d$ would consist of a \emph{single}
  picture, \ie $d\in \Pics$ and we would only deal with sequences of
  pictures of length $\approx m(\kappa)$. Hence, our channel would be a
  probability distribution on $\Pics$, but our stegosystem would only
  deal with sequences taken from this distribution.

  In the second approach, a document $d$ would already consists of sequence
  of pictures, \ie $d\in \Pics^{m(\kappa)}$. Hence, our channel would be a
  probability distribution on $\Pics^{m(\kappa)}$ and our stegosystem would
  also directly deal with elements of this distribution. 
\end{example}

Formally, a \myDef{channel} $\chan$ on the alphabet $\Sigma$ is a
function that maps an element $h\in \Sigma^{*}$ -- the \myDef{history}
-- and a number $n\in \nats$ -- the \myDef{document length} -- to a
probability distribution on $\Sigma^{n}$. We will denote this
probability distribution by $\chan_{h,n}$ instead of
$\chan(h,n)$. Typically, we will implicitly assume that
$\Sigma=\{0,1\}$ to simplify the following analysis concerning the
amount of information that is present in the channel $\chan$. The
\myDef{min-entropy of a channel} $\minent(\chan,n)$ for
a channel $\chan$ and a natural number $n\in \nats$ is defined as
$\minent(\chan,n) = \min_{h\in \Sigma^{*}}\{\minent(\chan_{h,n})\}$. As
demonstrated in \cite{hopper2009provably}, the number of bits embeddable
in a \emph{single document} is bounded by $\minent(\chan,n)$. 

\section{Stegosystems}
We are now able to finally describe the notion of a stegosystem. As
discussed in the beginning of the section, we will follow the definition
of \cite{hopper2009provably} and will not assume that a stegosystem
needs to run in polynomial time. In order to reduce the redundancy of
this work, we will only define secret-key stegosystems and then
explain the (relatively minor) differences to public-key systems later
on. Let $\mu,n$ and $\ell$ be polynomials throughout this chapter that
will model that the stegoencoder upon security parameter $\kappa$ takes
a message of length $\mu(\kappa)$ and embedds it into $\ell(\kappa)$
documents of length $n(\kappa)$. We thus call $\mu$ the \myDef{message
  length}, $n$ the \myDef{document length} and $\ell$ the \myDef{output
  length} of the stegosystem. 


A \myDef{stegosystem}
$(\Gen,\SEnc,\SDec)$ is a triple of \acp{PTM} such that the algorithm
$\Gen(1^{\kappa})$ produces a key $k\in \{0,1\}^{*}$ with $|k|\geq
\kappa$. The \myDef{stegoencoder} $\SEnc$ takes as input the key $k$, a
message $m\in \{0,1\}^{\mu(\kappa)}$, a history $h\in
\left(\Sigma^{n(\kappa)}\right)^{*}$ and some state informations $s\in
\{0,1\}^{*}$ and outputs a \emph{single document} $d\in
\Sigma^{n(\kappa)}$ and updated state information $s'\in
\{0,1\}^{*}$. Its goal is to embed a piece of $m$ into the document
$d$. It will also have access to samples of the probability distribution
$\chan_{h,n(\kappa)}$. The complete output of the run of the
stegoencoder is denoted by $\SEnc^{\chan}(k,m,h)$ and defined by the
following scheme:
\myAlgorithm{$\SEnc^{\chan}(k,m,h)$}{Key $k$, message $m$, history $h$}{
\State $s := \varnothing$ \Comment{initialize the empty state}
\For{$i=1,2,\ldots,\ell(\kappa)$}
\State $(d_{i},s) \gets \SEnc^{\chan_{h,n(\kappa)}}(k,m,h,s)$
\State $h := h\mid\mid d_{i}$
\EndFor
\State \AlgReturn{$d_{1},d_{2},\ldots,d_{\ell(\kappa)}$}
}{Complete run of stegoencoder $\SEnc$}



\section{Security Notions}

%%% Local Variables:
%%% TeX-master: "../main"
%%% End:

