\chapter{Preliminaries}
To understand the formal models of steganography presented in the next
chapter, we first need to introduce some basic notations concerning
probabilities, algorithms and cryptographic primitives. We denote the
set of natural numbers, including $0$, by $\nats$, the set of real
numbers by $\reals$ and the set of rational numbers by $\rats$. For an
alphabet $\Sigma$ and a string $s\in \Sigma^{*}$, we denote the length
of $s$ by $|s|$ and for two strings $s,s'\in \Sigma^{*}$, the
concatenation of $s$ and $s'$ is written as $s\concat s'$. For a set
$S$, denote by $\powerset(S)$ the set of all subsets of $S$. 



\section{Probabilities}
As the undetectable embedding of a message into a document is an
inherent random process, we will now give a short overview on the
probability theory needed in this work. As no continuous probability
spaces are used in this thesis, it is sufficient to focus on the
discrete case. For a thorough discussion of this subject, see \eg the
textbook of \citeauthor{mitzenmacher2005probability}
\cite{mitzenmacher2005probability}. A
\myDef{probability distribution} $\Pr$ upon a \myDef{probability space}
$\Omega$ -- a finite or countable infinite set -- is a function
$\Pr\colon\powerset(\Omega)\to [0,1]$ such that $\Pr(\emptyset)=0$,
$\Pr(\Omega)=1$ and $\Pr(A\cup B)=\Pr(A)+\Pr(B)-\Pr(A\cap B)$ for all
$A,B\subseteq \Omega$. The elements of $\Omega$ are called
\myDef{elementary events} and subsets of $\Omega$ are simply called
\myDef{events}. To simplify notation, we omit the probability space if
it is clear from the context or denote it by $\dom(\Pr)$. A very
important subset of elementary events is the set of all elementary
events that may occur, \ie those that have probability greater than
zero. This set is called the \myDef{support} of $\Pr$ and we denote it
by $\supp(\Pr)=\{\omega\in \dom(\Pr)\mid \Pr(\omega) > 0\}$. The
\myDef{min-entropy} measures the amount of randomness of a probability
distribution $\Pr$ and is defined as
$\minent(\Pr)=\inf_{\omega\in \supp(\Pr)}\{-\log \Pr(\omega)\}$. For two
events $A$ and $B$, the \myDef{conditional probability} that $A$ occurs
given that $B$ occurs is defined as
$\Pr(A\mid B) := \frac{\Pr(A\cap B)}{\Pr(B)}$. We say that $A$ and $B$
are \myDef{independent events}, if $\Pr(A\mid B)=\Pr(A)$.

\begin{example}
  To describe the throw of a six-sided dice, the elementary events are
  described by
  $\Omega=\dom(\Pr)=\{\epsdice{1},\epsdice{2},\epsdice{3},\epsdice{4},\epsdice{5},\epsdice{6}\}$. The
  probabilities are then given by
  $\Pr(\{\epsdice{1}\})=\Pr(\{\epsdice{2}\})=\Pr(\{\epsdice{3}\})=\Pr(\{\epsdice{4}\})=\Pr(\{\epsdice{5}\})=\Pr(\{\epsdice{6}\})=1/6$. The
  Events $A=\{\epsdice{1},\epsdice{3},\epsdice{5}\}$ (whether the thrown
  side shows an odd number of eyes) and $B=\{\epsdice{5},\epsdice{6}\}$
  (whether the number of eyes is strictly larger than four) are
  independent, as $\Pr(A)=1/2$, $\Pr(B)=1/3$ and
  $\Pr(A\cap B)=\Pr(\{\epsdice{5}\})=1/6$. \myNote{Examples always end
    with \examplesymbol}
\end{example}

It is often convenient to assign certain values from a set $S$ to the elementary
events. This is formally described by the notion of a
$S$-valued \myDef{random variable}, which is a mapping from $\Omega$
to $S$. If $\Pr$ is a probability distribution on $\Omega$ and $X\colon
\Omega\to S$ is a random variable, we define $\Pr[X=x] :=
\Pr(X^{-1}(x))$ as the probability that $X$ gets the value $x$. Most of
the time, the probability space that we use is clear from the
context. If it is not clear or if we want to remark the reader that a
certain value $y$ is chosen at random, we will denote this as $\Pr_{y}$
or write it down explicitly.


To measure the expected outcome of a real-valued random variable
$X\colon \Omega\to \reals$, we define the \myDef{expected value} of $X$
as $\Exp[X] := \sum_{x\in X(\Omega)}x\cdot \Pr[X=x]$. Two random
variables $X$ and $Y$ are \myDef{independent random variables}, if
$X^{-1}(x)$ and $Y^{-1}(y)$ are independent events for all
$x\in \img(X)$ and all $y\in \img(Y)$.
\begin{example}
  Continuing the previous example, the canonical random variable $X$
  would assign $X(\epsdice{1})=1$, $X(\epsdice{2})=2$ and so on. But we
  could also measure whether the number of eyes was odd by using the
  random variable $Y$ with
  $Y(\epsdice{1})=Y(\epsdice{3})=Y(\epsdice{5})=1$ and
  $Y(\epsdice{2})=Y(\epsdice{4})=Y(\epsdice{6})=0$. Hence
  $\Pr[Y=1]=\Pr(Y^{-1}(1))=\Pr(\{\epsdice{1},\epsdice{2},\epsdice{3}\})$.
  The respective expected values are $\Exp[X]=7/2$ and $\Exp[Y]=1/2$.
\end{example}


The simplest random variable one can think of is a binary indicator
variable that only takes values $0$ and $1$. A \myDef{Bernoulli random
  variable} $X$ with parameter $p$ only takes the values $0$ and $1$
with probability $p$ and $1-p$ \ie $\Pr[X=0]=p$ and $\Pr[X=1]=1-p$. If
$X_{1},X_{2},\ldots,X_{n}$ are independent Bernoulli random variables
with parameters $p_{1}=p_{2}=\ldots=p_{n}=p$, their sum
$X=\sum_{i=1}^{n}X_{i}$ is called a \myDef{Binomial random variable} $X$
with parameters $p$ and $n$. It takes values $0,1,\ldots,n$ with
$\Pr[X=k]=\binom{n}{k}\cdot p^{k}\cdot (1-p)^{n-k}$ for each
$k=0,1,\ldots,n$. 

It is often important to rule out some events by proving that they very
rarely occur. For the special case of a (generalized) Binomial random
variable $X$, the extremely helpful \emph{Chernoff bound} shows that $X$
very rarely deviates from its expected value.

\begin{theorem}[Chernoff Bound]
Let $X_{1},\ldots,X_{n}$ be independent Bernoulli random variables
with parameters $p_{1},p_{2},\ldots,p_{n}$ and $X=\sum_{i=1}^{n}X_{i}$
with expected value $\mu=\Exp[X]=\sum_{i=1}^{n}p_{i}$. For every $0< \delta < 1$,
\begin{align*}
  \Pr[|X-\mu|\geq \delta\cdot \mu]\leq 2\cdot \exp(-(\mu\cdot \delta^{2})/3).
\end{align*}
\end{theorem}


\begin{example}
  The Chernoff bound tells us that after throwing $1{,}000$ fair coins,
  the probability that at most $100$ heads or at least $900$ heads
  occurred, is bounded by $2\cdot \exp(-(500\cdot 0{.}8\cdot
  0{.}8)/3)\leq 10^{-48}$.
\end{example}


By letting the parameter $n$ grow to $\infty$, while keeping $p=p(n)$ as
a bounded function of $n$, the resulting random variable will also be
useful in the later chapters and is easily described by the following
theorem (see \eg \cite[pp. 98-99]{mitzenmacher2005probability} for a proof). 

\begin{theorem}
  Let $X_{n}$ be a Binomial random variable with parameters $n$ and $p(n)$,
  where $p$ is a function of $n$ and $\lim_{n\to \infty} n\cdot
  p(n)=\mu$ is a constant independent of $n$. Then, for any fixed
  $k$, 
  \begin{align*}
    \lim_{n\to \infty}\Pr[X_{n}=k]=\frac{\exp(-\mu)\cdot \mu^{k}}{k!}. 
  \end{align*}
\end{theorem}

This fact leads to the definition of a Poisson random variable. A
\myDef{Poisson random variable} with parameter $\mu$ takes values in
$\nats$ with probability $\Pr[X=k]=\frac{\exp(-\mu)\cdot \mu^{k}}{k!}$. These random
variable can be used to analyze a variety of \emph{balls into bin}
experiments relatively easy and we will make use of them later on. 

If $P$ and $Q$ are probability distributions upon a common ground set
$\Omega$, their \myDef{statistical distance} $D_{S}(P,Q)$ is defined as
\begin{align*}
  D_{S}(P,Q) := \sum_{x\in \Omega}| P(\{x\}) - Q(\{x\})|.
\end{align*}


\section{Algorithms}
We use \emph{Turing machines} as our model of computation in this
work. For a detailed introduction and formal definitions, see
the textbook of \citeauthor{papadimitriou1994complexity}
\cite{papadimitriou1994complexity}. The Turing machines in this work
will also be able to make independent fair coin flips and will thus be
called \myDef{\ac{PTM}}. We will use the
terms \ac{PTM} and algorithm interchangeably.  The output of such a
machine is thus a random variable upon the probability space
$\{0,1\}^{k}$, where $k$ is the maximum number of coin flips performed
by the machine. If $\machine$ is a \ac{PTM}, we write $\machine(x)$ for
the random variable that describes the output of $\machine$ on input
$x$. If $\machine$ uses at most $\rho(x)$ coin flips on input $x$ and
$\vec{r}\in \{0,1\}^{\rho(x)}$ is a vector describing those coin-flips,
we write $\machine(x; \vec{r})$ for the (deterministic) output of $\machine$
on input $x$ if we replace the result of the random coin flips by
$\vec{r}$. The \emph{running time} of $\machine$ on input $x$ with random
coins $\vec{r}$ -- the number of steps the machine performs -- is denoted by
$T_{\machine}(x,\vec{r})$. Similarly, the (worst-case expected) \myDef{running
  time} $T_{\machine}(n)$ of a \ac{PTM} $\machine$ is defined
as the expected number of steps that the machine performs, \ie
\begin{align*}
  T_{\machine}(n) := \max_{x\in \{0,1\}^{n}}\{\Exp_{\vec{r}}[T_{\machine}(x,\vec{r})]\},
\end{align*}
where the expected value is taken over the random choices of the coin
flips $\vec{r}$ and \emph{not} over some distribution on $x$. If the running time of
a \ac{PTM} $\machine$ is bounded by a polynomial, we say that $\machine$
is a \myDef{\ac{PPTM}} or an efficient algorithm. 

Often, the machine $\machine$ will also be equipped with different
\myDef{oracles}, that allow us to increase the abilities of $\machine$. See the
next section for examples of this. 
\begin{itemize}
\item For a random variable $X$ (\eg another machine), the PTM $\machine^{X}$
  can get a sample $x$ distributed according to $X$. If $X$ is the
  uniform distribution on a set $S$, we simply write
  $\machine^{S}$. The running time to receive
  a single sample is simply the encoding length of the sample.
\item If $f\colon U\to V$ is a function, $\machine^{f}$ can provide an element
  $u\in U$ and gets back the value $f(u)$.  The running time for this
  operation is the encoding length of $u$ plus the encoding length of
  $f(u)$. 
\end{itemize}
If $\machine$ can access several oracles $O_{1},O_{2},\ldots$, we write
$\machine^{O_{1},O_{2},\ldots}$. If an algorithm $\machine$ gets a sample $x$
distributed according to the random variable $X$, we denote this as
$x\gets X$ and $x\gets \machine$ for the output of the randomized
algorithm. If $\machine$ is not randomized, i.e. it can only output a
single value for fixed inputs, we denote this by $x := \machine$ to
highlight this difference. If $S$ is a finite set, we denote the
uniform sampling of a random element $s$ of $S$ by $s\sgets
S$. 

If $\Pr$ is a probability distribution, we say
that $\Pr$ is a \myDef{efficiently sampleable distribution}, if there is a \ac{PPTM}
$\machine$ that gets no input such that the output distribution of
$\Pr$ and $\machine$ is the same. A sequence of probability
distributions $\Pr_{1},\Pr_{2},\ldots$ will also be denoted as
$\{\Pr_{n}\}_{n\in \nats}$ and is called an \myDef{distribution ensemble}. A
% distribution ensemble $\{\Pr_{n}\}_{n\in \nats}$ is called an \myDef{efficiently
%   sampleable ensemble}, if there is a \ac{PPTM} $\machine$ such that
% its upon the unary encoding of a number $n$ -- denoted by $1^{n}$ -- is
% the same as $\Pr_{n}$, \ie $\machine(1^{n})=\Pr_{n}$.
Similarly, a function $f\colon U\to V$ is a \myDef{efficiently
  computable function}, if there is a \ac{PPTM} $\machine$ such that
$\machine(u)=f(u)$ for all $u\in U$. 

\section{Cryptographic Primitives}
\label{sec:primitives}
We will make use of a wide range of cryptographic primitives ranging
from \emph{one-way functions} to \emph{public-key cryptosystems}. Most
of the definitions are taken from or inspired by the excellent textbook
of \citeauthor{lindell2007introduction}
\cite{lindell2007introduction}. 

Two
main approaches for the definition of cryptographic primitives exist in
the literature. In the first approach, the length of the key (also
called the \myDef{security parameter}) is treated as a
constant. Consequently, the running time of all involved algorithms are
also a constant. The typical assumption in this model is that a
primitive is $(t,\epsilon)$-secure, \ie the advantage of every attacker 
with running time $t$ against the primitive is at most $\epsilon$. This
line of work was first introduced by \citeauthor{bellare1997concrete}
and is commonly called \myDef{concrete security}
\cite{bellare1997concrete}. 
 The second approach -- the \myDef{asymptotic
  security} -- treats the security parameter as a variable and analyzes
the security of the primitives in an asymptotic way, \ie for large
enough values. We typically denote this variable with $\kappa$ and the
corresponding key $k\in \{0,1\}^{\kappa}$ with $k$.
To define security in this setting, the notion of
negligible functions is needed. A function $\negl\colon \nats\to [0,1]$
is called \myDef{negligible} if for every polynomial $p$, there is
$n_{0}\in \nats$ such that $\negl(n) < p(n)^{-1}$ for every
$n\geq n_{0}$. Hence, a negligible function decreases faster than the
inverse of every polynomial. 
\begin{example}
  Typical examples for negligible functions are $n\mapsto 2^{-n}$,
  $n\mapsto 1{.}01^{-n}$,  $n\mapsto 2^{-0{.}1n}$, but also $n\mapsto
  n^{-\log n}$. 
\end{example}
The typical assumption in the asymptotic security setting is now that
upon security parameter $\kappa$, every attacker that runs in time $p(\kappa)$ for
a polynomial $p$ only has a negligible advantage of $\negl(\kappa)$ to break the
primitive. 

While the concrete approach gives more concrete bounds, the analysis of
the security in the asymptotic approach is often more helpful in
understanding the underlying arguments. Additionally, it is unclear for
which parameters $t$ and $\epsilon$ we can treat a primitive as
"secure". As one can typically easily translate asymptotic bounds into
concrete bound and as we want to emphasize upon the arguments rather
than those concrete bounds, we have decided to use the asymptotic
approach in this work. For a more thorough discussion of this models,
see the textbook of \citeauthor{lindell2007introduction}
\cite[pp. 49-52]{lindell2007introduction}.  For an example of the
concrete approach in steganography see \eg
\cite{berndt2016pattern}. Throughout this chapter, let $\ell$, $\ell'$
and $L$ be polynomials.

\subsection*{Indistinguishability}
A main idea behind several cryptographic primitives is the notion of
\myDef{indistinguishability} of two objects $O$ and $O'$. We say that
$O$ and $O'$ are indistinguishable, if no \ac{PPTM} can distinguish
them. Note that $O$ and $O'$ may differ significantly, but only with
respect to non-polynomial properties. In the following, we look at a
simple example: The indistinguishability of two distribution ensembles. 

Let $P=\{P_{\kappa}\}_{\kappa\in \nats}$ and
$Q=\{Q_{\kappa}\}_{\kappa\in \nats}$ be two distribution ensembles. 

For a \ac{PPTM} $\DDist$ (the \myDef{distribution distinguisher}), the
\emph{advantage of $\DDist$ to distinguish $P$ and $Q$} is defined as
\begin{align*}
\Adv^{\dist}_{\DDist,P,Q}(\kappa)=\left|\Pr[\DDist^{P_{\kappa}}(1^{\kappa})=1]-\Pr[\DDist^{Q_{\kappa}}(1^{\kappa})=1]\right|,
\end{align*}
where $\DDist^{D}$ has the ability to get samples distributed accordingly to
$D$ in unit time. The insecurity of $P$ and $Q$ is defined as
\begin{align*}
   \InSec_{P,Q}^{\dist}(\kappa) = \max_{\DDist}\{\Adv_{\DDist,P,Q}^{\dist}(\kappa)\}.
\end{align*}

To give a more fine-grained security analysis, we also a more refined
version of $\InSec_{P,Q}^{\dist}(\kappa)$. For two polynomials $q$ and
$t$, we denote by 
\begin{align*}
   \InSec_{P,Q}^{\dist}(q,t,\kappa) = \max_{\DDist}\{\Adv_{\DDist,P,Q}^{\dist}(\kappa)\},
\end{align*}
where the maximum is taken over all distribution distinguisher that run
in time $t(\kappa)$ and make at most $q(\kappa)$ queries to their
distribution oracle. 



 In order to simplify our notation, we sometimes
identify a set $M$ with the uniform distribution on $M$. If $M$ and $N$
are two finite sets, we hence write
$ \DDist^{M},\DDist^{N}, \Adv^{\dist}_{\DDist,M,N},$ and $\InSec_{M,N}^{\dist}$ with the
meaning that $M$ and $N$ are uniformly distributed. Similarly, if
$F\colon M\to N$ is a function, we write $F(M)$ for the distribution on $N$
that arises, if the argument to $F$ is chosen uniformly from $M$. 

By the following well-known theorem (see
\eg~\cite[p.173]{goldreich2004foundations}) we get that the statistical
distance is a stronger measure than the computational
indistinguishability.


\begin{theorem}
\label{thm:statistically}
Let $\{P_{\kappa}\}_{\kappa\in \mathbb{N}}$ and
$\{Q_{\kappa}\}_{\kappa\in \mathbb{N}}$ be two distribution ensemblöes
ensembles on the same domains.
Then it holds that for all sufficiently large $\kappa$:
\begin{align*}
 \InSec^{\dist}_{P,Q}(q,t,\kappa)\leq q(\kappa)\cdot D_{S}(P_{\kappa},Q_{\kappa}).
\end{align*}
\end{theorem}




\subsection*{One-Way Functions}
A function $\algf\colon \{0,1\}^{*}\to \{0,1\}^{*}$ is called a
\myDef{one-way function}, if the following properties hold:
\begin{itemize}
\item For all $n\in
  \nats$ and all $x\in \{0,1\}^{n}$, we have $\ell(n)\leq |\algf(x)|\leq
  \ell'(n)$. 
\item The function $\algf$ is efficiently computable.
\item For every \ac{PPTM} $\Inv$ (the \emph{inverting algorithm}), there
  exists a negligible function $\negl$ such that for all sufficiently
  large $n\in \nats$,
  \begin{align*}
    \Pr[\Inv(\algf(x))\in \algf^{-1}(\algf(x))]\leq \negl(n),
  \end{align*}
where the probability is taken over the random choice of $x\sgets \{0,1\}^{n}$ and the
internal coin flips of $\Inv$.
\end{itemize}

A wide range of works shows that the existence of one-way functions is
the minimal assumption needed for cryptography, as most of the following
primitives can be constructed out of them
\cite[pp. 181-225]{lindell2007introduction}. 

\subsection*{Hash Functions}
In the following, we will often use \myDef{keyed functions} $f\colon
\{0,1\}^{*}\times \{0,1\}^{*}\to \{0,1\}^{*}$. The first parameter of
$f$ is then called the \emph{key} of the function. To simplify notation,
for each key $k\in \{0,1\}^{*}$, we define the function $f_{k}\colon
\{0,1\}^{*}\to \{0,1\}^{*}$ with $f_{k}(x)=f(k,x)$.

A cryptographic primitive typically consists of a keyed function $f$ and
a \myDef{generator} algorithm $\Gen_{f}$ that upon input $1^{\kappa}$
produces a suitable key $k$ of size $\kappa$ for
$f$. It might be useful for practical purposes to allow $\Gen_{f}$ to
output keys that are longer than $\kappa$, but we can safely ignore this
issue in this thesis.

A \myDef{hash function} $(\algh,\Gen_{\algh})$ is a pair of \acp{PPTM} such that
$\Gen_{\algh}$ upon input $1^{\kappa}$ produces a key $k\in
\{0,1\}^{\kappa}$. The keyed function $\algh$ takes the key $k\gets
\Gen_{\algh}(1^{\kappa})$ and a string $x\in L(\kappa)$ and produces a
string $\algh_{k}(x)$ of length $\ell(\kappa) < L(\kappa)$.

 In order to define the
"security" of this function, we first need to define a corresponding
\emph{experiment}. This is a typical approach in cryptography and
steganography: For a primitive $\Pi$, we define an experiment $E(\Pi)$,
that takes an "attacker" $A$. Whenever the probability that $A$ passes
(its \myDef{advantage} $\Adv_{A,\pi}(\kappa)$) 
is negligible we say that $\Pi$ is secure. 

As it should be hard for an
adversary to find two different elements $x\neq x'$ such that
$\algh_{k}(x)=\algh_{k}(x')$, we need to find a corresponding experiment. A
\myDef{collision finder} $\Fi$ for a hash function $(\algh,\Gen_{\algh})$
 is a \ac{PPTM} that upon input 
$k\in \supp(\Gen_{\algh}(1^{\kappa}))$ outputs two strings
$x,x'\in \{0,1\}^{L(\kappa)}$. Its goal is to pass the following
experiment:
\myAlgorithm{$\Coll_{\Fi,(\algh,\Gen_{\algh})}(\kappa)$}{Hash function $(\algh,\Gen_{\algh})$, Finder $\Fi$, key length $\kappa$}{
\State $k \gets \Gen_{\algh}(1^{\kappa})$
\State $(x,x') \gets \Fi(k)$
\If{$x\neq x'$ and $\algh_{k}(x)=\algh_{k}(x')$} 
   \AlgReturn{1}
\Else
 \AlgReturn{0}
\EndIf
}{Collision-Finding Experiment}

A hash function $\Pi=(\algh,\Gen_{\algh})$ is called
\myDef{collision resistant}, if for all collision finders $\Fi$, there is a
negligible function $\negl$ such that
\begin{align*}
 \Adv_{\Fi,(\algh,\Gen_{\algh})}^{\hash}(\kappa) := \Pr[\Coll_{\Fi,\Pi}(\kappa)=1]\leq \negl(\kappa).
\end{align*}

As collision resistant hash functions compresses an input of length
$L(\kappa)$ into a smaller value of length $\ell(\kappa) < L(\kappa)$, they
are often used to create short signatures of a longer bitstring.

\subsection*{Pseudorandom Functions}
For two numbers $\ell,\ell'\in \nats$, denote the \myDef{set of all function} from
$\{0,1\}^{\ell}$ to $\{0,1\}^{\ell'}$ by $\Fun(\ell,\ell')$. Clearly, in order to
specify a random element of $\Fun(\ell,\ell')$, one needs $2^{\ell}\times \ell'$
bits and we can thus not use completely random functions in an efficient
setting. We will thus use efficient functions that are indistinguishable
from completely random function. A \myDef{\ac{PRF}} $(\algf,\Gen_{\algf})$ is a
pair of \acp{PPTM} such that $\Gen_{\algf}$ upon input $1^{\kappa}$ produces
a key $k\in \{0,1\}^{\kappa}$. The keyed function $\algf$
takes the key $k\gets \Gen_{\algf}(1^{\kappa})$ and a bitstring $x$ of
length $\fin_{\algf}(\kappa)$ and produces a string $\algf_{k}(x)$ of length
$\fout_{\algf}(\kappa)$. An attacker, called \myDef{distinguisher} $\Dist$ upon
input $1^{\kappa}$ gets
oracle access to a function that is either completely random or equals
$\algf_{k}$ for a randomly chosen key $k$ and its goal is to distinguish
between those cases. Hence, for every distinguisher $\Dist$ there is a
negligible function $\negl$ such that
\begin{align*}
  &\Adv_{\Dist,(\algf,\Gen_{\algf})}^{\prf}(\kappa) :=\\ &\left| \Pr[\Dist^{\algf_{k}}(1^{\kappa})=1] -
  \Pr[\Dist^{f}(1^{\kappa})]\right| \leq \negl(\kappa),
\end{align*}
where $k\gets \Gen_{\algf}(1^{\kappa})$ and $f\sgets
\Fun(\fin_{\algf}(\kappa),\fout_{\algf}(\kappa))$. 

As usual, the maximal advantage of any warden against
$(\algf,\Gen_{\algf})$ is called the \myDef{\ac{PRF}-insecurity}
$\InSec_{(algf,\Gen_{\algf})}^{\prf}(\kappa)$ and defined as
\begin{align*}
  \InSec_{(\algf,\Gen_{\algf})}^{\prf}(\kappa) := \max_{\Dist}\{\Adv^{\prf}_{\Dist,(\algf,\Gen_{\algf})}(\kappa)\}.
\end{align*}

As for the distribution distinguisher, we also use more refined notion
of insecurity: If $q$ and $t$ are two polynomials, let
\begin{align*}
  \InSec_{(\algf,\Gen_{\algf})}^{\prf}(q,t,\kappa) := \max_{\Dist}\{\Adv^{\prf}_{\Dist,(\algf,\Gen_{\algf})}(\kappa)\},
\end{align*}
where the maximum is taken over all distinguishers that run in time
$t(\kappa)$ and make at most $q(\kappa)$ queries to their function
oracle. 

Furthermore, if
$\fin_{\algf}(\kappa)=\fout_{\algf}(\kappa)$ and if every $\algf_{k}$ is a permutation we
say that the pair $(\algf,\Gen_{\algf})$ is a \myDef{\ac{PRP}}.

Note that due to the definition of \acp{PRF}, they share \emph{all}
properties of totally random functions that one can test in polynomial
time (up to a negligible probability). A typical security analysis of a
protocol that uses \acp{PRF} thus starts with the analysis of the
protocol if one replaces the \ac{PRF} with a totally random
function. This modified protocol is then examined with
probability- or information-theoretic means to conclude something about
the behaviour of the modified protocol. By replacing the totally
random function with a \ac{PRF}, one can conclude that the behaviour of
the modified protocol and the behaviour of the original protocol are
very similar. This allows one to also use the results of the modified
protocol for the original protocol. 


In the chapters concerned with
non-efficient steganography, we will drop the requirement that $\algf$ is
efficient computable and say that such a pair $(\algf,\Gen_{\algf})$ is a
\myDef{non-efficient \ac{PRF}}. 


\subsection*{Signature Schemes}
A \myDef{signature scheme} $(\Gen,\Sign,\Vrfy)$ is a triple of
\acp{PPTM} such that the algorithm $\Gen(1^{\kappa})$ produces a pair $(\pk,\sk)$ of
keys with $|\pk| = \kappa$ and $|\sk| = \kappa$. We call $\pk$ a
\emph{public key} and $\sk$ a \emph{secret/private key}. The \myDef{signing
algorithm} $\Sign$ takes as input the secret key $\sk$, a \emph{message}
$m\in \{0,1\}^{\ell(\kappa)}$ and outputs a \myDef{signature} $\sigma\in
\{0,1\}^{*}$. The \myDef{verifying algorithm} $\Vrfy$ takes as input the
public key $\pk$, a message $m\in \{0,1\}^{\ell(\kappa)}$ and a signature
$\sigma\in \{0,1\}^{*}$. It outputs a bit $b$ with $b=1$ iff $\sigma \in
\supp(\Sign(\pk,m))$. We will typically treat $\Sign$ and $\Vrfy$ as
keyed functions and will thus also write $\Sign_{\sk}$ and $\Vrfy_{\pk}$
for the corresponding function, where the key is fixed. A \myDef{forger}
$\Forg$ is a \ac{PPTM} that upon input $\pk$ and oracle access to
$\Sign_{\sk}$ tries to produce a pair $(m,\sigma)$ such that
$\Vrfy_{\pk}(m,\sigma)=1$. Formally, this is defined via the following
experiment $\Sig-Forge$.

\myAlgorithm{$\Sig-Forge_{\Forg,(\Gen,\Sign,\Vrfy)}(\kappa)$}{Signature
 Scheme $(\Gen,\Sign,\Vrfy)$, Forger $\Forg$, length $\kappa$}{
\State $(\pk,\sk) \gets \Gen(1^{\kappa})$
\State $(m,\sigma) \gets \Forg^{\Sign_{\sk}}(\pk)$ 
\State Let $Q$ be the set of messages given to $\Sign_{\sk}$ by $\Forg$
\If{$m\not\in Q$ and $\Vrfy_{\pk}(m,\sigma)=1$}
   \AlgReturn{1}
\Else
 \AlgReturn{0}
\EndIf
}{Signature-Forging Experiment}

A signature scheme $(\Gen,\Sign,\Vrfy)$ is called \myDef{existentially
  unforgeable} if for every forger $\Forg$, there is a negligible
function $\negl$ such that
\begin{align*}
  \Adv^{\sig}_{\Forg,(\Gen,\Sign,\Vrfy)}(\kappa) :=
  \Pr[\Sig-Forge_{\Forg,\Pi}(\kappa)=1]\leq \negl(\kappa).
\end{align*}

As usual, the maximal advantage of any forger against
$(\Gen,\Sign,\Vrfy)$ is called the \myDef{sig-insecurity}
$\InSec_{(\Gen,\Sign,\Vrfy)}^{\sig}(\kappa)$ and defined as
\begin{align*}
  \InSec_{(\Gen,\Sign,\Vrfy)}^{\sig}(\kappa) := \max_{\Forg}\{\Adv^{\sig}_{\Forg,(\Gen,\Sign,\Vrfy)}(\kappa)\}.
\end{align*}


Note that this definition of security implies that a existentially
unforgeable signature scheme is \emph{publicly verifiable} and has the
property of \emph{non-repudiation} \cite{lindell2007introduction}, two
important aspects that we will also make use of.

\subsection*{Symmetric Encryption Schemes}
A \myDef{\ac{SES}} $(\Gen, \Enc,\Dec)$ is a triple of
\acp{PPTM} such that $\Gen(1^{\kappa})$ produces a key
$k\in \{0,1\}^{\kappa}$. The \myDef{encryption
  algorithm} $\Enc$ takes as input the key $k$ and a \emph{plaintext}
$m\in \{0,1\}^{\ell(\kappa)}$ and outputs a \emph{ciphertext}
$c\in \{0,1\}^{*}$. The \myDef{decryption algorithm} $\Dec$ takes as
input the key $k$ and a ciphertext $c$ and outputs a plaintext
$m\in \{0,1\}^{\ell(\kappa)}$. In order to make sure that the decryption
is successful, we assume that there exists a negligible function $\negl$
such that 
\begin{align*}
\max_{m\in \{0,1\}^{\ell(\kappa)}, k\gets \Gen(1^{\kappa})}\{\Pr[\Dec(k,\Enc(k,m))\neq m]\}\leq\negl(\kappa).  
\end{align*}



An \myDef{attacker} $(\Att_{1},\Att_{2})$ on
the encryption scheme is a pair of \acp{PPTM}. In the \emph{first
  round}, the algorithm $\Att_{1}$ produces upon input of $1^{\kappa}$
and with oracle access to $\Enc_{k}$ two messages $m_{0},m_{1}\in
\{0,1\}^{\ell(\kappa)}$. In the \emph{second round}, $\Att_{2}$ is given
the encryption of $m_{b}$ and should decide whether $b=0$ or
$b=1$. This security notion is known as security against \acp{CPA}. 
Formally, this is defined via the following experiment
$\CPA-Dist$.

\myAlgorithm{$\CPA-Dist_{(\Att_{1},\Att_{2}),(\Gen,\Enc,\Dec)}(\kappa)$}{\ac{SES}
  $(\Gen,\Enc,\Dec)$, Attacker $(\Att_{1},\Att_{2})$, length $\kappa$}{
\State $k \gets \Gen(1^{\kappa})$
\State $(m_{0},m_{1},s) \gets \Att_{1}^{\Enc_{k}}(1^{\kappa})$
\Comment{$s$ contains \emph{state information}}
\State $b \sgets \{0,1\}$
\State $c \gets \Enc_{k}(m_{b})$
\State $b' \gets \Att_{2}^{\Enc_{k}}(1^{\kappa},c,s)$
\If{$b=b'$}
   \AlgReturn{1}
\Else
 \AlgReturn{0}
\EndIf
}{Chosen-Plaintext-Attack experiment}

A \ac{SES} $(\Gen,\Enc,\Dec)$ is \myDef{\acs{CPA}-secure} if for every
attacker $(\Att_{1},\Att_{2})$, there is a negligible function $\negl$
such that 
\begin{align*}
  &\Adv_{(\Att_{1},\Att_{2}),(\Gen,\Enc,\Dec)}^{\cpa}(\kappa) :=\\
  &\left|\Pr[\CPA-Dist_{(\Att_{1},\Att_{2}),(\Gen,\Enc,\Dec)}(\kappa)=1]
    - \frac{1}{2}\right|\leq \negl(\kappa).
\end{align*}

As usual, the maximal advantage of any attacker against
$(\Gen,\Enc,\Dec)$ is called the \myDef{\ac{CPA}-insecurity}
$\InSec_{(\Gen,\Enc,\Dec)}^{\cpa}(\kappa)$ and defined as
\begin{align*}
  \InSec_{(\Gen,\Enc,\Dec)}^{\cpa}(\kappa) := \max_{(\Att_{1},\Att_{2})}\{\Adv^{\cpa}_{(\Att_{1},\Att_{2}),(\Gen,\Enc,\Dec)}(\kappa)\}.
\end{align*}

Similarly to above, a more refined version of the insecurity exists. If
$q$ and $t$ are two polynomials, let
\begin{align*}
  \InSec_{(\Gen,\Enc,\Dec)}^{\cpa}(q,t,\kappa) := \max_{(\Att_{1},\Att_{2})}\{\Adv^{\cpa}_{(\Att_{1},\Att_{2}),(\Gen,\Enc,\Dec)}(\kappa)\},
\end{align*}
where the maximum is taken over all attackers that run in time
$t(\kappa)$ and make at most $q(\kappa)$ queries to the encryption
oracle. 

An even stronger security notion is the notion of security against \acfp{CPA+}, where the
attacker $\Att_{1}$ outputs a single message $m$ and the string $c$ is either
$\Enc_{k}(m)$ ($b=0$) or a completely random bitstring of length
$|\Enc_{k}(m)|$ ($b=1$). The goal of $\Att_{2}$ is to reconstruct the
bit $b$ from $c$. Denote this modification of $\CPA-Dist$ by
$\CPAD-Dist$.  Informally, this implies that the ciphertext
constructed by the \ac{SES} are indistinguishable from random strings. 
A \acl{SES} $(\Gen,\Enc,\Dec)$ is \myDef{\acs{CPA+}-secure} if for every
attacker $(\Att_{1},\Att_{2})$, there is a negligible function $\negl$
such that 
\begin{align*}
  &\Adv_{(\Att_{1},\Att_{2}),(\Gen,\Enc,\Dec)}^{\cpad}(\kappa) :=\\
  &\left|\Pr[\CPAD-Dist_{(\Att_{1},\Att_{2}),(\Gen,\Enc,\Dec)}(\kappa)=1]-\frac{1}{2}\right|\leq
    \negl(\kappa).
\end{align*}

As usual, the maximal advantage of any attacker against
$(\Gen,\Enc,\Dec)$ is called the \myDef{\ac{CPA+}-insecurity}
$\InSec_{(\Gen,\Enc,\Dec)}^{\cpad}(\kappa)$ and defined as
\begin{align*}
  \InSec_{(\Gen,\Enc,\Dec)}^{\cpad}(\kappa) := \max_{(\Att_{1},\Att_{2})}\{\Adv^{\cpad}_{(\Att_{1},\Att_{2}),(\Gen,\Enc,\Dec)}(\kappa)\}.
\end{align*}

Clearly, \acs{CPA+}-security implies \acs{CPA}-security, but the other
implication is not true: For example, the encryption algorithm $\Enc$
may always appends a certain string at the end of each
ciphertext. Fortunately, most known \ac{CPA}-secure \ac{SES} are also
\ac{CPA+}-secure. 

\subsubsection*{Random Counter Mode}
We will sometimes use simple, yet incredibly useful \ac{SES} called the
\myDef{random counter mode}. Let $(\algf,\Gen_{\algf})$ be a \ac{PRF} that maps
input strings of length $\fin_{\algf}(\kappa)$ into output strings of
length $\fout_{\algf}(\kappa)$. The following algorithms then yields a
\ac{SES} $(\Gen^{\algf},\Enc^{\algf},\Dec^{\algf})$:
\begin{itemize}
\item The generator algorithm $\Gen^{\algf}$ simply uses $\Gen_{\algf}$ to create a
  symmetric key $k$, \ie $\Gen^{\algf}(1^{\kappa})=\Gen_{\algf}(1^{\kappa})$. 
\item The encryption algorithm works as follows for messages
  $m=m_{1}m_{2}\ldots m_{n(\kappa)}$ with $m_{i}\in \{0,1\}^{\fin_{\algf}(\kappa)}$:

\myAlgorithm{$\Enc^{\algf}$}{key $k$, $m=m_{1}m_{2}\ldots m_{n(\kappa)}\in
  \{0,1\}^{n(\kappa)\cdot \fin_{\algf}(\kappa)}$}{
\State $\kappa := |k|$
\State $r \sgets \{0,1\}^{\fin_{\algf}(\kappa)}$ \Comment{$r$ is treated as
  string \emph{and} number}
\For{$i=1,\ldots,n(\kappa)$}
\State $c_{i} := \algf_{k}(r+i \bmod 2^{\fin_{\algf}(\kappa)})\oplus m_{i}$
\EndFor
\State \AlgReturn $r,c_{1},c_{2},\ldots,c_{n(\kappa)}$
}{Random Counter Mode Encryption}
\item Similarly, the decryption inverts the encryption:
\myAlgorithm{$\Dec^{\algf}$}{key $k$, $c=c_{0}c_{1}\ldots c_{n(\kappa)}\in
  \{0,1\}^{(n(\kappa)+1)\cdot \fout_{\algf}(\kappa)}$}{
\State $\kappa := |k|$
\State $r := c_{0}$
\For{$i=1,\ldots,n(\kappa)$}
\State $m_{i} := \algf_{k}(r+i \bmod 2^{\fin_{\algf}(\kappa)})\oplus c_{i}$
\EndFor
\State \AlgReturn $m_{1},m_{2},\ldots,m_{n(\kappa)}$
}{Random Counter Mode Decryption}
\end{itemize}

Clearly, every ciphertext $c=\Enc^{\algf}(k,m)$ is decoded correctly. Concerning
the security, \citeauthor{bellare1997concrete} already proved the
following theorem in \cite{bellare1997concrete}, where they called this
construction the \emph{XOR-Scheme}.

\begin{theorem}[Theorem 13 in the full version of
  \cite{bellare1997concrete}]
\label{thm:counter_mode}
  If the pair $(\algf,\Gen_{\algf})$ is a \acl{PRF}, the \acl{SES}
  $(\Gen^{\algf},\Enc^{\algf},\Dec^{\algf})$ is \ac{CPA+}-secure. More
  precisely if the \ac{SES} works on messages of length $n(\kappa)\cdot
  \fin_{\algf}(\kappa)$, we have for polynomials $q$ and $t$ that
  \begin{align*}
    &\InSec_{(\Gen^{\algf},\Enc^{\algf},\Dec^{\algf})}^{\cpa}(q,t,\kappa)\leq\\
    &\frac{q(\kappa)\cdot (n(\kappa)+1)\cdot \fout_{\algf}(\kappa)\cdot (q(\kappa)-1)}{\fin_{\algf}(\kappa)\cdot 2^{\fout_{\algf}(\kappa)}}+2\InSec^{\prf}_{(\algf,\Gen_{\algf})}(2qn,t,\kappa).
  \end{align*}
\end{theorem}

\subsection*{Public-Key Encryption Schemes}
While \acp{SES} are very useful, the problem of the key management
remains complicated. If $n$ parties want to communicate via a \ac{SES},
each pair $i,j\in \{1,\ldots,n\}$ needs to share a key $k_{i,j}$. Hence,
$\binom{n}{2}$ keys are needed if every party wants to communicate with
every other party. And furthermore, those $\binom{n}{2}$ somehow need to
be exchanged over a secure communication channel before the actual
communication may take part. In order to remedy these problems,
\citeauthor{diffie1976publickey} introduced the notion of
\emph{public-key cryptography} in their groundbreaking work
\cite{diffie1976publickey}. 

A \myDef{\ac{PKES}} $(\Gen,\PKEnc,\PKDec)$ is a triple of \acp{PPTM}
such that $\Gen(1^{\kappa})$ produces a pair of keys $(\pk,\sk)$ with
$|\pk| = \kappa$ and $|\sk| =  \kappa$. The key $\pk$ is called the
\myDef{public key} and the key $\sk$ is called the \myDef{secret key}
(or \emph{private key}). While $\pk$ will be publicly available to all
parties, the secret key $\sk$ must be kept private by its owner. The
\myDef{public-key encryption algorithm} $\PKEnc$ takes as input the
public key $\pk$ and a plaintext $m\in \{0,1\}^{\ell(\kappa)}$ and
outputs a ciphertext $c\in \{0,1\}^{*}$. The \myDef{public-key decryption
  algorithm} $\PKDec$ takes as input the secret key $\sk$ and the
ciphertext $c$ and produces a plaintext $m\in
\{0,1\}^{\ell(\kappa)}$. In order to make sure that the decryption
is successful, we assume that there exists a negligible function $\negl$
such that 
\begin{align*}
\max_{\substack{m\in \{0,1\}^{\ell(\kappa)},\\(\pk,\sk)\gets \Gen(1^{\kappa})}}\{\Pr[\PKDec(\sk,\PKEnc(\pk,m))\neq
m]\}\leq\negl(\kappa).   
\end{align*}



While an attacker against a \ac{SES} was given oracle access to the
encryption algorithm, this is not needed in the public-key setting:
Everyone knows the public key $\pk$ needed to encrypt messages. On the
other hand, research has shown that the security requirements for
\acp{PKES} are much higher. Informally, we will allow an attacker to
first choose a message that should be encrypted. In the next step, the
attacker is allowed to insert arbitrary ciphertexts into the
communication and watch Bob's behaviour upon receiving those
texts. Formally we equip an attacker with a
\emph{decryption oracle} in order to perform this kind of attack. 

An \myDef{public-key attacker}
$(\Att_{1},\Att_{2})$ on the \ac{PKES} is a pair of \acp{PPTM}. In the
\emph{first round}, the algorithm $\Att_{1}$ produces upon input $\pk$
and with oracle access to $\PKDec_{\sk}$ two messages $m_{0},m_{1}\in
\{0,1\}^{\ell(\kappa)}$. A random bit $b\sgets \{0,1\}$ is then chosen
and in the \emph{second round}, $\Att_{2}$ is given
the encryption $c$ of $m_{b}$ and should decide whether $b=0$ or
$b=1$. While we still allow $\Att_{2}$ to have oracle access to the
decoding algorithm $\PKDec_{\sk}$, clearly we must forbid that it
uses it to decrypt $c$. 
This security notion is known as security
against \acp{CCA}. Formally, this is defined via the following experiment
$\CCA-Dist$. 


\myAlgorithm{$\CCA-Dist_{(\Att_{1},\Att_{2}),(\Gen,\PKEnc,\PKDec)}(\kappa)$}{\ac{PKES}
  $(\Gen,\PKEnc,\PKDec)$, Attacker $(\Att_{1},\Att_{2})$, length $\kappa$}{
\State $(\pk,\sk) \gets \Gen(1^{\kappa})$
\State $(m_{0},m_{1},s) \gets \Att_{1}^{\PKDec_{\sk}}(\pk)$
\Comment{$s$ contains \emph{state information}}
\State $b \sgets \{0,1\}$
\State $c \gets \PKEnc_{\pk}(m_{b})$
\State $b' \gets \Att_{2}^{\PKDec_{\sk}}(pk,c,s)$
\If{$\Att_{2}$ queries $\PKDec_{\sk}(c)$ or $b\neq b'$}
 \AlgReturn{0}
\Else
   \AlgReturn{1}
\EndIf
}{Chosen-Ciphertext-Attack Experiment}

A \ac{PKES} $(\Gen,\PKEnc,\PKDec)$ is called \myDef{\acs{CCA}-secure}, if for
every attacker $(\Att_{1},\Att_{2})$, there is a negligible function
$\negl$ such that
\begin{align*}
  &\Adv_{(\Att_{1},\Att_{2}),(\Gen,\PKEnc,\PKDec)}^{\cca}(\kappa) :=\\
  &\left|\Pr[\CCA-Dist_{(\Att_{1},\Att_{2}),(\Gen,\PKEnc,\PKDec)}(\kappa)=1]
    - \frac{1}{2}\right|\leq \negl(\kappa).
\end{align*}

As usual, the maximal advantage of any attacker against
$(\Gen,\PKEnc,\PKDec)$ is called the \myDef{\ac{CCA}-insecurity}
$\InSec_{(\Gen,\PKEnc,\PKDec)}^{\cca}(\kappa)$ and defined as
\begin{align*}
  \InSec_{(\Gen,\PKEnc,\PKDec)}^{\cca}(\kappa) := \max_{(\Att_{1},\Att_{2})}\{\Adv^{\cca}_{(\Att_{1},\Att_{2}),(\Gen,\PKEnc,\PKDec)}(\kappa)\}.
\end{align*}

As in the symmetric key, this notion of security can also be
strengthened to security against \acfp{CCA+}, where the attacker needs
to distinguish the ciphertext of a chosen message ($b=0$) from a
completely random bitstring ($b=1$) of corresponding length
$|\PKEnc_{\pk}(m)|$. Denote this modification of $\CCA-Dist$ by
$\CCAD-Dist$. This
implies that the output of the \ac{PKES} is indistinguishable from
random strings. A \ac{PKES} $(\Gen,\PKEnc,\PKDec)$ is
\myDef{\acs{CCA+}-secure} if for every attacker $(\Att_{1},\Att_{2})$,
there is a negligible function $\negl$ such that
\begin{align*}
  &\Adv_{(\Att_{1},\Att_{2}),(\Gen,\PKEnc,\PKDec)}^{\ccad}(\kappa) :=\\
  &\left|\Pr[\CCAD-Dist_{(\Att_{1},\Att_{2}),(\Gen,\PKEnc,\PKDec)}(\kappa)=1]-\frac{1}{2}\right|\leq
    \negl(\kappa).
\end{align*}
The \myDef{\ac{CCA+}-insecurity} is defined as
\begin{align*}
  \InSec_{(\Gen,\PKEnc,\PKDec)}^{\ccad}(\kappa) := \max_{(\Att_{1},\Att_{2})}\{\Adv^{\ccad}_{(\Att_{1},\Att_{2}),(\Gen,\PKEnc,\PKDec)}(\kappa)\}.
\end{align*}


%%% Local Variables:
%%% TeX-master: "../main"
%%% End:
