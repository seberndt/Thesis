\chapter{Universal Non-Efficient Secret-Key Steganography}
\label{chap:universal_nonefficient_secret}
\begin{flushright}{\slshape    
Time is on my side, yes it is.} \\ \medskip
    --- The Rolling Stones
\end{flushright}

As already noted in \autoref{chap:models}, in the steganographic model
introduced by \citeauthor{hopper2009provably} in
\cite{hopper2009provably}, an imbalance between Alice and Warden was
introduced: While the running time of Warden is restricted to a
polynomial in $\kappa$, the running time of Alice is not restricted at
all. Note that this imbalance is present in several works, \eg
\cite{hopper2002provably, hopper2009provably, hopper2004toward} and also
used therein. In this chapter, we will focus on the following questions
raised by these works and by the work of \citeauthor{dedic2009lower}
in \cite{dedic2009lower}:
\begin{enumerate}
\item What is the relationship between cryptography and steganography?
\item Can one use this imbalance to construct an unconditionally secure
  stegosystem?
\item What is the tradeoff between  the rate and the query
  complexity of a secure and reliable stegosystem?
\end{enumerate}

The main results of this chapter are the following two theorems that
prove the existence of rate-efficient \emph{unconditional (but inefficient)} secure stegosystems
and \emph{unconditionally} relate the query complexity and the transmission
rates of all universal stegosystems thereby improving upon results in
\cite{hopper2009provably} and \cite{dedic2009lower}:

\begin{theorem}[Informal]\label{th:main:existexp}
For every $1>\alpha_{1}\geq \alpha_{2}>0$,
there exists a non-efficient stegosystem $\Steg$, with security parameter $\kappa$, 
such that for every channel $\chan$ with
min-entropy $\minent(\chan,n(\kappa))> 2\cdot
\kappa^{\alpha_{2}}$ and documents of length $\kappa^{\alpha_{1}}$,
 the stegosystem~$\Steg$ 
\begin{itemize}
\item  hides an $\ell(\kappa) \cdot \kappa^{\alpha_{2}}$-bit message
  into $\ell(\kappa)$ documents ($\rate_{\Steg,\chan}(\kappa)=\kappa^{\alpha_{2}}$)
\item 
  takes $\query_{\Steg,\chan}(\kappa)\le \kappa 2^{\kappa^{\alpha_{2}}} $ samples per documents, and 
\item 
 has insecurity and  unreliability negligible in $\kappa$ 
 (if $\chan$ does  not break the used hard functions):
  \begin{align*}
  \InSec_{\Steg,\chan}^{\sscha}(\kappa) + \unrel(\kappa) \ \le \  \negl(\kappa). 
  \end{align*}
\end{itemize}
\end{theorem}

\begin{theorem}[Informal]\label{th:main:lower:bound}
  There exists  a  channel $\chan$ such that for every universal 
  stegosystem $\Steg$, it holds that
  \begin{align*} 
    &\InSec_{\Steg,\chan}^{\sscha}(\kappa) + \unrel_{\Steg,\chan}(\kappa) \ge  \\
    &\frac{1}{2}-  \frac{e\cdot \query_{\Steg,\chan}(\kappa)}{2^{\rate_{\Steg,\chan}(\kappa)}}  -o(1).
  \end{align*}
\end{theorem}



A preliminary version of the results of this chapter was published as
\cite{berndt2016optimal}. In the next Section, we discuss our first
question regarding the relationship between steganography and
cryptography. We will then discuss the relevant literature for this
chapter and prove our two main results.


\section{The Relationship Between Steganography and Cryptography}
Although there is a strong connection between these areas,
steganography is \emph{not} cryptography.
Our example below shows even more, namely that 
polynomial-time bounded steganography is \emph{not} cryptography.
A commonly heard argument for the premise that steganography is 
cryptography goes as follows: 
{\advance\leftmargini 1em
\begin{quote}
  Let $m$ and $m'$ be two different secret messages and 
  $d$ and $d'$ be stego-documents which embed $m$, resp. $m'$. 
  If the distributions of $d$ and $d'$ are indistinguishable 
  from the distribution of the cover-documents,
  then by the triangle-inequality, the distributions of $s$ and $s'$
  are also indistinguishable. Hence, a secure
  stegosystem is also a secure cryptosystem.
\end{quote}}
While the argument concerning the triangle-inequality is true, one can
not simply use the stegosystem as a cryptosystem, as the stegosystem
needs access to samples from the channel. Arguably, the most researched
channel is those of natural digital pictures (say in the \acs{JPEG} format). A
typical stegosystem for this channel
takes a sample picture and
modifies it in a way that is not detectable. A cryptosystem that
simulates this stegosystem thus needs a way to get a sample
picture. But the standard definition of cryptosystems does \emph{not} assume
such access and it is highly unlikely that an efficient 
algorithm to simulate sampling
for this channel can be constructed.  We will note later on
that ignoring this access leads to misunderstandings, e.g. in the often
cited work \cite{hopper2009provably} of \citeauthor{hopper2009provably}.

Beside providing a rigorous definition for computationally secure
steganography, the main contribution of 
\cite{hopper2009provably} is demonstrating that a
widely believed complexity-theoretic assumption \textendash\ the
existence of one-way functions \textendash\ and access to a channel
oracle are both necessary and sufficient conditions for the existence of
secure and reliable steganography:

\begin{theorem}[{\cite[Corollary 1]{hopper2009provably},informal}]
\label{th:equiv:sec:stego:OWF}
Relative to an oracle for channel~$\chan$,
secure (and reliable) stegosystems exist if and only if  
one-way functions exist.
% and secure steganography are equivalent. 
\end{theorem}

This claim is now widely circulated in the literature.
In her handbook~\cite[p.~101]{fridrich2009steganography} on steganography, Fridrich writes: 
{\advance\leftmargini 1em
\begin{quote}
  ``One of the most intriguing implications of this complexity-theoretic
  view of steganography is the fact that secure stegosystems exist
  if and only if secure one-way (hash) functions exist [...]''. %under the assumption  
\end{quote}}

While one direction of \autoref{th:equiv:sec:stego:OWF} \textendash\
namely that one can construct secure stegosystems from one-way functions
\textendash\ is correct, as argued in \autoref{sec:rejsam}, we will
prove that steganography does not necessarily implies the existence of
one-way functions. The following theorem of
\citeauthor{hopper2009provably} in \cite{hopper2009provably} argues that
one-way functions are necessary for secure steganography:

\begin{theorem}[\cite{hopper2009provably}, informal]
\label{th:hopper:steg:impl:one-way}
For all channels $\chan$ it is true: if secure and reliable steganography
for $\chan$ exists then there exist one-way functions relative to an
oracle for $\chan$.\footnote{\citeauthor{hopper2009provably}  prove even stronger result
  using a weaker notion of security. Theorem~9 in
  \cite{hopper2009provably} says that if there is a stegosystem $\Steg$
  that is \acs{SS-KHA}-$D$-$\chan$ secure for some hiddentext distribution
  $D$ and some channel $\chan$, then there exists a pseudorandom
  generator, relative to an oracle for $\chan$.}
\end{theorem}

Combining \autoref{th:main:existexp} with
\autoref{th:equiv:sec:stego:OWF}
one would conclude that (1) relative to
an oracle for channel $\chan$ one-way functions exist and much more
startling, that (2) \emph{one-way functions exist in the standard
  model}, i.e., without assuming oracle access to the channel~$\chan$. As
a proof on the existence of one-way functions seems to be far away from
our current knowledge, one must wonder at the validity of
\autoref{th:equiv:sec:stego:OWF}. Indeed, we found errors in the proof
of \autoref{th:hopper:steg:impl:one-way} which consequently do not
allow to conclude \autoref{th:equiv:sec:stego:OWF}.


There are three issues concerning this proof.

\begin{itemize}
\item  \myNote{time complexity} Firstly, the time complexity of the
proposed \emph{\ac{FEG}}, a kind of oracle used in
the construction and the use of relativized primitives. The aim was to
provide an algorithm for an \ac{FEG}, assuming the existence of a stegosystem
$\Steg$ that is \acs{SS-KHA}-$D$-$\chan$ secure for some hiddentext distribution
$D$ and some channel $\chan$ (for the exact definitions,
see~\cite{hopper2009provably}). 

The proposed construction for the \ac{FEG} uses, as a subroutine, the encoder
of $\Steg$ having an oracle access to $\chan$.  Since no restrictions on the
running time of the encoder are given, it does not follow that the
obtained algorithm for the \ac{FEG} is bounded by a polynomial. This problem can
be fixed by assuming that the stegosystem runs in polynomial time.
Note, however, that making the assumption of polynomial time complexity 
for stegosystems, the claim of \cite[Section~4.3]{hopper2009provably}
concerning rate-optimality is false, as the proposed system requires
exponential time. By proving \autoref{th:main:existexp}, we will show
that the assumption on the running time of a stegosystem are very
relevant. 


\item \myNote{randomization} Secondly, according to the definition, an \ac{FEG} is a \emph{function}.
However, the \ac{FEG} relative to an
oracle $\chan$ does not seem to be deterministic, as it sometimes returns
the samples generated by the channel oracle. This does not seem to be
fixable easily, but one can make use of randomized cryptographic
primitives in order to give an alternative proof.

\item \myNote{channel access} The third obstacle still remains: In order to construct a
cryptographic primitive out of a stegosystem, one needs to simulate the
access to the channel oracle. If this simulation can be carried out in
polynomial time, the constructed primitive is indeed efficient. But, as
discussed in \autoref{sec:relativized}, such an assumption is quite
artificial. And indeed, if the channel oracle can not be simulated in
polynomial time, the constructed cryptographic primitive is not
efficient. It seems that the only remedy to this is to define another
way of \emph{relativized primitives}, where the primitive has also
access to the channel oracle as in this work.
\end{itemize}

\section{Known Upper and Lower Bounds on the Security of the Rejection
  Sampling Stegosystem}
We use this section to present the relevant literature concerned with
upper and lower bounds for the rejection sampling stegosystem.
In the following, let $(\algf,\Gen_{\algf})$ be a \acl{PRF} and
$\Pi=(\Gen,\Enc,\Dec)$ be a \acl{SES}.  
\subsection*{Upper Bounds}
As discussed in
\autoref{thm:rejsam:secure} in \autoref{sec:rejsam}, the rejection
sampling system $\RejSam^{\algf,\Pi}$ with rate $\fout_{\algf}(\kappa)$ fulfills the following
inequalities
\begin{align*}
    & \InSec_{\RejSam^{\algf,\Pi},\chan}(\kappa) \leq \Phi_{\chan}^{\algf,\Pi}(\kappa\cdot
    2^{\fout_{\algf}(\kappa)}), \text{ and}\tag{*}\label{eq:insec:hopper}\\
  & \unrel_{\RejSam^{\algf,\Pi},\chan}(\kappa) \leq \Phi_{\chan}^{\algf,\Pi}(\kappa\cdot 2^{\fout_{\algf}(\kappa)}),
\end{align*}
with
\begin{align*}
  &\Phi_{\chan}^{\algf,\Pi}(t)  :=\\
  &\InSec^{\prf}_{(\algf,\Gen_{\algf}),\chan}(t)+\InSec^{\cpad}_{\Pi,\chan}(t)+\negl(\kappa)+2^{\fout_{\algf}(\kappa)-\minent(\chan,n(\kappa))}
\end{align*}
for a negligible function $\negl$. 

Hence, the system is reliable and secure if and only if the term
$\Phi_{\chan}^{\algf,\Pi}(\kappa\cdot 2^{\fout_{\algf}(\kappa)})$ is negligible. We
notice that, if the transmission rate exceeds the logarithm of the key
length $\kappa$, then the proofs provided in~\cite{hopper2009provably}
do not guarantee that unreliability and insecurity (recall, even against
polynomial-time bounded warden) of the proposed stegosystems are
negligible.

More precisely, in case the number of bits $\fout_{\algf}(\kappa)$
embedded in a single document grows asymptotically faster than
$\log \kappa$, the term $\Phi_{\chan}^{\algf,\Pi}(\kappa\cdot
2^{\fout_{\algf}(\kappa)})$ is not guaranteed to be negligible, as the
two terms
$\InSec^{\prf}_{(\algf,\Gen_{\algf})}(\kappa\cdot
2^{\fout_{\algf}(\kappa)})$ and $\InSec^{\cpad}_{\Pi}(\kappa\cdot
2^{\fout_{\algf}(\kappa)})$ are not
guaranteed to be negligible in $\kappa$ even if the existence of
\acp{PRF} and \acp{SES} is assumed. This is due to the fact that one
assumes security of \acp{PRF} and \acp{SES} against polynomial-time
attacker and the term $\kappa 2^{\fout_{\algf}(\kappa)}$ is super-polynomial for
$\fout_{\algf}(\kappa) \in \omega(\log \kappa)$.
Thus, if a channel $\chan$ allows to embed up to 
$\poly(\kappa)$ bits per document, \ie if its min-entropy $\minent(\chan,n(\kappa))$ is very high, 
the stegosystem of \citeauthor{hopper2009provably} is not scalable to
meet the optimal rate:
for any $\fout_{\algf}(\kappa)\le \poly(\kappa)$ its query complexity is  
$\query_{\RejSam^{\algf,\Pi},\chan}(\kappa)=\kappa 2^{\fout_{\algf}(\kappa)}$  but its 
insecurity and unreliability is guaranteed negligible only for very low
rates 
$\fout_{\algf}(\kappa) \in  \landauO(\log \kappa)$.
We illustrate this in \autoref{fig:queries:vs:rate}.


\begin{figure}[ht]
  \centering
\begin{tikzpicture}[xscale=0.5, yscale=0.5]
  \draw[-{>[scale=2]}] (0,0) to (0,5.3);
  \draw[-{>[scale=2]}] (0,0) to (12.2,0);
  \node[align=center] at (-.1,4.5) [left] {\#queries $q$\\(log scale)};
  \node at (11,-1.2) [above] {rate $r$};

  \draw (3,-.1) to (3,.1);
  \node at (3,-1.2) [above] {$\log \kappa$};
  \draw (6,-.1) to (6,.1);
  \node at (6,-1.2)  [above] {$\sqrt{\kappa}$};
  \draw (9,-.1) to (9,.1);
  \node at (9,-1.2)  [above]  {$\kappa^{1-\delta}$};
%  \draw[dashed,color=orange] (9,0) to (9,4);

  \draw (-.1,3) to (.1,3);
  \node at (-.1,3) [left] {$\poly(\kappa)$};

% Dedic
  \draw[dashed] (3,0) to (3,4);
  \draw[domain=0:3,smooth,color=black!20!white] plot ({\x},{\x});
  \node[color=black!40!white,fill=white] at (9,4.0) {$\InSec(\kappa)+\unrel(\kappa) \le \epsilon$};

  \draw[fill=red!20!white]  (0,0) -- (3,3) -- (3,0) -- cycle;
%  \draw[domain=3:10,smooth,color=red!40!white] plot ({\x},{3*exp(3-\x)});
  % \fill[domain=3:10,fill=red!40!white] (3,0) -- plot ({\x},{3*exp(3-\x)}) --
  % (10,0) -- cycle;


  \draw[domain=0:3,smooth,color=green!60!black,very thick] plot ({\x},{\x});

  % \draw[domain=3:9,smooth,color=green!60!black,very thick] plot ({\x},{\x});

% Open
\node at (6,2) {\huge ?};

% \node[draw=black,thick,rounded corners=2pt,below left=2mm,fill=white] at
% (15.5,9.5) {%
%   \begin{tikzpicture}[node distance=.0cm]
%   \node[color=red!40!white] (dedic) {Not possible [Dedi{\'c}]};    
%   \node[color=blue!30!white, below = of dedic] (hopper) {Achievable [Hopper]};    
%   \node[color=green!60!black, below = of hopper] (we) {Achieveable [this
%     work]};    
%   \node[color=orange, below = of we] (it) {Information-theoretic upper bound};    
%   \end{tikzpicture}
% };

\end{tikzpicture}
\caption{Known results (under cryptographic assumptions): the green line shows 
the dependence between the rate and number of queries to ensure negligible insecurity 
and unreliability of the system of \citeauthor{hopper2009provably}
(\autoref{thm:rejsam:secure}). 
This bound is sharp: any system of rate and with number of queries % complexity 
in the red area is insecure or unreliable
(due to \autoref{thm:lower_bound_crypto} by \citeauthor{dedic2009lower}).
The situation for $\fout_{\algf}(\kappa)\in \omega(\log \kappa) $ has remained open, so far.
}
\label{fig:queries:vs:rate}
\end{figure}

Using the rejection sampling technique, 
\citeauthor{dedic2009lower} constructed two new universal
stegosystems with upper bounds on the insecurity and unreliability
similar to those of \autoref{thm:rejsam:secure} in their work
\cite{dedic2009lower}. 
Similarly to the system of \citeauthor{hopper2009provably}, if the
number of bits $b$ per document grows asymptotically faster than
$\log \kappa$, the security of the system is not guaranteed, even if the
encoder and decoder use \acp{PRF}.
Thus, also the results given in~\cite{dedic2009lower} do not
guarantee the existence of universal steganography of 
negligible unreliability and insecurity in case $b \in \omega(\log \kappa)$.

\subsection*{Lower Bounds}

In \cite{dedic2009lower}, \citeauthor{dedic2009lower} prove (under cryptographic
assumptions) the existence of channels
such that the number of samples the encoder of any
secure and reliable universal stegosystem must obtain from those 
channels is exponential in the number of bits embedded 
per document. In our terms, their result can be stated as:
\begin{theorem}[{\cite[Theorem 2]{dedic2009lower}, informal}] 
\label{thm:lower_bound_crypto}
  For every universal
  stegosystem $\Steg$ 
  there exists a channel $\chan$ such that  
  \begin{align*} 
    &\InSec_{\Steg,\chan}(\kappa) + \unrel_{\Steg,\chan}(\kappa) \ \ge\\
    &\frac{1}{2} - \frac{e\cdot \query_{\Steg,\chan}(\kappa)}{2^{\rate_{\Steg,\chan}(\kappa)}} -\Psi(\query_{\Steg,\chan}(\kappa)) - o(1),     \tag{**}
    \label{eq:insec:dedic}
  \end{align*}
  where $\Psi$ describes a term caused by the insecurity
  of the \ac{PRF} used in the construction of $\chan$, \ie
  \begin{align*}
    \Psi(t) := \InSec^{\prf}_{(\algf,\Gen_{\algf})}(t)+\negl(\kappa)
  \end{align*}
for a negligible function $\negl$. 
\end{theorem}

They thus prove that the exponential query complexity
$\kappa \cdot 2^{\fout_{\algf}(\kappa)}$ of $\RejSam^{\algf,\Pi}$ is asymptotically
optimal: indeed, if
$\query_{\Steg,\chan}(\kappa) \in o(\kappa \cdot 2^{\fout_{\algf}(\kappa)})$ and
$\query_{\Steg,\chan}(\kappa)\in \kappa^{\landauO(1)}$, the right hand
side of Equation~\eqref{eq:insec:dedic} goes to $1/2$. However, analogously to our discussion on
the upper bound in Equation~\eqref{eq:insec:hopper}, we notice that the lower bound is not meaningful if
$\query_{\Steg,\chan}(\kappa)\in \omega(\poly(\kappa))$ (even if
$\query_{\Steg,\chan}(\kappa)\in o(\kappa 2^{\fout_{\algf}(\kappa)})$), as the right
hand side of the inequality does not necessarily need to go to $0$ in
this case.  The red area in \autoref{fig:queries:vs:rate} illustrates
this lower bound.


Later, \citeauthor{hopper2009provably} provided another lower 
bound on the insecurity and unreliability  \cite[Theorem 5]{hopper2009provably} .
They show that for every universal stegosystem $\Steg$
and for any $\kappa$ there exists a channel such that: 
\begin{align*} 
&  \InSec_{\Steg,\chan}(\kappa,\query_{\Steg,\chan}(\kappa)) + \unrel(\kappa) \  \ge\\
    &1-  \frac{\query_{\Steg,\chan}(\kappa)}{2^{\rate_{\Steg,\chan}(\kappa)}} -2^{-\kappa},
  \tag{***}
  \label{eq:insec:hopper:lower}
\end{align*}
where $\InSec(\kappa,q)$, in contrast to $\InSec(\kappa)$, denotes insecurity over wardens of time 
complexity and size $> q$
%
Note that in the case of $\rate_{\Steg,\chan}(\kappa)\in \omega(\log \kappa)$, %lower 
the bounds in Equation~\eqref{eq:insec:dedic}
and Equation~\eqref{eq:insec:hopper:lower}
are incomparable in the following sense.
Due to Equation~\eqref{eq:insec:dedic}, if 
in a reliable universal stegosystem $\Steg$
the number of queries is dominated by $2^{\rate_{\Steg,\chan}(\kappa)}$ 
then there exists a \emph{polynomial-time} bounded Warden 
whose advantage to detect $\Steg$ is big. 
The time complexity of the Warden 
must not depend on the query complexity $\query_{\Steg,\chan}$ of $\Steg$
but Equation~\eqref{eq:insec:dedic}
needs the 
assumption that pseudorandom functions exist
and it may be meaningless if the rate exceeds $\log \kappa$.
The bound in Equation~\eqref{eq:insec:hopper:lower} does 
not need any cryptographic assumption, it is meaningful
for any $\rate_{\Steg,\chan}(\kappa)$, but the Warden who detects $\Steg$ 
needs time and size bigger than the query complexity $\query_{\Steg,\chan}(\kappa)$ of $\Steg$.
Thus, in cases of super-polynomial query complexity, the Warden is not 
polynomial-time bounded anymore implying $\InSec_{\Steg,\chan}(\kappa,q) \gg\InSec_{\Steg,\chan}(\kappa)$.


\section{Our Contributions}

Thus, as shown above, if high rate is required we have no guarantee that the
discussed systems are %remain 
secure and reliable. And indeed, \emph{no} secure
and reliable universal stegosystem (irrespective of its query complexity) 
with rate larger than $\log \kappa$ was known before, even under unproven 
cryptographic assumptions.  
Note that the secure stegosystems used
in practice typically achieve a rate of  $\sqrt{\kappa}$
% where, recall $n:=n(\kappa)$ denotes the length of a single document 
 \textendash\ much larger than $\log(n)$ \cite{ker2013moving}.
A longstanding conjecture, the \emph{Square Root Law of Steganographic
Capacity} \cite{filler2009squaremarkov, ker2008square} 
deals with just this fact. It says that a rate
of the form $(1-\varepsilon)\sqrt{\kappa}$ is always achievable (not necessarily
in a setting of universal steganography).
We thus have the situation, that the best known theoretical rate is
$\log \kappa$, while all practical rates are of order~$\sqrt{\kappa}$.  


One of the main results of this chapter -- \autoref{th:main:existexp} -- is the construction 
of a universal stegosystem
that is scalable with respect to the rate 
up to $n(\kappa)^\alpha$ for every $\alpha < 1$. 
However, to achieve this rate, an
exponential number of queries is needed. On the other hand 
we prove in \autoref{th:main:lower:bound} that this query complexity is minimal. %optimal. 
We give a complete answer to the 
question shown in \autoref{fig:queries:vs:rate} of determining the relationship between rate 
and number of queries.
For an illustration of our results see 
\autoref{fig:queries:our:results}.


\label{sec:contribution}
\begin{figure}[ht]
  \centering
\begin{tikzpicture}[xscale=0.5, yscale=0.5]
  \draw[-{>[scale=2]}] (0,0) to (0,10.5);
  \draw[-{>[scale=2]}] (0,0) to (12.2,0);
  \node[align=center] at (-.1,9.2) [left] {\#queries $q$\\(log scale)};
  \node at (11,-1.2) [above] {rate $r$};

  \draw (3,-.1) to (3,.1);
  \node at (3,-1.2) [above] {$\log \kappa$};
  \draw (6,-.1) to (6,.1);
  \node at (6,-1.2)  [above] {$\sqrt{\kappa}$};
  \draw (9,-.1) to (9,.1);
  \node at (9,-1.2)  [above]  {$\kappa^{1-\delta}$};
  \draw[dashed,color=orange] (9,0) to (9,10);

  \draw (-.1,3) to (.1,3);
  \node at (-.1,3) [left] {$\poly(\kappa)$};

% Dedic

  \draw[domain=0:9,smooth,color=black!20!white] plot ({\x},{\x});
  \node[color=black!40!white,fill=white] at (9,10) {$\InSec(\kappa)+\unrel(\kappa) \le \epsilon$};

  \draw[fill=red!20!white]  (0,0) -- (8.9,8.9) -- (8.9,0) -- cycle;
  \draw[dashed] (3,0) to (3,10);
  % \draw[domain=3:10,smooth,color=red!40!white] plot ({\x},{3*exp(3-\x)});
  % \fill[domain=3:10,fill=red!40!white] (3,0) -- plot ({\x},{3*exp(3-\x)}) --
  % (10,0) -- cycle;
  
  \draw[domain=0:8.9,smooth,color=green!60!black,very thick] plot ({\x},{\x});

% Open

% \node[draw=black,thick,rounded corners=2pt,below left=2mm,fill=white] at
% (15.5,9.5) {%
%   \begin{tikzpicture}[node distance=.0cm]
%   \node[color=red!40!white] (dedic) {Not possible [Dedi{\'c}]};    
%   \node[color=blue!30!white, below = of dedic] (hopper) {Achievable [Hopper]};    
%   \node[color=green!60!black, below = of hopper] (we) {Achieveable [this
%     work]};    
%   \node[color=orange, below = of we] (it) {Information-theoretic upper bound};    
%   \end{tikzpicture}
% };

\end{tikzpicture}
\caption{Results (without any assumptions) of this paper: 
our stegosystem achieves negligible insecurity 
and unreliability for the number of queries depending 
on the rate as shown by the green plot. 
This bound is sharp: any system of rate and query complexity 
in the red area is unreliable or insecure against
polynomial-time bounded wardens.}
\label{fig:queries:our:results}
\end{figure}

\section{Pseudorandom Functions of Very High Hardness}
\label{sec:secure-very:hard:prf}

We construct two families of pseudorandom 
functions that are secure against adversaries of sub-exponential 
running times. Our result does not rely on any unproven assumptions
but to construct the family, super-exponential time is needed.

In our stegosystem we will apply a
super-polynomial time computable pseudorandom function based on an
algorithm $\algo{G}$ that also takes
super-polynomial  time, which is given by the following result due
to \citeauthor{goldreich1992sparse} in \cite{goldreich1992sparse}. In order to
simplify the notation throughout this and the next section, let 
$\alpha_1, \alpha_2$ be constants with $1>\alpha_{1}\geq
\alpha_{2}>0$  and~let
\begin{equation}\label{eq:def:const}
  n=n(\kappa)=\kappa^{\alpha_{1}},  b=\fout_{\algf}(\kappa)=\kappa^{\alpha_{2}},   N=2^{n}\cdot b,\   \text{and}\
B=2^{b}\cdot b.
\end{equation}

\begin{theorem}[{\cite[Lemma $5$]{goldreich1992sparse}}]
Let $\varphi(n)$ be any sub-exponential function in $n$. There are
(non-polynomial) generators which expand truly random strings of length $n$
into pseudorandom string of length $\varphi(n)$. 
\end{theorem}

A close inspection of the result and its proof immediately implies the
following theorem:

\begin{theorem}
\label{thm:goldreich_gen}
There is a deterministic algorithm $\algo{G}$ running in time
$\landauO(2^{2^{n}})$, that on input $x\in \{0,1\}^{\kappa}$ produces a
string $\algo{G}(x)\in \{0,1\}^{N}$, and a negligible function $\negl$
such that for all polynomials $t$:
\begin{align*}
  \InSec^{\dist}_{\algo{G}(\{0,1\}^{\kappa}),\{0,1\}^{N}}(1,t,N)\leq \negl(\kappa).  
\end{align*}
There is also another deterministic algorithm $\algo{G'}$ running in
time $\landauO(2^{2^{b}})$, that on input
$x\in \{0,1\}^{\kappa}$ produces a string $\algo{G'}(x)\in \{0,1\}^{B}$, and a
negligible function $\negl'$ such that for all polynomials $t$:
\begin{align*}
  \InSec^{\dist}_{\algo{G'}(\{0,1\}^{\kappa}),\{0,1\}^{B}}(1,t,B)\leq \negl'(\kappa).
\end{align*}
\end{theorem}


The theorem says  that no polynomial time algorithm in $N$
(recall $N=2^{n}\cdot b$) can
distinguish between the distribution $\algo{G}(\{0,1\}^{\kappa})$ and the uniform
distribution on $\{0,1\}^{N}$. Similarly, no polynomial time algorithm
in $B$ can distinguish $\algo{G'}(\{0,1\}^{\kappa})$ and the uniform
distribution on $\{0,1\}^{B}$. Note that this is very close to the
optimum. A distribution distinguisher that runs in time
$\Omega(2^{\kappa})$ could compute $\algo{G}(\{0,1\}^{\kappa})$ and
simply check whether the received query belongs to it. 


The running
time of $\algo{G}$ and $\algo{G'}$ is exponential in $N$ (resp. $B$), while the running time of the
distinguisher is polynomial in $N$ (resp. $B$). 
Note that the usual construction to obtain a pseudorandom function from
a pseudorandom generator due to \citeauthor{goldreich1986construct}
\cite{goldreich1986construct} is not suited for our situation: Its security
proof relies on the ability of the attacker on the function to simulate
the generator. As a simulation of the generator takes exponential time
and our attackers are polynomial, we can not use this approach. Instead,
we observe that the generators produce very long strings. We will
interpret these strings as the table of a function.

For a bit string $\omega=\omega_{1}\omega_{2}\ldots$ of length
$2^{X}\cdot Y$, for some positive integers $X$ and $Y$,
let the function 
$F_{\omega}\colon \{0,1\}^{X}\to \{0,1\}^{Y}$
be defined as
\[F_{\omega}(z)=\omega_{i_{z}\cdot Y}\omega_{i_{z}\cdot Y+1}\ldots
  \omega_{(i_{z}+1)\cdot Y-1},\]
if $z$ is the binary representation of the number $i_{z}$.
\begin{example}
For example, the bit string $\omega=01\, 11\, 10\, 11\, 00\, 01\, 01\, 10$ corresponds
to the function $F_{\omega}\colon \{0,1\}^{3}\to \{0,1\}^{2}$ with
e.g. $F_{\omega}(000)=01$, $F_{\omega}(001)=11$, and
$F_{\omega}(111)=10$. 
\end{example}

The definition of $F_\omega$ implies a bijection between
$\{0,1\}^{2^{X}\cdot Y}$ and the set of all function from
$\{0,1\}^{X}\to \{0,1\}^{Y}$ \ie $\Fun(X,Y)$.

We will now construct \acp{PRF} out of the algorithms $\algo{G}$ and
$\algo{G'}$ that we will call $(\algf,\Gen)$ respectively
$(\algf',\Gen)$.  Both \acp{PRF} share the same key-generator $\Gen$,
that upon input $1^{\kappa}$ chooses $k\sgets \{0,1\}^{\kappa}$ and
returns the key $k$. The keyed function $\algf$ takes a key $k$ and a
bitstring $x$ of length $n$, computes $\omega = \algo{G}(k)$ and returns
$F_{\omega}(x)$. Similarly, $\algf'$ takes a key $k'$ and a bitstring
$x'$ of length $b$, computes $\omega'=\algo{G'}(k')$ and returns the
value $F_{\omega'}(x')$. As $\algo{G}$ and $\algo{G'}$ are
non-efficient, so are $\algf$ and $\algf'$: A single call to
$\algf_{k}(x)$ takes time $\landauO(2^{2^{n}})$ while a single call to
$\algf'_{k'}(x')$ takes time $\landauO(2^{2^{b}})$. The following
theorem shows that $\algf$ is not distinguishable from $\Fun(n,b)$ by
any algorithm with time complexity $\poly(N)$ and $\algf'$ is not
distinguishable from $\Fun(b,b)$ by any algorithm with time complexity
$\poly(B)$.

\begin{theorem}
\label{thm:dist_prf_exist}
For all functions $q$ and $t$ of $\kappa$ such that there are
polynomial $p$ and $p'$ with $t(\kappa)\leq p(N)$ and $t(\kappa)\leq p'(B)$, we have
\begin{enumerate}
 \item[1.]     
  $\InSec^{\prf}_{(\algf,\Gen)}(q,t,\kappa)\ \leq \
  \InSec^{\dist}_{\algo{G}(\{0,1\}^{\kappa}),\{0,1\}^{N}}(1,p,N)$ and
  \item[2.]       
  $\InSec^{\prf}_{(\algf',\Gen)}(q,t,\kappa)\ \leq \
  \InSec^{\dist}_{\algo{G'}(\{0,1\}^{\kappa}),\{0,1\}^{B}}(1,p',B)$.
\end{enumerate}
\end{theorem}
The proof of the theorem relies simply on the fact that any
\ac{PPTM} attacking $\algf$ has only access to an excerpt of
size $\poly(\kappa)$ of $\algo{G}(x)$. Theorem \ref{thm:goldreich_gen} states that even
access to the whole string of length $N\gg \poly(\kappa)$ does not help
an adversary. The advantage of any adversary is thus only negligible. 

\begin{proof}
  We only prove the theorem for $\algf$, as the proof for $\algf'$ is analogous.

  Let $\Dist$ be any \ac{PPTM} (distinguisher) that runs in time
  $t(\kappa)$ and tries to distinguish $\algf$ from
  $\Fun(n,b)$ by making $q(\kappa)$ queries. The algorithm $\Dist$ has
  access to a function oracle $f$, which is either uniformly chosen from
  $\Fun(n,b)$ or equal to $\algf_{k}$ for a certain $k\in
  \{0,1\}^{\kappa}$.  We will now construct a distribution distinguisher $\DDist$ for
  $\algo{G}$, such that
  \begin{align*}
    &\left|\Pr[\DDist^{\algo{G}(\{0,1\}^{\kappa})}(1^{\kappa})=1]-\Pr[\DDist^{\{0,1\}^{N}}(1^{\kappa})=1]\right|\ = \ \\
    &\left|\Pr_{k}[\Dist^{\algf_{k}}(1^{\kappa})=1]-\Pr_{f}
      [\Dist^{f}(1^{\kappa})=1]\right|,
  \end{align*}
where the probabilities are taken over the samples from the
distributions and the choice of $k\sgets \{0,1\}^{\kappa}$ and $f\sgets
\Fun(n,b)$. 
  The distribution distinguisher $\DDist$ makes a single query to its distribution
  oracle and receives a bit string $\omega\in \{0,1\}^{N}$,
  which is either a random string or produced by $\algo{G}(x)$.
  Whenever $\Dist$ makes a query $z$ to its function
  oracle, $\DDist$ returns $F_{\omega}(z)$. In the end, $\DDist$ returns
  the same value as $\Dist$. We thus have
  \begin{align*}
    \Pr_{k}[\Dist^{\algf_{k}}(1^{\kappa})=1]\ = \ \Pr[\DDist^{\algo{G}(\{0,1\}^{\kappa})}(1^{\kappa})=1]
  \end{align*}
  and because of the bijection between $\Fun(n,b)$ and
  $\{0,1\}^{N}$, we have
  \begin{align*}
    \Pr_{f}[\Dist^{f}(1^{\kappa})=1]\ = \ \Pr[\DDist^{\{0,1\}^{N}}(1^{\kappa})=1].
  \end{align*}
  The computation of $F_{\omega}(z)$ takes time $\landauO(N)$ for each
  query and the  simulation of $\Dist$ takes time $t(\kappa)$. In total,
  $\DDist$ makes a single query and has running time $N\cdot
  q(\kappa)+t(\kappa) \leq p(N)$.
\end{proof}

The following corollary thus follows directly from  \autoref{thm:goldreich_gen} 
 and from \autoref{thm:dist_prf_exist} and
sums up the results of this section.

\begin{corollary}
  There exists non-efficient \acp{PRF} $(\algf,\Gen)$ and
  $(\algf',\Gen)$ and negligible functions $\negl$ and $\negl'$ such
  that 
  \begin{itemize}
  \item $\fin_{\algf}(\kappa)=n$ and $\fin_{\algf'}(\kappa)=b$,
  \item $\fout_{\algf}(\kappa)=\fout_{\algf'}(\kappa)=b$,
  \item $\InSec^{\prf}_{(\algf,\Gen)}(q,t,\kappa)\leq \negl(\kappa)$ for all
    functions $q$ and $t$ with $t(\kappa) \leq \poly(N)$,
  \item and $\InSec^{\prf}_{(\algf',\Gen)}(q,t,\kappa)\leq \negl'(\kappa)$
    for all functions $q$ and $t$ with $t(\kappa) \leq \poly(B)$.
  \end{itemize}
\end{corollary}

\section{Rate-efficient Steganography }
%\section{Unconditionally Secure and Reliable Steganography Exists}
\label{sec:secure-steg-exists}

In this section we prove that there exists secure, reliable and 
rate-efficient steganography.
Our result does not rely on any unproven assumption.

To construct a universal stegosystem, which is unconditionally secure
we will use the \acp{PRF} $(\algf,\Gen)$ and $(\algf',\Gen)$
of the previous section
in the rejecting-sampling algorithm.
As in the previous section, 
let $\alpha_1,\alpha_2$ be constants with 
$1>\alpha_{1}\geq \alpha_{2}>0$
and let $n, b,N, B$ be as defined in (\autoref{eq:def:const}) 
in \autoref{sec:secure-very:hard:prf}. 
 %on page~\pageref{eq:def:const}.

In the following, let $\Steg$ be the rejection sampling stegosystem that
is constructed by using the \ac{PRF} $(\algf,\Gen)$ to construct the set
of functions and $\Pi=(\Gen^{\algf'},\Enc^{\algf'},\Dec^{\algf'})$ be
the \ac{SES} derived from the random counter mode explained in
\autoref{sec:primitives}, \ie $\Steg=\RejSam^{\algf,\Pi}$.

\citeauthor{backes2005active}  proved in \cite{backes2005active} that $\RejSam$ is
secure against \ac{SS-CHA} wardens as
as long as the family of
functions used is pseudorandom and as long as the number of bits
embedded in a single document is at most $\log \kappa$. We will expand
this result and prove that one can embed up to $o(\kappa)$ bits into
a single document.


By using our \acp{PRF} of very high hardness, we prove that the stegosystem  $\Steg$
is secure for every channel with
sufficient min-entropy that is sampleable in exponential time. 
Moreover, it remains secure for any channel which does not break 
the security of those \acp{PRF}.
This analysis resembles the analysis in
\cite{backes2005active}, but spells out the relation of
$\InSec^{\sscha}_{\Steg,\chan}$ and
$\InSec^{\prf}_{(\algf,\Gen),\chan}$ respectively
$\InSec^{\prf}_{(\algf,\Gen)}$.



  \begin{theorem}
  \label{thm:stego_exists}
  The rejection-sampling stegosystem $\Steg$ satisfies the following
  properties relative to channel $\chan$ for all polynomials $q$ and $t$:
  \begin{itemize}
  \item \ 
    
    \vspace{-1.5cm}
    \begin{align*}
      &\InSec^{\sscha}_{\Steg,\chan}(q,t,\kappa) \leq\\
      &\quad \InSec^{\prf}_{(\algf,\Gen),\chan}\left(q(\ell+1)2^{b}\kappa\ ,\
        q(\ell+q)2^{b}\kappa\ ,\ 2^{b}\cdot \kappa\right)+\\
      &\quad q(\kappa)(\ell(\kappa)+1)\left(2^{b-\minent(\chan,n)}+\eta^{2^{b}\kappa}\right)+\\
      &\quad \InSec^{\prf}_{(\algf',\Gen)}(\ell+1\ ,\ (\ell+1)^{2}\ ,\ \kappa)  +\\
      &\quad 2\InSec^{\prf}_{(\algf',\Gen),\chan}(2q\ell\ ,\ t\ ,\ \kappa) +
      \frac{q(\kappa)\cdot (n+1)\cdot b\cdot 
        (q(\kappa)-1)}{n\cdot 2^{b}}
    \end{align*}
for a constant $\eta < 1$.
  \item  \ 

    \vspace{-1.5cm}
    \begin{align*}
      &\unrel_{\Steg,\chan}(\kappa) \leq\\
       &\quad
         \InSec^{\prf}_{(\algf,\Gen),\chan}\left((\ell+1)2^{b}\kappa\ ,\ (\ell+1)2^{b}
        \kappa\ ,\ 2^{b}\kappa\right)+\\
      &\quad  (\ell(\kappa)+1)\cdot \exp(-\kappa)+(\ell(\kappa)+1)^{2} \cdot
        \kappa^{2}\cdot  2^{2b-\minent(\chan,n)}
    \end{align*}

  \end{itemize}

  \end{theorem}

 
 
\begin{proof}
\myNote{Security}
  In order to bound the insecurity of the stegosystem, we construct for
  every warden $\Ward$ that runs in time $t(\kappa)$ and makes
  $q(\kappa)$ queries a distinguisher $\Dist$ on  $(\algf,\Gen)$
  such that
  \begin{align*}
    & \Adv^{\sscha}_{\Ward,\Steg,\chan}(q,t,\kappa)\leq\\
    &\quad
    \Adv^{\prf}_{\Dist,(\algf,\Gen),\chan}\left(q(\ell+1)2^{b}\kappa\ ,\
        q(\ell+q)2^{b}\kappa\ ,\ 2^{b}\cdot \kappa\right)+\\
      &\quad q(\kappa)(\ell(\kappa)+1)\left(2^{b-\minent(\chan,n)}+\eta^{2^{b}\kappa}\right)+\\
      &\quad \InSec^{\prf}_{(\algf',\Gen)}(\ell+1\ ,\ (\ell+1)^{2}\ ,\ \kappa)  +\\
      &\quad 2\InSec^{\prf}_{(\algf',\Gen),\chan}(2q\ell\ ,\ t\ ,\ \kappa) +
      \frac{q(\kappa)\cdot (n+1)\cdot b\cdot 
        (q(\kappa)-1)}{n\cdot 2^{b}}
  \end{align*}
for a constant $\eta < 1$. This yields the security of the stegosystem.

 Let $\Ward$ be any such
  warden on the stegosystem $\Steg$ with respect to the channel
  $\chan$. The distinguisher $\Dist$ has access to a function oracle $f$,
  which is either uniformly chosen from $\Fun(n,b)$ or
  equal to $\algf_{k}$ for a certain $k\in \{0,1\}^{\kappa}$. The
  distinguisher $\Dist$ simulates the warden $\Ward$. Whenever $\Ward$
  makes a query to
  the channel-oracle, $\Dist$ uses its channel-oracle to produce such a
  sample. Whenever $\Ward_{1}$ or $\Ward_{2}$ make a query $(m,h)$ to the challenging oracle,
  $A$ uses the encoding algorithm $\SEnc_{f}(k,m,h)$, where the call to
  the function $\algf$ is replaced by a call to $f$. After the first
  phase, $\Ward_{1}$ produces a triple $(m,h,s)$ and $\Dist$
  generates a random bit $b\sgets \{0,1\}$. If $b=0$, it computes
  $d_{1},\ldots,d_{\ell(\kappa)} \gets \SEnc_{f}(k,m,h)$ and if $b=1$,
  it samples $\ell(\kappa)$ random documents
  $d_{1},\ldots,d_{\ell(\kappa)}$ from $\chan$. The distinguisher then simulates
  $\Ward_{2}$ on input $d_{1},\ldots,d_{\ell(\kappa)},s$ and returns $1$
  iff the output of $\Ward_{2}$ equals the bit $b$.

\myNote{$f=\algf_{k}$}
  If $f=\algf_{k}$, the distinguisher $\Dist$
  simply simulates the run of $\Ward$ against the stegosystem and we
  thus have
  \begin{align*}
    \Pr_{k}[\Dist^{\algf_{k}}(1^{\kappa})=1]=\frac{1}{2}+\Adv_{\Ward,\Steg,\chan}(\kappa).
  \end{align*}

\myNote{$f\sgets \Fun(n,b)$}
  If $f$ is truly randomly and $m=m_{1}m_{2}\ldots m_{\ell(\kappa)}$ is
  a message of length $\ell(\kappa)\cdot b$ such that $m_{i}\neq m_{j}$
  for every $i\neq j$, we can think of $\SEnc_{f}(k,m,h)$ as
  $\ell(\kappa)$-fold product of the probability distribution
  $\SEnc_{f}(k,m_{i},\cdot)$, where $f$ is chosen randomly for every $i$.
  The output of $\SEnc_{f}(k,m_{i},\cdot)$ is nearly identical to the
  channel (see \autoref{thm:backes}), if the corresponding message of
  length $b$ is also chosen uniformly. The statistical distance is
  bounded by 
  \begin{align*}
    \ell(\kappa)\cdot \left(2^{\fout_{F}(\kappa)-\minent(\chan,n)}+\eta^{2^{\fout_{F}(\kappa)}\cdot
  \kappa}\right)
  \end{align*}
for a constant $\eta < 1$.
 \autoref{thm:counter_mode}
  implies that for $\Ward$, the difference between the behavior of
  $\SEnc_{f}(k,m_{i},h)$ on a uniformly chosen message $m_{i}$ or an
  $m_{i}$ generated by the encryption $\Enc^{\algf'}$ is bounded by
  \begin{align*}
    2\InSec^{\prf}_{(\algf',\Gen),\chan}(q\ell,t,\kappa)+ \frac{q(\kappa)\cdot (n+1)\cdot b\cdot 
        (q(\kappa)-1)}{n\cdot 2^{b}}.
  \end{align*}

  
  As we do not give $m$ to $\SEnc$, but rather the message encrypted message
  $m'=\Dec^{\algf'}(k,m)=m'_{1}m'_{2}\ldots m'_{\ell(\kappa)+1}$, the probability that
  there are $i\neq j$ such that $m'_{i}=m'_{j}$ is at most
  \begin{align*}
    \InSec^{\prf}_{(\algf',\Gen)}(\ell+1,(\ell+1)^{2},\kappa)+\frac{(\ell(\kappa)+1)^{2}}{2^{b}},    
  \end{align*}
  by constructing an attacker on $\algf'$ which guesses values
  $x_{1},\ldots,x_{\ell(\kappa)+1}$ and tests, whether 
  $f(x_{1}),f(x_{2}),\ldots,f(x_{\ell(\kappa)+1})$ are pairwise different. 

  As the notion of  statistical distance is stronger than computational
  indistinguishability (see \autoref{thm:statistically}), we can thus
  conclude that there is a constant $\eta < 1$ such that for every
  \ac{PPTM} $\DDist$, we have
  \begin{align*}
    &\Adv_{\DDist,\chan,\SEnc_{f}}(\kappa) \leq\\   
    &\quad\ell(\kappa)\cdot \left(2^{\fout_{F}(\kappa)-\minent(\chan,n)}+\eta^{2^{\fout_{F}(\kappa)}\cdot
  \kappa}\right)+\\
    &\quad 2\InSec^{\prf}_{(\algf',\Gen),\chan}(q\ell,t,\kappa)+ \frac{q(\kappa)\cdot (n+1)\cdot b\cdot 
        (q(\kappa)-1)}{n\cdot 2^{b}}+\\
    &\quad     \InSec^{\prf}_{(\algf',\Gen)}(\ell+1,(\ell+1)^{2},\kappa)+\frac{(\ell(\kappa)+1)^{2}}{2^{b}},
  \end{align*}
  as $\Ward$ makes at most $q(\kappa)$ calls to its challenging
  oracle. If $f$ is truly random, $\Ward$ thus needs to distinguish
  between the channel distribution and between $\SEnc_{f}$. 
  This immediately implies that 
  \begin{align*}
    &\Pr_{f}[\Dist^{f}(1^{\kappa})=1] \leq\\
    &\quad \frac{1}{2}+   \ell(\kappa)\cdot \left(2^{\fout_{F}(\kappa)-\minent(\chan,n)}+\eta^{2^{\fout_{F}(\kappa)}\cdot
  \kappa}\right)+\\
    &\quad 2\InSec^{\prf}_{(\algf',\Gen),\chan}(q\ell,t,\kappa)+ \frac{q(\kappa)\cdot (n+1)\cdot b\cdot 
        (q(\kappa)-1)}{n\cdot 2^{b}}+\\
    &\quad     \InSec^{\prf}_{(\algf',\Gen)}(\ell+1,(\ell+1)^{2},\kappa)+\frac{(\ell(\kappa)+1)^{2}}{2^{b}}.
  \end{align*}

  As $\Adv_{\Dist,(\algf,\Gen)}(\kappa)=\left| \Pr[\Dist^{\algf_{k}}(1^{\kappa})=1] -
    \Pr[\Dist^{f}(1^{\kappa})]\right|$, this concludes our security
  analysis of $\InSec^{\sscha}_{\Steg,\chan}(q,t,\kappa)$.

  The simulation of each call to $\SEnc_{f}$ can be carried out in time
  $\landauO((\ell(\kappa)+1)\cdot 2^{b}\cdot \kappa)$ if one has access
  to the channel oracle.  The number of calls to the function oracle $f$
  is bounded by
  $\landauO(q(\kappa)\cdot (\ell(\kappa)+1)\cdot 2^{b}\cdot \kappa)$ The
  running time of $\Dist$ is thus at most
  $\landauO(q(\kappa)\cdot (\ell(\kappa)+1)\cdot 2^{b}\cdot
  \kappa)+t(\kappa)$ and the number of queries that $\Dist$ performs is
  at most
  $\landauO(q(\kappa)\cdot(\ell(\kappa)+1)\cdot 2^{b}\cdot \kappa)$.
  
\myNote{Reliability}
      Concerning the reliability, we construct for every message $m=m_{1},\ldots,m_{\ell(\kappa)}$ and
      every history $h$ a
      different distinguisher $\Dist_{m,h}$ against $\algf$. The attacker $\Dist_{m,h}$ with
      function oracle $f$  first computes the decoding
      $m'\gets\SDec_{f}(k,\SEnc_{f}(k,m,h))$ and returns $1$ if $m=m'$. 

\myNote{$f=\algf_{k}$}
If $f=\algf_{k}$,
      we have
      \begin{align*}
        &\Pr_{k}[\Dist_{m,h}^{\algf_{k}}(1^{\kappa})=1]\ = \
        \Pr_{k}[m\neq \SDec(k,\SEnc(k,m,h))]. 
      \end{align*}

\myNote{$f\sgets \Fun(n,b)$}
      If $f$ is a truly random function from
      $\Fun(n,b)$ and all samples $d_{1},d_{2},\ldots $
      taken from the channel oracle $\chan$ are different, the
      probabilities $\Pr[f(d_{i})=m_{j}]$ are independent, as we can assume
      that a new random function is evaluated on each $d_{i}$. Denote
      the event that all of the $d_{i}$ are pairwise different with
      $\overline{\collision}$. The probability that none of this samples
      evaluates to $m$ is then bounded by
      \begin{align*}
        &\Pr_{f}[m\neq \SDec_{f}(\SEnc_{f}(m,h))\mid \overline{\collision}
        ]\ \leq \\
        &\sum_{j=1}^{\ell(\kappa)+1} \prod_{i=1}^{2^{b}\cdot
          \kappa}\Pr_{f}[f(d_{i})\neq m_j]\ \leq\\
        &\tag{*}
          \label{eq:rel:unbounded:one}
        (\ell(\kappa)+1)\cdot \left(1-\frac{1}{2^{b}}\right)^{2^{b}\cdot
        \kappa}\ \leq \\
        &(\ell(\kappa)+1)\cdot \exp(-\kappa). 
      \end{align*}
      By definition, the maximal probability of any element from the
      channel is bounded from above by
      $2^{-\minent(\chan,n)}$. The probability
      that $d_{i}=d_{j}$ for $i\neq j$ is thus bounded by
      $2^{-\minent(\chan,n(\kappa))}$. Hence
      \begin{align*}
        &\Pr[\collision]\leq \\
        &((\ell(\kappa)+1)\cdot \kappa\cdot 2^{b})^{2}\cdot
        2^{-\minent(\chan,n)}=\\
        &\tag{**}
          \label{eq:rel:unbounded:two}
        (\ell(\kappa)+1)^{2}\cdot \kappa^{2}\cdot 2^{2b-\minent(\chan,n)}.
      \end{align*}
      If $p=\Pr[\collision]$, we can combine
      \eqref{eq:rel:unbounded:one} and \eqref{eq:rel:unbounded:two} and conclude
      \begin{align*}
        &\Pr_{f}[m\neq \SDec_{f}(\SEnc_{f}(m,h))]\ =\\
        &\Pr_{f}[m\neq \SDec_{f}(\SEnc_{f}(m,h))\mid
        \overline{\collision}]\cdot (1-p)
        +\\
        &\quad\Pr_{f}[m\neq \SDec_{f}(\SEnc_{f}(m,h))\mid
        \collision]\cdot p
      \end{align*}
      Hence, we have
      \begin{align*}
        &\Pr_{f}[m\neq \SDec_{f}(\SEnc_{f}(m,h))]\ \leq\\
        &(\ell(\kappa)+1)\cdot \exp(-\kappa)+(\ell(\kappa)+1)^{2}\cdot \kappa^{2}\cdot
          2^{2b-\minent(\chan,n)}.
      \end{align*}
      We thus have
      \begin{align*}
        &\left|\Pr_{k}[\Dist_{m,h}^{\algf_{k}}(1^{\kappa})=1] -
          \Pr_{f}[\Dist_{m,h}^{f}(1^{\kappa})=1]\right|=\\
        &\bigl|\Pr_{k}[m\neq
          \SDec(k,\SEnc(k,m,h))]-\\
        &\quad (\ell(\kappa)+1)\cdot
          \exp(-\kappa)+(\ell(\kappa)+1)^{2}\cdot 
          \kappa^{2}\cdot 2^{2b-\minent(\chan,n)}\bigl|.
      \end{align*}

      The simulation of the call to $\SEnc_{f}$ can be carried out in
      time $\landauO((\ell(\kappa)+1)\cdot 2^{b}\cdot \kappa)$ with
      $2^{b}\cdot \kappa$ calls to the function oracle $f$. The running
      time of $\Dist_{m,h}$ is thus at most
      $\landauO((\ell(\kappa)+1)\cdot 2^{b}\cdot \kappa)+t$ and the
      number of queries of $\Dist_{m,h}$ is at most
      $\landauO((\ell(\kappa)+1)\cdot 2^{b}\cdot \kappa)$.

      Reordering these terms thus gives us
      \begin{align*}
        &\unrel_{\Steg,\chan}(\kappa) \leq\\
       &\quad
         \InSec^{\prf}_{(\algf,\Gen),\chan}\left((\ell+1)2^{b}\kappa\ ,\ (\ell+1)2^{b}
        \kappa\ ,\ 2^{b}\kappa\right)+\\
      &\quad  (\ell(\kappa)+1)\cdot \exp(-\kappa)+(\ell(\kappa)+1)^{2} \cdot
        \kappa^{2}\cdot  2^{2b-\minent(\chan,n)}.
      \end{align*}
\end{proof}


 


By combining \autoref{thm:goldreich_gen}, 
\autoref{thm:dist_prf_exist} and \autoref{thm:stego_exists}
together, we can conclude the existence of a secure black-box
stegosystem (see \autoref{th:main:existexp}
for an informal statement) and in particular:


\begin{theorem}
  \label{thm:stego_negl}
  Let $\chan$ be a channel and let $\alpha_1, \alpha_2$ be
  constants with $1>\alpha_{1}\geq \alpha_{2}>0$. Furthermore, let
  $\negl_{\algo{G}}$ and $\negl_{\algo{G'}}$ be two negligible functions such that for
  every polynomial $p$, it holds that
  \begin{align*}
    &\InSec^{\dist}_{G(\{0,1\}^{\kappa}),\{0,1\}^{N},\chan}(1,p,N) \ \leq \
      \negl_{G}(\kappa),\\
    &\InSec^{\dist}_{G'(\{0,1\}^{\kappa}),\{0,1\}^{B},\chan}(1,p,B) \ \leq \ \negl_{G'}(\kappa).
  \end{align*}
  
  Furthermore, let $(\algf,\Gen)$ and $(\algf',\Gen)$ be the \acp{PRF}
  constructed from $\algo{G}$ and $\algo{G'}$, let
  $n(\kappa)=\kappa^{\alpha_{1}}$ be the document length and
  $\ell(\kappa)\cdot b(\kappa)=\ell(\kappa)\cdot \kappa^{\alpha_{2}}$ be the
  message length for some polynomial $\ell$. If
  $\minent(\chan,n(\kappa))> 2b(\kappa)$, then the rejection sampling
  stegosystem
  \begin{align*}
    \Steg=\RejSam^{\algf,(\Gen^{\algf'},\Enc^{\algf'},\Dec^{\algf'})}
  \end{align*}
  is a secure,
  reliable and rate-efficient stegosystem on $\chan$. 
\end{theorem}

 \begin{proof}
\myNote{Security}
  Recall that $N=2^{n}\cdot b$ and $B=2^{b}\cdot b$.
   Assume  $\Ward$ is a Warden with
   $\Adv^{\sscha}_{\Ward,\Steg,\chan}(\kappa)=\InSec^{\sscha}_{\Steg,\chan}(q,t,\kappa)$. 
   \autoref{thm:stego_exists} then implies that
    \begin{align*}
      &\InSec^{\sscha}_{\Steg,\chan}(q,t,\kappa) \leq\\
      &\quad \InSec^{\prf}_{(\algf,\Gen),\chan}\left(q(\ell+1)2^{b}\kappa\ ,\
        q(\ell+q)2^{b}\kappa\ ,\ 2^{b}\cdot \kappa\right)+ \tag{a}\\
      &\quad q(\kappa)(\ell(\kappa)+1)\left(2^{b-\minent(\chan,n)}+\eta^{2^{b}\kappa}\right)+\tag{b}\\
      &\quad \InSec^{\prf}_{(\algf',\Gen)}(\ell+1\ ,\ (\ell+1)^{2}\ ,\ \kappa)  +\tag{c}\\
      &\quad 2\InSec^{\prf}_{(\algf',\Gen),\chan}(2q\ell\ ,\ t\ ,\ \kappa) +\tag{d}\\
      &\quad \frac{q(\kappa)\cdot (n+1)\cdot b\cdot 
        (q(\kappa)-1)}{n\cdot 2^{b}} \tag{e}
    \end{align*}
    for a constant $\eta < 1$. 

    \myNote{(b)+(e)}Clearly, both of the two terms
    $q(\kappa)(\ell(\kappa)+1)\left(2^{b-\minent(\chan,n)}+\eta^{2^{b}\kappa}\right)$
    and
    $\frac{q(\kappa)\cdot (n+1)\cdot b\cdot (q(\kappa)-1)}{n\cdot
      2^{b}}$ are negligible in $\kappa$, as $\chan$ has sufficiently high
    min-entropy. There is thus a negligible function $\negl$ such that
 \begin{align*}
   &q(\kappa)(\ell(\kappa)+1)\left(2^{b-\minent(\chan,n)}+\eta^{2^{b}\kappa}\right)+\\
   &\quad\frac{q(\kappa)\cdot (n+1)\cdot b\cdot 
        (q(\kappa)-1)}{n\cdot 2^{b}} \leq\\
   &\negl(\kappa).
 \end{align*}

 \myNote{(a)}As $q,t,\ell$ and $b$ are polynomials, by using
 \autoref{thm:dist_prf_exist}, we have
 \begin{align*}
      &\InSec^{\prf}_{(\algf,\Gen),\chan}\left(q(\ell+1)2^{b}\kappa\ ,\
        q(\ell+q)2^{b}\kappa\ ,\ 2^{b}\cdot \kappa\right)\leq\\
   &\InSec^{\dist}_{\algo{G}(\{0,1\}^{\kappa}),\{0,1\}^{N}}(1,p,N)
 \end{align*}
for a polynomial $p$.
This insecurity is negligible by the assumption and there is thus a
negligible function $\negl'$ such that
\begin{align*}
      &\InSec^{\prf}_{(\algf,\Gen),\chan}\left(q(\ell+1)2^{b}\kappa\ ,\
        q(\ell+q)2^{b}\kappa\ ,\ 2^{b}\cdot \kappa\right)\leq\\
      &\negl'(\kappa).
\end{align*}

\myNote{(c)+(d)}Furthermore, \autoref{thm:dist_prf_exist} also implies (as $q,t$ and
$\ell$ are polynomials) that
\begin{align*}
  &\InSec^{\prf}_{(\algf',\Gen)}\left(\ell+1,(\ell+1)^{2},\kappa\right)+2\InSec^{\prf}_{(\algf',\Gen),\chan}(q\ell,t,\kappa)\ \leq\\
  &3\InSec^{\dist}_{\algo{G'}(\{0,1\}^{\kappa}),\{0,1\}^{B},\chan}(1,p,B)
\end{align*}
for some polynomial $p$. 
This insecurity is negligible in $\kappa$ by assumption and there is
thus a negligible function $\negl''$ such that
\begin{align*}
  &\InSec^{\prf}_{(\algf',\Gen)}\left(\ell+1,(\ell+1)^{2},\kappa\right)+2\InSec^{\prf}_{(\algf',\Gen),\chan}(q\ell,t,\kappa)\leq\\
  &\negl''(\kappa).
\end{align*}

In conclusion, we have
\begin{align*}
  \InSec^{\sscha}_{\Steg,\chan}(q,t,\kappa)\leq \negl(\kappa)+\negl'(\kappa)+\negl''(\kappa).
\end{align*}
The stegosystem $\Steg$ is thus secure on $\chan$.

\myNote{Reliability}
Concerning the unreliability, we can proceed similarly. 
\autoref{thm:stego_exists} implies that
    \begin{align*}
      &\unrel_{\Steg,\chan}(\kappa) \leq\\
       &\quad
         \InSec^{\prf}_{(\algf,\Gen),\chan}\left((\ell+1)2^{b}\kappa\ ,\ (\ell+1)2^{b}
        \kappa\ ,\ 2^{b}\kappa\right)+\\
      &\quad  (\ell(\kappa)+1)\cdot \exp(-\kappa)+(\ell(\kappa)+1)^{2} \cdot
        \kappa^{2}\cdot  2^{2b-\minent(\chan,n)}.
    \end{align*}
Due to sufficient min-entropy of $\chan$ and the fact, that
$\ell$, $b$ and $n$ are polynomials, there is a negligible
function $\negl$ such that
\begin{align*}
      & (\ell(\kappa)+1)\cdot \exp(-\kappa)+(\ell(\kappa)+1)^{2} \cdot
        \kappa^{2}\cdot  2^{2b-\minent(\chan,n)}\leq \\
  &\negl(\kappa).
\end{align*}

 As above, \autoref{thm:dist_prf_exist} shows that
 \begin{align*}
   &\InSec^{\prf}_{(\algf,\Gen),\chan}\left((\ell+1)2^{b}\kappa\ ,\ (\ell+1)2^{b}
     \kappa\ ,\ 2^{b}\kappa\right)\leq\\
   &\InSec^{\dist}_{\algo{G}(\{0,1\}^{\kappa},\{0,1\}^{N},\chan}(1,p,N)
 \end{align*}
 for a polynomial $p$.  This is negligible by assumption
there is thus a negligible function $\negl'$
such that
\begin{align*}
  \unrel_{\Steg,\chan}(\kappa)\leq \negl(\kappa)+\negl'(\kappa). 
\end{align*}
The stegosystem $\Steg$ is thus reliable on $\chan$.

As we embed $b(\kappa)= \kappa^{\alpha_{2}}$ bits into a single document
of length $n(\kappa)=\kappa^{\alpha_{1}}$, the
transmission rate $\fout_{\algf}(\kappa)$ is equal to $b(\kappa)$.  As
the min-entropy
$\minent(\chan,n)\leq \kappa^{\alpha_{1}}$, the
stegosystem $\Steg$ is rate-efficient on $\chan$, as
$\alpha_{1},\alpha_{2}$ are constants.
 \end{proof}


Note that the precondition concerning the two negligible functions
$\negl_{G},\negl_{G'}$ is always fulfilled, if the channel oracle can be
simulated in time $\poly(N)=\exp(\kappa^{\alpha_{1}})$. This is due to
\autoref{thm:dist_prf_exist}, that states the security of the
pseudorandom function. We have thus shown that the imbalance between the
running times of the stegoencoder and the warden introduced and used in
\cite{hopper2009provably} by \citeauthor{hopper2009provably} has
dramatic consequences: Unconditionally secure, reliable and rate-efficient universal
stegosystems exists in this scenario.

\section{Unconditional Lower Bound}
\label{sec:lower_bound}
In order to give an unconditional lower bound, we make use of a lower
bound by \citeauthor{dedic2009lower} in \cite{dedic2009lower}. By providing the warden $\Ward$
with an efficient test whether a document belongs to the support of the
channel, they prove:

\begin{theorem}[{\cite[Theorem 1]{dedic2009lower}}]
\label{thm:lower_bound}
  For every universal (not necessarily of polynomial-time complexity)
  stegosystem $\Steg$  there exists a  channel $\chan$ such that  
  \begin{align*}
    \InSec^{*}_{\Steg,\chan}(\kappa) + \unrel_{\Steg,\chan}(\kappa) \ \ge \ 
       \frac{1}{2} - \frac{e\cdot \query_{\Steg,\chan}(\kappa)}{2^{\rate_{\Steg,\chan}(\kappa)}} -o(1),    
  \end{align*}
  where $\InSec^{*}$ denotes the insecurity against polynomial \ac{SS-CHA}-wardens with
  an auxiliary \emph{oracle for testing membership} in the support of $\chan$.

\end{theorem}
\citeauthor{dedic2009lower} then argue that the assumption that a warden has an oracle
for membership-testing is not feasible, if the channel is chosen
completely random. By making use of the fact that the warden can
choose a history, while the stegoencoder can not, we will show how an
\emph{efficient} warden is able to test membership of a completely
random channel.

Let $\bm{S}_n$ be the set of all subsets of $\{0,1\}^{n}$ of
cardinality $n/2$. For $S\in \bm{S}_n$, let $\chan_{S}$ be the following
channel, where $\vec{1}$ denotes the vector of length $n$ that contains
a $1$ at every position:
\begin{itemize}
\item $\chan(S)_{\varnothing,n}$ is the uniform distribution on
  $\{0,1\}^{n}$.
\item $\chan(S)_{\vec{1}||d,n}$ is the uniform distribution on all
  strings in $\{0,1\}^{n}$ that start with $1$, if $d\in S$ or the
  uniform distribution on all strings in $\{0,1\}^{n}$ that start with
  $0$, if $d\not\in S$ (i.e. the first position indicates the membership
  of $d$ in $S$).
\item $\chan(S)_{h,n}$ is the uniform distribution on $S$ for all other
  histories.
\end{itemize}
The warden $\Ward$ for the family $\{\chan(S)\ \mid\ S\in \bm{S}_n\}_{n\in \nats}$ now
works as follows: It randomly chooses a history
$h\gets \{0,1\}^{n}\setminus \{\vec{1}\}$ and $m=00\ldots 0$ 
\textendash\  a message of length $\ell(\kappa)\cdot b(\kappa)$ containing only $0$-bits \textendash\ 
and gets
the results $d_{1},d_{2},\ldots,d_{\ell(\kappa)}$ from the challenging oracle on
$h$ and $m$. For $i\in \{1,\ldots,\ell(\kappa)\}$, it takes a sample
$s_{i}\gets \chan(S)_{\vec{1}||d_{i},n}$. If every sample $s_{i}$ starts
with $1$, the warden returns ``Non-Stego'' (\ie $b'=1$ in the
\ac{SS-CHA}-experiment) and else ``Stego'' (\ie $b'=0$ in the \ac{SS-CHA}-experiment). The
warden $\Ward$ is thus able to test membership in $S$ efficiently by making
use of the channel. Note that the stegoencoder can not make use of these
capabilities of $\chan(S)$ as Alice can only make queries to $\chan(S)_{h,n}$, 
where $h$ does not start with $\vec{1}$. We use here the 
definition for channel access as in \cite{hopper2009provably}, which 
assumes that the encoder has an access to the marginal channel distributions $\chan_h$
for the histories $h$ starting with adversarial chosen prefixes. 



We can thus efficiently simulate an oracle for membership-testing and
\autoref{thm:lower_bound} thus implies 
(the formal statement of \autoref{th:main:lower:bound}):
\begin{theorem}
\label{th:main:lower:bound:formal}
  For every universal (not necessarily efficient) stegosystem
  $\Steg$   there exists a channel $\chan$ such that  
  \[
    \InSec^{\sscha}_{\Steg,\chan}(\kappa) + \unrel_{\Steg,\chan}(\kappa) \ \ge \ 
       \frac{1}{2} - \frac{e\cdot \query_{\Steg,\chan}(\kappa)}{2^{\rate_{\Steg,\chan}(\kappa)}} - o(1).
  \]
\end{theorem}
Note that in contrast to \autoref{thm:lower_bound_crypto}, no
cryptographic assumption is necessary and in contrast to 
\autoref{thm:lower_bound}, no membership-oracle is necessary. Our lower
bound thus holds unconditionally. Furthermore, this lower bound holds
even when the running time of the stegosystem $\Steg$ is much larger (say
$2^{2^{\kappa}}$) than the running time of $\Ward$ (say
$\poly(\kappa)$). As the stegosystem $\Steg$ of
\autoref{sec:secure-steg-exists} has $\query_{\Steg,\chan}(\kappa)=2^{\kappa^{\alpha_{2}}}\cdot \kappa$
and $\rate_{\Steg,\chan}(\kappa)=\kappa^{\alpha_{2}}$,
\autoref{th:main:lower:bound:formal} also directly implies that
  $\Steg$ has (asymptotically) optimal query complexity. 

Note that this method only works because of the asymmetry between Alice
and Warden: While Warden has oracle-access for all possible histories,
Alice can only use the history chosen by Warden.

\section{Conclusions and Further Work}


We first gave the first universally secure, reliable and rate-efficient stegosystem by using
pseudorandom functions of very high hardness. The running time of the
stegosystem is roughly $2^{2^{o(\kappa)}}$.
The work of \citeauthor{dedic2009lower} in \cite{dedic2009lower}
gives the best known lower bound of a running
time of $\omega(\poly(\kappa))$ for any universal secure, reliable 
stegosystem (under cryptographic assumptions and of the rate $\omega(\log \kappa)$). We proved
that by making use of the imbalance between encoder and warden, this
lower bound also holds without any assumption and for any rate-efficient stegosystem.  

We also showed that the common phrase ``Steganography is Cryptography'' is
provably wrong as the communication channel is a very important part of
the steganographic setting. 




%%% Local Variables:
%%% TeX-master: "../main"
%%% End:


%  LocalWords:  Fridrich FEG
