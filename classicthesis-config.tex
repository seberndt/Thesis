% ****************************************************************************************************
% classicthesis-config.tex 
% formerly known as loadpackages.sty, classicthesis-ldpkg.sty, and classicthesis-preamble.sty 
% Use it at the beginning of your ClassicThesis.tex, or as a LaTeX Preamble 
% in your ClassicThesis.{tex,lyx} with % ****************************************************************************************************
% classicthesis-config.tex 
% formerly known as loadpackages.sty, classicthesis-ldpkg.sty, and classicthesis-preamble.sty 
% Use it at the beginning of your ClassicThesis.tex, or as a LaTeX Preamble 
% in your ClassicThesis.{tex,lyx} with % ****************************************************************************************************
% classicthesis-config.tex 
% formerly known as loadpackages.sty, classicthesis-ldpkg.sty, and classicthesis-preamble.sty 
% Use it at the beginning of your ClassicThesis.tex, or as a LaTeX Preamble 
% in your ClassicThesis.{tex,lyx} with % ****************************************************************************************************
% classicthesis-config.tex 
% formerly known as loadpackages.sty, classicthesis-ldpkg.sty, and classicthesis-preamble.sty 
% Use it at the beginning of your ClassicThesis.tex, or as a LaTeX Preamble 
% in your ClassicThesis.{tex,lyx} with \input{classicthesis-config}
% ****************************************************************************************************  
% If you like the classicthesis, then I would appreciate a postcard. 
% My address can be found in the file ClassicThesis.pdf. A collection 
% of the postcards I received so far is available online at 
% http://postcards.miede.de
% ****************************************************************************************************


% ****************************************************************************************************
% 0. Set the encoding of your files. UTF-8 is the only sensible encoding nowadays. If you can't read
% äöüßáéçèê∂åëæƒÏ€ then change the encoding setting in your editor, not the line below. If your editor
% does not support utf8 use another editor!
% ****************************************************************************************************
\PassOptionsToPackage{utf8}{inputenc}
	\usepackage{inputenc}

% ****************************************************************************************************
% 1. Configure classicthesis for your needs here, e.g., remove "drafting" below 
% in order to deactivate the time-stamp on the pages
% ****************************************************************************************************
\PassOptionsToPackage{eulerchapternumbers,listings,drafting,%
					 pdfspacing,%floatperchapter,%linedheaders,%
					 subfig,beramono,eulermath,
					 }{classicthesis}                                        
% ********************************************************************
% Available options for classicthesis.sty 
% (see ClassicThesis.pdf for more information):
% drafting
% parts nochapters linedheaders
% eulerchapternumbers beramono eulermath pdfspacing minionprospacing
% tocaligned dottedtoc manychapters
% listings floatperchapter subfig
% ********************************************************************


% ****************************************************************************************************
% 2. Personal data and user ad-hoc commands
% ****************************************************************************************************
\newcommand{\myTitle}{Upper and Lower Bounds on Provably Secure Steganography\xspace}
\newcommand{\mySubtitle}{Inauguraldissertation\xspace}
\newcommand{\myDegree}{Doctor Rerum Naturalium (Dr. rer. nat.)\xspace}
\newcommand{\myName}{Sebastian Berndt\xspace}
\newcommand{\myProf}{Maciej Liśkiewicz\xspace}
\newcommand{\myOtherProf}{Put name here\xspace}
\newcommand{\mySupervisor}{Put name here\xspace}
\newcommand{\myFaculty}{Put data here\xspace}
\newcommand{\myDepartment}{Institute for Theoretical Computer Science\xspace}
\newcommand{\myUni}{University of Lübeck\xspace}
\newcommand{\myLocation}{Lübeck\xspace}
\newcommand{\myTime}{March 2016\xspace}
\newcommand{\myVersion}{version 0.1\xspace}

% ********************************************************************
% Setup, finetuning, and useful commands
% ********************************************************************
\newcounter{dummy} % necessary for correct hyperlinks (to index, bib, etc.)
\newlength{\abcd} % for ab..z string length calculation
\providecommand{\mLyX}{L\kern-.1667em\lower.25em\hbox{Y}\kern-.125emX\@}
\newcommand{\ie}{i.\,e.\xspace}
\newcommand{\Ie}{I.\,e.\xspace}
\newcommand{\eg}{e.\,g.\xspace}
\newcommand{\wlogeneral}{w.\,l.\,o.\,g.\xspace}
\newcommand{\Eg}{E.\,g.\xspace} 
% ****************************************************************************************************


% ****************************************************************************************************
% 3. Loading some handy packages
% ****************************************************************************************************
% ******************************************************************** 
% Packages with options that might require adjustments
% ******************************************************************** 
%\PassOptionsToPackage{ngerman,american}{babel}   % change this to your language(s)
% Spanish languages need extra options in order to work with this template
%\PassOptionsToPackage{spanish,es-lcroman}{babel}
	\usepackage{babel}                  

\usepackage{csquotes}
\PassOptionsToPackage{%
    %backend=biber, %instead of bibtex
	backend=bibtex8,bibencoding=ascii,%
	language=auto,%
%	style=numeric-comp,%
    %style=authoryear-comp, % Author 1999, 2010
        style=alphabetic,
    %bibstyle=authoryear,dashed=false, % dashed: substitute rep. author with ---
    sorting=nyt, % name, year, title
    maxbibnames=10, % default: 3, et al.
    %backref=true,%
    natbib=true % natbib compatibility mode (\citep and \citet still work)
}{biblatex}
    \usepackage{biblatex}

\PassOptionsToPackage{fleqn}{amsmath}       % math environments and more by the AMS 
    \usepackage{amsmath}

% ******************************************************************** 
% General useful packages
% ******************************************************************** 
\PassOptionsToPackage{T1}{fontenc} % T2A for cyrillics
    \usepackage{fontenc}     
\usepackage{textcomp} % fix warning with missing font shapes
\usepackage{scrhack} % fix warnings when using KOMA with listings package          
\usepackage{xspace} % to get the spacing after macros right  
\usepackage{mparhack} % get marginpar right
\usepackage{fixltx2e} % fixes some LaTeX stuff --> since 2015 in the LaTeX kernel (see below)
%\usepackage[latest]{latexrelease} % will be used once available in more distributions (ISSUE #107)
\PassOptionsToPackage{printonlyused,smaller}{acronym} 
    \usepackage{acronym} % nice macros for handling all acronyms in the thesis
    %\renewcommand{\bflabel}[1]{{#1}\hfill} % fix the list of acronyms --> no longer working
    %\renewcommand*{\acsfont}[1]{\textsc{#1}} 
    \renewcommand*{\aclabelfont}[1]{\acsfont{#1}}
% ****************************************************************************************************


% ****************************************************************************************************
% 4. Setup floats: tables, (sub)figures, and captions
% ****************************************************************************************************
\usepackage{tabularx} % better tables
    \setlength{\extrarowheight}{3pt} % increase table row height
\newcommand{\tableheadline}[1]{\multicolumn{1}{c}{\spacedlowsmallcaps{#1}}}
\newcommand{\myfloatalign}{\centering} % to be used with each float for alignment
\usepackage{caption}
% Thanks to cgnieder and Claus Lahiri
% http://tex.stackexchange.com/questions/69349/spacedlowsmallcaps-in-caption-label
% [REMOVED DUE TO OTHER PROBLEMS, SEE ISSUE #82]    
%\DeclareCaptionLabelFormat{smallcaps}{\bothIfFirst{#1}{~}\MakeTextLowercase{\textsc{#2}}}
%\captionsetup{font=small,labelformat=smallcaps} % format=hang,
\captionsetup{font=small} % format=hang,
\usepackage{subfig}  
% ****************************************************************************************************


% ****************************************************************************************************
% 5. Setup code listings
% ****************************************************************************************************
% \usepackage{listings} 
% %\lstset{emph={trueIndex,root},emphstyle=\color{BlueViolet}}%\underbar} % for special keywords
% \lstset{language=[LaTeX]Tex,%C++,
%     morekeywords={PassOptionsToPackage,selectlanguage},
%     keywordstyle=\color{RoyalBlue},%\bfseries,
%     basicstyle=\small\ttfamily,
%     %identifierstyle=\color{NavyBlue},
%     commentstyle=\color{Green}\ttfamily,
%     stringstyle=\rmfamily,
%     numbers=none,%left,%
%     numberstyle=\scriptsize,%\tiny
%     stepnumber=5,
%     numbersep=8pt,
%     showstringspaces=false,
%     breaklines=true,
%     %frameround=ftff,
%     %frame=single,
%     belowcaptionskip=.75\baselineskip
%     %frame=L
% } 
\usepackage[plain]{algorithm}

% Remove the Label "Algorithm" from the algorithms
\captionsetup[algorithm]{labelformat=empty}
% Remove the counter from the algorithms
\renewcommand{\thealgorithm}{}

\usepackage{algpseudocode}
\algrenewcommand\algorithmicrequire{\textbf{Input:}}
\algrenewcommand\algorithmicindent{2em}%

\usepackage{tcolorbox}

% ****************************************************************************************************             


% ****************************************************************************************************
% 6. PDFLaTeX, hyperreferences and citation backreferences
% ****************************************************************************************************
% ********************************************************************
% Using PDFLaTeX
% ********************************************************************
\PassOptionsToPackage{pdftex,hyperfootnotes=false,pdfpagelabels}{hyperref}
    \usepackage{hyperref}  % backref linktocpage pagebackref
\pdfcompresslevel=9
\pdfadjustspacing=1 
\PassOptionsToPackage{pdftex}{graphicx}
    \usepackage{graphicx} 
 

% ********************************************************************
% Hyperreferences
% ********************************************************************
\hypersetup{%
    %draft, % = no hyperlinking at all (useful in b/w printouts)
    colorlinks=true, linktocpage=true, pdfstartpage=3, pdfstartview=FitV,%
    % uncomment the following line if you want to have black links (e.g., for printing)
    %colorlinks=false, linktocpage=false, pdfstartpage=3, pdfstartview=FitV, pdfborder={0 0 0},%
    breaklinks=true, pdfpagemode=UseNone, pageanchor=true, pdfpagemode=UseOutlines,%
    plainpages=false, bookmarksnumbered, bookmarksopen=true, bookmarksopenlevel=1,%
    hypertexnames=true, pdfhighlight=/O,%nesting=true,%frenchlinks,%
    urlcolor=webbrown, linkcolor=RoyalBlue, citecolor=webgreen, %pagecolor=RoyalBlue,%
    %urlcolor=Black, linkcolor=Black, citecolor=Black, %pagecolor=Black,%
    pdftitle={\myTitle},%
    pdfauthor={\textcopyright\ \myName, \myUni, \myFaculty},%
    pdfsubject={},%
    pdfkeywords={},%
    pdfcreator={pdfLaTeX},%
    pdfproducer={LaTeX with hyperref and classicthesis},%
}   

% ********************************************************************
% Setup autoreferences
% ********************************************************************
% There are some issues regarding autorefnames
% http://www.ureader.de/msg/136221647.aspx
% http://www.tex.ac.uk/cgi-bin/texfaq2html?label=latexwords
% you have to redefine the makros for the 
% language you use, e.g., american, ngerman
% (as chosen when loading babel/AtBeginDocument)
% ********************************************************************
\makeatletter
\@ifpackageloaded{babel}%
    {%
       \addto\extrasamerican{%
			\renewcommand*{\figureautorefname}{Figure}%
			\renewcommand*{\tableautorefname}{Table}%
			\renewcommand*{\partautorefname}{Part}%
			\renewcommand*{\chapterautorefname}{Chapter}%
			\renewcommand*{\sectionautorefname}{Section}%
			\renewcommand*{\subsectionautorefname}{Section}%
			\renewcommand*{\subsubsectionautorefname}{Section}%     
                }%
       \addto\extrasngerman{% 
			\renewcommand*{\paragraphautorefname}{Absatz}%
			\renewcommand*{\subparagraphautorefname}{Unterabsatz}%
			\renewcommand*{\footnoteautorefname}{Fu\"snote}%
			\renewcommand*{\FancyVerbLineautorefname}{Zeile}%
			\renewcommand*{\theoremautorefname}{Theorem}%
			\renewcommand*{\appendixautorefname}{Anhang}%
			\renewcommand*{\equationautorefname}{Gleichung}%        
			\renewcommand*{\itemautorefname}{Punkt}%
                }%  
            % Fix to getting autorefs for subfigures right (thanks to Belinda Vogt for changing the definition)
            \providecommand{\subfigureautorefname}{\figureautorefname}%             
    }{\relax}
\makeatother


% ****************************************************************************************************
% 7. Last calls before the bar closes
% ****************************************************************************************************
% ********************************************************************
% Development Stuff
% ********************************************************************
\listfiles
%\PassOptionsToPackage{l2tabu,orthodox,abort}{nag}
%   \usepackage{nag}
%\PassOptionsToPackage{warning, all}{onlyamsmath}
%   \usepackage{onlyamsmath}



% ********************************************************************
% Sebastians Stuff
% ********************************************************************
\usepackage{amssymb} 
\usepackage{amsthm} 
\usepackage{bm}
\usepackage{imakeidx}
\makeindex

\usepackage{tikz}
\usetikzlibrary{positioning}
\usetikzlibrary{decorations.markings}
\usetikzlibrary{arrows.meta}

\usepackage{pgfplots}
\usepgfplotslibrary{fillbetween}
\pgfplotsset{compat=1.12}



\makeatletter

\newcounter{algorithmicH}% New algorithmic-like hyperref counter
\let\oldalgorithmic\algorithmic
\renewcommand{\algorithmic}{%
  \stepcounter{algorithmicH}% Step counter
  \oldalgorithmic}% Do what was always done with algorithmic environment
% \providecommand\theHALG@line{\thealgorithm.\arabic{ALG@line}}
\providecommand\theHALG@line{ALG@line.\thealgorithmicH.\arabic{ALG@line}}
% \renewcommand{\theHALG@line}{ALG@line.\thealgorithmicH.\arabic{ALG@line}}
\makeatother




\newcommand{\myAlgorithm}[4]{
  \begin{algorithm}[H]
\caption{#1: #4}
  \begin{tcolorbox}[title={#1}, colframe=black!10, coltitle=black]
    \begin{algorithmic}[1]
\Require #2
      #3
    \end{algorithmic}
  \end{tcolorbox}
  \end{algorithm}
}
\newcommand{\myNoteHelp}[1]{\marginpar{\textcolor{black}{#1}}}
\newcommand{\myNote}[1]{\myNoteHelp{\textrm{#1}}}

\newcommand{\myDef}[1]{\emph{#1}\myNoteHelp{#1}\index{#1}}
\newcommand{\myDefInformal}[1]{\emph{#1}\myNoteHelp{#1}\index{#1
    (informal)}}

\newcommand{\AlgReturn}[1]{ \textbf{return} #1}
\newcommand{\algo}[1]{ \mathsf{#1}}

\newcommand{\nats}{\mathbb{N}}
\newcommand{\landauO}{\mathcal{O}}
\newcommand{\reals}{\mathbb{R}}
\newcommand{\sgets}{\twoheadleftarrow}
\newcommand{\rats}{\mathbb{Q}}
\newcommand{\concat}{\mid \mid}
\newcommand{\powerset}{\mathcal{P}}
\newcommand{\machine}{\mathsf{M}}
\newcommand{\chan}{\mathcal{C}}
\newcommand{\stegos}{\mathcal{S}}
\newcommand{\DomainHistory}{\left(\Sigma^{n(\kappa)}\right)^{*}}
\DeclareMathOperator{\dom}{dom}
\DeclareMathOperator{\fout}{out}
\DeclareMathOperator{\fin}{in}
\DeclareMathOperator{\supp}{supp}
\DeclareMathOperator{\Exp}{Exp}
\DeclareMathOperator{\img}{img}
\DeclareMathOperator{\query}{query}
\DeclareMathOperator{\rate}{rate}
\DeclareMathOperator{\collision}{Collision}
\DeclareMathOperator{\Coll}{\mathsf{Coll}}
\DeclareMathOperator{\Sig-Forge}{\mathsf{Sig-Forge}}
\DeclareMathOperator{\DDisting}{\mathsf{Dist-Disting}}
\DeclareMathOperator{\SSCHA-Dist}{\mathsf{SS-CHA-Dist}}
\DeclareMathOperator{\SSCCA-Dist}{\mathsf{SS-CCA-Dist}}
\DeclareMathOperator{\SSRCCA-Dist}{\mathsf{SS-RCCA-Dist}}
\DeclareMathOperator{\CPA-Dist}{\mathsf{CPA-Dist}}
\DeclareMathOperator{\CPAD-Dist}{\mathsf{CPA\$-Dist}}
\DeclareMathOperator{\CCA-Dist}{\mathsf{CCA-Dist}}
\DeclareMathOperator{\CCAD-Dist}{\mathsf{CCA\$-Dist}}
\DeclareMathOperator{\negl}{\mathsf{negl}}
\DeclareMathOperator{\poly}{\mathsf{poly}}
\DeclareMathOperator{\lin}{\mathsf{lin}}

\DeclareMathOperator{\Fun}{Fun}
\DeclareMathOperator{\Pics}{Pics}

\DeclareMathOperator{\Gen}{\mathsf{Gen}}
\DeclareMathOperator{\Inv}{\mathsf{Inv}}
\DeclareMathOperator{\algf}{\algo{F}}
\DeclareMathOperator{\algh}{\algo{H}}
\DeclareMathOperator{\Fi}{\mathsf{Fi}}
\DeclareMathOperator{\Dist}{\mathsf{Dist}}
\DeclareMathOperator{\DDist}{\mathsf{DDist}}
\DeclareMathOperator{\Sign}{\mathsf{Sign}}
\DeclareMathOperator{\Vrfy}{\mathsf{Vrfy}}
\DeclareMathOperator{\Forg}{\mathsf{Fo}}
\DeclareMathOperator{\Enc}{\mathsf{Enc}}
\DeclareMathOperator{\Dec}{\mathsf{Dec}}
\DeclareMathOperator{\PKEnc}{\mathsf{PKEnc}}
\DeclareMathOperator{\PKDec}{\mathsf{PKDec}}
\DeclareMathOperator{\SEnc}{\mathsf{SEnc}}
\DeclareMathOperator{\Steg}{\mathsf{S}}
\DeclareMathOperator{\SDec}{\mathsf{SDec}}
\DeclareMathOperator{\Att}{\mathsf{A}}
\DeclareMathOperator{\Ward}{\mathsf{W}}
\DeclareMathOperator{\RejSam}{\mathsf{RejSam}}

\newcommand{\pk}{\textit{pk}}
\newcommand{\sk}{\textit{sk}}







\DeclareMathOperator{\Adv}{\mathbf{Adv}}
\DeclareMathOperator{\InSec}{\mathbf{InSec}}
\DeclareMathOperator{\unrel}{\mathbf{UnRel}}
\DeclareMathOperator{\hash}{hash}
\DeclareMathOperator{\prf}{prf}
\DeclareMathOperator{\sig}{sig}
\DeclareMathOperator{\cpa}{cpa}
\DeclareMathOperator{\dist}{dist}
\DeclareMathOperator{\sscha}{ss-cha}
\DeclareMathOperator{\sscca}{ss-cca}
\DeclareMathOperator{\ssrcca}{ss-rcca}
\DeclareMathOperator{\cpad}{cpa\scalebox{.7}{\$}}
\DeclareMathOperator{\cca}{cca}
\DeclareMathOperator{\ccad}{cca\scalebox{.7}{\$}}

\newcommand{\minent}{H_{\infty}}

\newcommand{\examplesymbol}{\( \diamond \)}


\newtheorem{theorem}{Theorem}
\newtheorem{corollary}[theorem]{Corollary}


\theoremstyle{definition}
\newtheorem*{exampleth}{Example}
\newenvironment{example}{\begin{exampleth}%
  \renewcommand{\qedsymbol}{\examplesymbol}\pushQED{\qed}}%
  {\popQED\end{exampleth}}



\makeatletter
\AtBeginDocument{%
  \renewcommand*{\AC@hyperlink}[2]{%
    \begingroup
      \hypersetup{hidelinks}%
      \hyperlink{#1}{#2}%
    \endgroup
  }%
}
\makeatother



% ****************************************************************************************************
% ********************************************************************
% Last, but not least...
% ********************************************************************
\usepackage{classicthesis} 
\usepackage{cleveref}
\usepackage{epsdice}
% ****************************************************************************************************


% ****************************************************************************************************
% 8. Further adjustments (experimental)
% ****************************************************************************************************
% ********************************************************************
% Changing the text area
% ********************************************************************
%\linespread{1.05} % a bit more for Palatino
%\areaset[current]{312pt}{761pt} % 686 (factor 2.2) + 33 head + 42 head \the\footskip
%\setlength{\marginparwidth}{7em}%
%\setlength{\marginparsep}{2em}%

% ********************************************************************
% Using different fonts
% ********************************************************************
%\usepackage[oldstylenums]{kpfonts} % oldstyle notextcomp
%\usepackage[osf]{libertine}
%\usepackage[light,condensed,math]{iwona}
%\renewcommand{\sfdefault}{iwona}
%\usepackage{lmodern} % <-- no osf support :-(
%\usepackage{cfr-lm} % 
%\usepackage[urw-garamond]{mathdesign} <-- no osf support :-(
%\usepackage[default,osfigures]{opensans} % scale=0.95 
%\usepackage[sfdefault]{FiraSans}
% ****************************************************************************************************

% ****************************************************************************************************  
% If you like the classicthesis, then I would appreciate a postcard. 
% My address can be found in the file ClassicThesis.pdf. A collection 
% of the postcards I received so far is available online at 
% http://postcards.miede.de
% ****************************************************************************************************


% ****************************************************************************************************
% 0. Set the encoding of your files. UTF-8 is the only sensible encoding nowadays. If you can't read
% äöüßáéçèê∂åëæƒÏ€ then change the encoding setting in your editor, not the line below. If your editor
% does not support utf8 use another editor!
% ****************************************************************************************************
\PassOptionsToPackage{utf8}{inputenc}
	\usepackage{inputenc}

% ****************************************************************************************************
% 1. Configure classicthesis for your needs here, e.g., remove "drafting" below 
% in order to deactivate the time-stamp on the pages
% ****************************************************************************************************
\PassOptionsToPackage{eulerchapternumbers,listings,drafting,%
					 pdfspacing,%floatperchapter,%linedheaders,%
					 subfig,beramono,eulermath,
					 }{classicthesis}                                        
% ********************************************************************
% Available options for classicthesis.sty 
% (see ClassicThesis.pdf for more information):
% drafting
% parts nochapters linedheaders
% eulerchapternumbers beramono eulermath pdfspacing minionprospacing
% tocaligned dottedtoc manychapters
% listings floatperchapter subfig
% ********************************************************************


% ****************************************************************************************************
% 2. Personal data and user ad-hoc commands
% ****************************************************************************************************
\newcommand{\myTitle}{Upper and Lower Bounds on Provably Secure Steganography\xspace}
\newcommand{\mySubtitle}{Inauguraldissertation\xspace}
\newcommand{\myDegree}{Doctor Rerum Naturalium (Dr. rer. nat.)\xspace}
\newcommand{\myName}{Sebastian Berndt\xspace}
\newcommand{\myProf}{Maciej Liśkiewicz\xspace}
\newcommand{\myOtherProf}{Put name here\xspace}
\newcommand{\mySupervisor}{Put name here\xspace}
\newcommand{\myFaculty}{Put data here\xspace}
\newcommand{\myDepartment}{Institute for Theoretical Computer Science\xspace}
\newcommand{\myUni}{University of Lübeck\xspace}
\newcommand{\myLocation}{Lübeck\xspace}
\newcommand{\myTime}{March 2016\xspace}
\newcommand{\myVersion}{version 0.1\xspace}

% ********************************************************************
% Setup, finetuning, and useful commands
% ********************************************************************
\newcounter{dummy} % necessary for correct hyperlinks (to index, bib, etc.)
\newlength{\abcd} % for ab..z string length calculation
\providecommand{\mLyX}{L\kern-.1667em\lower.25em\hbox{Y}\kern-.125emX\@}
\newcommand{\ie}{i.\,e.\xspace}
\newcommand{\Ie}{I.\,e.\xspace}
\newcommand{\eg}{e.\,g.\xspace}
\newcommand{\wlogeneral}{w.\,l.\,o.\,g.\xspace}
\newcommand{\Eg}{E.\,g.\xspace} 
% ****************************************************************************************************


% ****************************************************************************************************
% 3. Loading some handy packages
% ****************************************************************************************************
% ******************************************************************** 
% Packages with options that might require adjustments
% ******************************************************************** 
%\PassOptionsToPackage{ngerman,american}{babel}   % change this to your language(s)
% Spanish languages need extra options in order to work with this template
%\PassOptionsToPackage{spanish,es-lcroman}{babel}
	\usepackage{babel}                  

\usepackage{csquotes}
\PassOptionsToPackage{%
    %backend=biber, %instead of bibtex
	backend=bibtex8,bibencoding=ascii,%
	language=auto,%
%	style=numeric-comp,%
    %style=authoryear-comp, % Author 1999, 2010
        style=alphabetic,
    %bibstyle=authoryear,dashed=false, % dashed: substitute rep. author with ---
    sorting=nyt, % name, year, title
    maxbibnames=10, % default: 3, et al.
    %backref=true,%
    natbib=true % natbib compatibility mode (\citep and \citet still work)
}{biblatex}
    \usepackage{biblatex}

\PassOptionsToPackage{fleqn}{amsmath}       % math environments and more by the AMS 
    \usepackage{amsmath}

% ******************************************************************** 
% General useful packages
% ******************************************************************** 
\PassOptionsToPackage{T1}{fontenc} % T2A for cyrillics
    \usepackage{fontenc}     
\usepackage{textcomp} % fix warning with missing font shapes
\usepackage{scrhack} % fix warnings when using KOMA with listings package          
\usepackage{xspace} % to get the spacing after macros right  
\usepackage{mparhack} % get marginpar right
\usepackage{fixltx2e} % fixes some LaTeX stuff --> since 2015 in the LaTeX kernel (see below)
%\usepackage[latest]{latexrelease} % will be used once available in more distributions (ISSUE #107)
\PassOptionsToPackage{printonlyused,smaller}{acronym} 
    \usepackage{acronym} % nice macros for handling all acronyms in the thesis
    %\renewcommand{\bflabel}[1]{{#1}\hfill} % fix the list of acronyms --> no longer working
    %\renewcommand*{\acsfont}[1]{\textsc{#1}} 
    \renewcommand*{\aclabelfont}[1]{\acsfont{#1}}
% ****************************************************************************************************


% ****************************************************************************************************
% 4. Setup floats: tables, (sub)figures, and captions
% ****************************************************************************************************
\usepackage{tabularx} % better tables
    \setlength{\extrarowheight}{3pt} % increase table row height
\newcommand{\tableheadline}[1]{\multicolumn{1}{c}{\spacedlowsmallcaps{#1}}}
\newcommand{\myfloatalign}{\centering} % to be used with each float for alignment
\usepackage{caption}
% Thanks to cgnieder and Claus Lahiri
% http://tex.stackexchange.com/questions/69349/spacedlowsmallcaps-in-caption-label
% [REMOVED DUE TO OTHER PROBLEMS, SEE ISSUE #82]    
%\DeclareCaptionLabelFormat{smallcaps}{\bothIfFirst{#1}{~}\MakeTextLowercase{\textsc{#2}}}
%\captionsetup{font=small,labelformat=smallcaps} % format=hang,
\captionsetup{font=small} % format=hang,
\usepackage{subfig}  
% ****************************************************************************************************


% ****************************************************************************************************
% 5. Setup code listings
% ****************************************************************************************************
% \usepackage{listings} 
% %\lstset{emph={trueIndex,root},emphstyle=\color{BlueViolet}}%\underbar} % for special keywords
% \lstset{language=[LaTeX]Tex,%C++,
%     morekeywords={PassOptionsToPackage,selectlanguage},
%     keywordstyle=\color{RoyalBlue},%\bfseries,
%     basicstyle=\small\ttfamily,
%     %identifierstyle=\color{NavyBlue},
%     commentstyle=\color{Green}\ttfamily,
%     stringstyle=\rmfamily,
%     numbers=none,%left,%
%     numberstyle=\scriptsize,%\tiny
%     stepnumber=5,
%     numbersep=8pt,
%     showstringspaces=false,
%     breaklines=true,
%     %frameround=ftff,
%     %frame=single,
%     belowcaptionskip=.75\baselineskip
%     %frame=L
% } 
\usepackage[plain]{algorithm}

% Remove the Label "Algorithm" from the algorithms
\captionsetup[algorithm]{labelformat=empty}
% Remove the counter from the algorithms
\renewcommand{\thealgorithm}{}

\usepackage{algpseudocode}
\algrenewcommand\algorithmicrequire{\textbf{Input:}}
\algrenewcommand\algorithmicindent{2em}%

\usepackage{tcolorbox}

% ****************************************************************************************************             


% ****************************************************************************************************
% 6. PDFLaTeX, hyperreferences and citation backreferences
% ****************************************************************************************************
% ********************************************************************
% Using PDFLaTeX
% ********************************************************************
\PassOptionsToPackage{pdftex,hyperfootnotes=false,pdfpagelabels}{hyperref}
    \usepackage{hyperref}  % backref linktocpage pagebackref
\pdfcompresslevel=9
\pdfadjustspacing=1 
\PassOptionsToPackage{pdftex}{graphicx}
    \usepackage{graphicx} 
 

% ********************************************************************
% Hyperreferences
% ********************************************************************
\hypersetup{%
    %draft, % = no hyperlinking at all (useful in b/w printouts)
    colorlinks=true, linktocpage=true, pdfstartpage=3, pdfstartview=FitV,%
    % uncomment the following line if you want to have black links (e.g., for printing)
    %colorlinks=false, linktocpage=false, pdfstartpage=3, pdfstartview=FitV, pdfborder={0 0 0},%
    breaklinks=true, pdfpagemode=UseNone, pageanchor=true, pdfpagemode=UseOutlines,%
    plainpages=false, bookmarksnumbered, bookmarksopen=true, bookmarksopenlevel=1,%
    hypertexnames=true, pdfhighlight=/O,%nesting=true,%frenchlinks,%
    urlcolor=webbrown, linkcolor=RoyalBlue, citecolor=webgreen, %pagecolor=RoyalBlue,%
    %urlcolor=Black, linkcolor=Black, citecolor=Black, %pagecolor=Black,%
    pdftitle={\myTitle},%
    pdfauthor={\textcopyright\ \myName, \myUni, \myFaculty},%
    pdfsubject={},%
    pdfkeywords={},%
    pdfcreator={pdfLaTeX},%
    pdfproducer={LaTeX with hyperref and classicthesis},%
}   

% ********************************************************************
% Setup autoreferences
% ********************************************************************
% There are some issues regarding autorefnames
% http://www.ureader.de/msg/136221647.aspx
% http://www.tex.ac.uk/cgi-bin/texfaq2html?label=latexwords
% you have to redefine the makros for the 
% language you use, e.g., american, ngerman
% (as chosen when loading babel/AtBeginDocument)
% ********************************************************************
\makeatletter
\@ifpackageloaded{babel}%
    {%
       \addto\extrasamerican{%
			\renewcommand*{\figureautorefname}{Figure}%
			\renewcommand*{\tableautorefname}{Table}%
			\renewcommand*{\partautorefname}{Part}%
			\renewcommand*{\chapterautorefname}{Chapter}%
			\renewcommand*{\sectionautorefname}{Section}%
			\renewcommand*{\subsectionautorefname}{Section}%
			\renewcommand*{\subsubsectionautorefname}{Section}%     
                }%
       \addto\extrasngerman{% 
			\renewcommand*{\paragraphautorefname}{Absatz}%
			\renewcommand*{\subparagraphautorefname}{Unterabsatz}%
			\renewcommand*{\footnoteautorefname}{Fu\"snote}%
			\renewcommand*{\FancyVerbLineautorefname}{Zeile}%
			\renewcommand*{\theoremautorefname}{Theorem}%
			\renewcommand*{\appendixautorefname}{Anhang}%
			\renewcommand*{\equationautorefname}{Gleichung}%        
			\renewcommand*{\itemautorefname}{Punkt}%
                }%  
            % Fix to getting autorefs for subfigures right (thanks to Belinda Vogt for changing the definition)
            \providecommand{\subfigureautorefname}{\figureautorefname}%             
    }{\relax}
\makeatother


% ****************************************************************************************************
% 7. Last calls before the bar closes
% ****************************************************************************************************
% ********************************************************************
% Development Stuff
% ********************************************************************
\listfiles
%\PassOptionsToPackage{l2tabu,orthodox,abort}{nag}
%   \usepackage{nag}
%\PassOptionsToPackage{warning, all}{onlyamsmath}
%   \usepackage{onlyamsmath}



% ********************************************************************
% Sebastians Stuff
% ********************************************************************
\usepackage{amssymb} 
\usepackage{amsthm} 
\usepackage{bm}
\usepackage{imakeidx}
\makeindex

\usepackage{tikz}
\usetikzlibrary{positioning}
\usetikzlibrary{decorations.markings}
\usetikzlibrary{arrows.meta}

\usepackage{pgfplots}
\usepgfplotslibrary{fillbetween}
\pgfplotsset{compat=1.12}



\makeatletter

\newcounter{algorithmicH}% New algorithmic-like hyperref counter
\let\oldalgorithmic\algorithmic
\renewcommand{\algorithmic}{%
  \stepcounter{algorithmicH}% Step counter
  \oldalgorithmic}% Do what was always done with algorithmic environment
% \providecommand\theHALG@line{\thealgorithm.\arabic{ALG@line}}
\providecommand\theHALG@line{ALG@line.\thealgorithmicH.\arabic{ALG@line}}
% \renewcommand{\theHALG@line}{ALG@line.\thealgorithmicH.\arabic{ALG@line}}
\makeatother




\newcommand{\myAlgorithm}[4]{
  \begin{algorithm}[H]
\caption{#1: #4}
  \begin{tcolorbox}[title={#1}, colframe=black!10, coltitle=black]
    \begin{algorithmic}[1]
\Require #2
      #3
    \end{algorithmic}
  \end{tcolorbox}
  \end{algorithm}
}
\newcommand{\myNoteHelp}[1]{\marginpar{\textcolor{black}{#1}}}
\newcommand{\myNote}[1]{\myNoteHelp{\textrm{#1}}}

\newcommand{\myDef}[1]{\emph{#1}\myNoteHelp{#1}\index{#1}}
\newcommand{\myDefInformal}[1]{\emph{#1}\myNoteHelp{#1}\index{#1
    (informal)}}

\newcommand{\AlgReturn}[1]{ \textbf{return} #1}
\newcommand{\algo}[1]{ \mathsf{#1}}

\newcommand{\nats}{\mathbb{N}}
\newcommand{\landauO}{\mathcal{O}}
\newcommand{\reals}{\mathbb{R}}
\newcommand{\sgets}{\twoheadleftarrow}
\newcommand{\rats}{\mathbb{Q}}
\newcommand{\concat}{\mid \mid}
\newcommand{\powerset}{\mathcal{P}}
\newcommand{\machine}{\mathsf{M}}
\newcommand{\chan}{\mathcal{C}}
\newcommand{\stegos}{\mathcal{S}}
\newcommand{\DomainHistory}{\left(\Sigma^{n(\kappa)}\right)^{*}}
\DeclareMathOperator{\dom}{dom}
\DeclareMathOperator{\fout}{out}
\DeclareMathOperator{\fin}{in}
\DeclareMathOperator{\supp}{supp}
\DeclareMathOperator{\Exp}{Exp}
\DeclareMathOperator{\img}{img}
\DeclareMathOperator{\query}{query}
\DeclareMathOperator{\rate}{rate}
\DeclareMathOperator{\collision}{Collision}
\DeclareMathOperator{\Coll}{\mathsf{Coll}}
\DeclareMathOperator{\Sig-Forge}{\mathsf{Sig-Forge}}
\DeclareMathOperator{\DDisting}{\mathsf{Dist-Disting}}
\DeclareMathOperator{\SSCHA-Dist}{\mathsf{SS-CHA-Dist}}
\DeclareMathOperator{\SSCCA-Dist}{\mathsf{SS-CCA-Dist}}
\DeclareMathOperator{\SSRCCA-Dist}{\mathsf{SS-RCCA-Dist}}
\DeclareMathOperator{\CPA-Dist}{\mathsf{CPA-Dist}}
\DeclareMathOperator{\CPAD-Dist}{\mathsf{CPA\$-Dist}}
\DeclareMathOperator{\CCA-Dist}{\mathsf{CCA-Dist}}
\DeclareMathOperator{\CCAD-Dist}{\mathsf{CCA\$-Dist}}
\DeclareMathOperator{\negl}{\mathsf{negl}}
\DeclareMathOperator{\poly}{\mathsf{poly}}
\DeclareMathOperator{\lin}{\mathsf{lin}}

\DeclareMathOperator{\Fun}{Fun}
\DeclareMathOperator{\Pics}{Pics}

\DeclareMathOperator{\Gen}{\mathsf{Gen}}
\DeclareMathOperator{\Inv}{\mathsf{Inv}}
\DeclareMathOperator{\algf}{\algo{F}}
\DeclareMathOperator{\algh}{\algo{H}}
\DeclareMathOperator{\Fi}{\mathsf{Fi}}
\DeclareMathOperator{\Dist}{\mathsf{Dist}}
\DeclareMathOperator{\DDist}{\mathsf{DDist}}
\DeclareMathOperator{\Sign}{\mathsf{Sign}}
\DeclareMathOperator{\Vrfy}{\mathsf{Vrfy}}
\DeclareMathOperator{\Forg}{\mathsf{Fo}}
\DeclareMathOperator{\Enc}{\mathsf{Enc}}
\DeclareMathOperator{\Dec}{\mathsf{Dec}}
\DeclareMathOperator{\PKEnc}{\mathsf{PKEnc}}
\DeclareMathOperator{\PKDec}{\mathsf{PKDec}}
\DeclareMathOperator{\SEnc}{\mathsf{SEnc}}
\DeclareMathOperator{\Steg}{\mathsf{S}}
\DeclareMathOperator{\SDec}{\mathsf{SDec}}
\DeclareMathOperator{\Att}{\mathsf{A}}
\DeclareMathOperator{\Ward}{\mathsf{W}}
\DeclareMathOperator{\RejSam}{\mathsf{RejSam}}

\newcommand{\pk}{\textit{pk}}
\newcommand{\sk}{\textit{sk}}







\DeclareMathOperator{\Adv}{\mathbf{Adv}}
\DeclareMathOperator{\InSec}{\mathbf{InSec}}
\DeclareMathOperator{\unrel}{\mathbf{UnRel}}
\DeclareMathOperator{\hash}{hash}
\DeclareMathOperator{\prf}{prf}
\DeclareMathOperator{\sig}{sig}
\DeclareMathOperator{\cpa}{cpa}
\DeclareMathOperator{\dist}{dist}
\DeclareMathOperator{\sscha}{ss-cha}
\DeclareMathOperator{\sscca}{ss-cca}
\DeclareMathOperator{\ssrcca}{ss-rcca}
\DeclareMathOperator{\cpad}{cpa\scalebox{.7}{\$}}
\DeclareMathOperator{\cca}{cca}
\DeclareMathOperator{\ccad}{cca\scalebox{.7}{\$}}

\newcommand{\minent}{H_{\infty}}

\newcommand{\examplesymbol}{\( \diamond \)}


\newtheorem{theorem}{Theorem}
\newtheorem{corollary}[theorem]{Corollary}


\theoremstyle{definition}
\newtheorem*{exampleth}{Example}
\newenvironment{example}{\begin{exampleth}%
  \renewcommand{\qedsymbol}{\examplesymbol}\pushQED{\qed}}%
  {\popQED\end{exampleth}}



\makeatletter
\AtBeginDocument{%
  \renewcommand*{\AC@hyperlink}[2]{%
    \begingroup
      \hypersetup{hidelinks}%
      \hyperlink{#1}{#2}%
    \endgroup
  }%
}
\makeatother



% ****************************************************************************************************
% ********************************************************************
% Last, but not least...
% ********************************************************************
\usepackage{classicthesis} 
\usepackage{cleveref}
\usepackage{epsdice}
% ****************************************************************************************************


% ****************************************************************************************************
% 8. Further adjustments (experimental)
% ****************************************************************************************************
% ********************************************************************
% Changing the text area
% ********************************************************************
%\linespread{1.05} % a bit more for Palatino
%\areaset[current]{312pt}{761pt} % 686 (factor 2.2) + 33 head + 42 head \the\footskip
%\setlength{\marginparwidth}{7em}%
%\setlength{\marginparsep}{2em}%

% ********************************************************************
% Using different fonts
% ********************************************************************
%\usepackage[oldstylenums]{kpfonts} % oldstyle notextcomp
%\usepackage[osf]{libertine}
%\usepackage[light,condensed,math]{iwona}
%\renewcommand{\sfdefault}{iwona}
%\usepackage{lmodern} % <-- no osf support :-(
%\usepackage{cfr-lm} % 
%\usepackage[urw-garamond]{mathdesign} <-- no osf support :-(
%\usepackage[default,osfigures]{opensans} % scale=0.95 
%\usepackage[sfdefault]{FiraSans}
% ****************************************************************************************************

% ****************************************************************************************************  
% If you like the classicthesis, then I would appreciate a postcard. 
% My address can be found in the file ClassicThesis.pdf. A collection 
% of the postcards I received so far is available online at 
% http://postcards.miede.de
% ****************************************************************************************************


% ****************************************************************************************************
% 0. Set the encoding of your files. UTF-8 is the only sensible encoding nowadays. If you can't read
% äöüßáéçèê∂åëæƒÏ€ then change the encoding setting in your editor, not the line below. If your editor
% does not support utf8 use another editor!
% ****************************************************************************************************
\PassOptionsToPackage{utf8}{inputenc}
	\usepackage{inputenc}

% ****************************************************************************************************
% 1. Configure classicthesis for your needs here, e.g., remove "drafting" below 
% in order to deactivate the time-stamp on the pages
% ****************************************************************************************************
\PassOptionsToPackage{eulerchapternumbers,listings,drafting,%
					 pdfspacing,%floatperchapter,%linedheaders,%
					 subfig,beramono,eulermath,
					 }{classicthesis}                                        
% ********************************************************************
% Available options for classicthesis.sty 
% (see ClassicThesis.pdf for more information):
% drafting
% parts nochapters linedheaders
% eulerchapternumbers beramono eulermath pdfspacing minionprospacing
% tocaligned dottedtoc manychapters
% listings floatperchapter subfig
% ********************************************************************


% ****************************************************************************************************
% 2. Personal data and user ad-hoc commands
% ****************************************************************************************************
\newcommand{\myTitle}{Upper and Lower Bounds on Provably Secure Steganography\xspace}
\newcommand{\mySubtitle}{Inauguraldissertation\xspace}
\newcommand{\myDegree}{Doctor Rerum Naturalium (Dr. rer. nat.)\xspace}
\newcommand{\myName}{Sebastian Berndt\xspace}
\newcommand{\myProf}{Maciej Liśkiewicz\xspace}
\newcommand{\myOtherProf}{Put name here\xspace}
\newcommand{\mySupervisor}{Put name here\xspace}
\newcommand{\myFaculty}{Put data here\xspace}
\newcommand{\myDepartment}{Institute for Theoretical Computer Science\xspace}
\newcommand{\myUni}{University of Lübeck\xspace}
\newcommand{\myLocation}{Lübeck\xspace}
\newcommand{\myTime}{March 2016\xspace}
\newcommand{\myVersion}{version 0.1\xspace}

% ********************************************************************
% Setup, finetuning, and useful commands
% ********************************************************************
\newcounter{dummy} % necessary for correct hyperlinks (to index, bib, etc.)
\newlength{\abcd} % for ab..z string length calculation
\providecommand{\mLyX}{L\kern-.1667em\lower.25em\hbox{Y}\kern-.125emX\@}
\newcommand{\ie}{i.\,e.\xspace}
\newcommand{\Ie}{I.\,e.\xspace}
\newcommand{\eg}{e.\,g.\xspace}
\newcommand{\wlogeneral}{w.\,l.\,o.\,g.\xspace}
\newcommand{\Eg}{E.\,g.\xspace} 
% ****************************************************************************************************


% ****************************************************************************************************
% 3. Loading some handy packages
% ****************************************************************************************************
% ******************************************************************** 
% Packages with options that might require adjustments
% ******************************************************************** 
%\PassOptionsToPackage{ngerman,american}{babel}   % change this to your language(s)
% Spanish languages need extra options in order to work with this template
%\PassOptionsToPackage{spanish,es-lcroman}{babel}
	\usepackage{babel}                  

\usepackage{csquotes}
\PassOptionsToPackage{%
    %backend=biber, %instead of bibtex
	backend=bibtex8,bibencoding=ascii,%
	language=auto,%
%	style=numeric-comp,%
    %style=authoryear-comp, % Author 1999, 2010
        style=alphabetic,
    %bibstyle=authoryear,dashed=false, % dashed: substitute rep. author with ---
    sorting=nyt, % name, year, title
    maxbibnames=10, % default: 3, et al.
    %backref=true,%
    natbib=true % natbib compatibility mode (\citep and \citet still work)
}{biblatex}
    \usepackage{biblatex}

\PassOptionsToPackage{fleqn}{amsmath}       % math environments and more by the AMS 
    \usepackage{amsmath}

% ******************************************************************** 
% General useful packages
% ******************************************************************** 
\PassOptionsToPackage{T1}{fontenc} % T2A for cyrillics
    \usepackage{fontenc}     
\usepackage{textcomp} % fix warning with missing font shapes
\usepackage{scrhack} % fix warnings when using KOMA with listings package          
\usepackage{xspace} % to get the spacing after macros right  
\usepackage{mparhack} % get marginpar right
\usepackage{fixltx2e} % fixes some LaTeX stuff --> since 2015 in the LaTeX kernel (see below)
%\usepackage[latest]{latexrelease} % will be used once available in more distributions (ISSUE #107)
\PassOptionsToPackage{printonlyused,smaller}{acronym} 
    \usepackage{acronym} % nice macros for handling all acronyms in the thesis
    %\renewcommand{\bflabel}[1]{{#1}\hfill} % fix the list of acronyms --> no longer working
    %\renewcommand*{\acsfont}[1]{\textsc{#1}} 
    \renewcommand*{\aclabelfont}[1]{\acsfont{#1}}
% ****************************************************************************************************


% ****************************************************************************************************
% 4. Setup floats: tables, (sub)figures, and captions
% ****************************************************************************************************
\usepackage{tabularx} % better tables
    \setlength{\extrarowheight}{3pt} % increase table row height
\newcommand{\tableheadline}[1]{\multicolumn{1}{c}{\spacedlowsmallcaps{#1}}}
\newcommand{\myfloatalign}{\centering} % to be used with each float for alignment
\usepackage{caption}
% Thanks to cgnieder and Claus Lahiri
% http://tex.stackexchange.com/questions/69349/spacedlowsmallcaps-in-caption-label
% [REMOVED DUE TO OTHER PROBLEMS, SEE ISSUE #82]    
%\DeclareCaptionLabelFormat{smallcaps}{\bothIfFirst{#1}{~}\MakeTextLowercase{\textsc{#2}}}
%\captionsetup{font=small,labelformat=smallcaps} % format=hang,
\captionsetup{font=small} % format=hang,
\usepackage{subfig}  
% ****************************************************************************************************


% ****************************************************************************************************
% 5. Setup code listings
% ****************************************************************************************************
% \usepackage{listings} 
% %\lstset{emph={trueIndex,root},emphstyle=\color{BlueViolet}}%\underbar} % for special keywords
% \lstset{language=[LaTeX]Tex,%C++,
%     morekeywords={PassOptionsToPackage,selectlanguage},
%     keywordstyle=\color{RoyalBlue},%\bfseries,
%     basicstyle=\small\ttfamily,
%     %identifierstyle=\color{NavyBlue},
%     commentstyle=\color{Green}\ttfamily,
%     stringstyle=\rmfamily,
%     numbers=none,%left,%
%     numberstyle=\scriptsize,%\tiny
%     stepnumber=5,
%     numbersep=8pt,
%     showstringspaces=false,
%     breaklines=true,
%     %frameround=ftff,
%     %frame=single,
%     belowcaptionskip=.75\baselineskip
%     %frame=L
% } 
\usepackage[plain]{algorithm}

% Remove the Label "Algorithm" from the algorithms
\captionsetup[algorithm]{labelformat=empty}
% Remove the counter from the algorithms
\renewcommand{\thealgorithm}{}

\usepackage{algpseudocode}
\algrenewcommand\algorithmicrequire{\textbf{Input:}}
\algrenewcommand\algorithmicindent{2em}%

\usepackage{tcolorbox}

% ****************************************************************************************************             


% ****************************************************************************************************
% 6. PDFLaTeX, hyperreferences and citation backreferences
% ****************************************************************************************************
% ********************************************************************
% Using PDFLaTeX
% ********************************************************************
\PassOptionsToPackage{pdftex,hyperfootnotes=false,pdfpagelabels}{hyperref}
    \usepackage{hyperref}  % backref linktocpage pagebackref
\pdfcompresslevel=9
\pdfadjustspacing=1 
\PassOptionsToPackage{pdftex}{graphicx}
    \usepackage{graphicx} 
 

% ********************************************************************
% Hyperreferences
% ********************************************************************
\hypersetup{%
    %draft, % = no hyperlinking at all (useful in b/w printouts)
    colorlinks=true, linktocpage=true, pdfstartpage=3, pdfstartview=FitV,%
    % uncomment the following line if you want to have black links (e.g., for printing)
    %colorlinks=false, linktocpage=false, pdfstartpage=3, pdfstartview=FitV, pdfborder={0 0 0},%
    breaklinks=true, pdfpagemode=UseNone, pageanchor=true, pdfpagemode=UseOutlines,%
    plainpages=false, bookmarksnumbered, bookmarksopen=true, bookmarksopenlevel=1,%
    hypertexnames=true, pdfhighlight=/O,%nesting=true,%frenchlinks,%
    urlcolor=webbrown, linkcolor=RoyalBlue, citecolor=webgreen, %pagecolor=RoyalBlue,%
    %urlcolor=Black, linkcolor=Black, citecolor=Black, %pagecolor=Black,%
    pdftitle={\myTitle},%
    pdfauthor={\textcopyright\ \myName, \myUni, \myFaculty},%
    pdfsubject={},%
    pdfkeywords={},%
    pdfcreator={pdfLaTeX},%
    pdfproducer={LaTeX with hyperref and classicthesis},%
}   

% ********************************************************************
% Setup autoreferences
% ********************************************************************
% There are some issues regarding autorefnames
% http://www.ureader.de/msg/136221647.aspx
% http://www.tex.ac.uk/cgi-bin/texfaq2html?label=latexwords
% you have to redefine the makros for the 
% language you use, e.g., american, ngerman
% (as chosen when loading babel/AtBeginDocument)
% ********************************************************************
\makeatletter
\@ifpackageloaded{babel}%
    {%
       \addto\extrasamerican{%
			\renewcommand*{\figureautorefname}{Figure}%
			\renewcommand*{\tableautorefname}{Table}%
			\renewcommand*{\partautorefname}{Part}%
			\renewcommand*{\chapterautorefname}{Chapter}%
			\renewcommand*{\sectionautorefname}{Section}%
			\renewcommand*{\subsectionautorefname}{Section}%
			\renewcommand*{\subsubsectionautorefname}{Section}%     
                }%
       \addto\extrasngerman{% 
			\renewcommand*{\paragraphautorefname}{Absatz}%
			\renewcommand*{\subparagraphautorefname}{Unterabsatz}%
			\renewcommand*{\footnoteautorefname}{Fu\"snote}%
			\renewcommand*{\FancyVerbLineautorefname}{Zeile}%
			\renewcommand*{\theoremautorefname}{Theorem}%
			\renewcommand*{\appendixautorefname}{Anhang}%
			\renewcommand*{\equationautorefname}{Gleichung}%        
			\renewcommand*{\itemautorefname}{Punkt}%
                }%  
            % Fix to getting autorefs for subfigures right (thanks to Belinda Vogt for changing the definition)
            \providecommand{\subfigureautorefname}{\figureautorefname}%             
    }{\relax}
\makeatother


% ****************************************************************************************************
% 7. Last calls before the bar closes
% ****************************************************************************************************
% ********************************************************************
% Development Stuff
% ********************************************************************
\listfiles
%\PassOptionsToPackage{l2tabu,orthodox,abort}{nag}
%   \usepackage{nag}
%\PassOptionsToPackage{warning, all}{onlyamsmath}
%   \usepackage{onlyamsmath}



% ********************************************************************
% Sebastians Stuff
% ********************************************************************
\usepackage{amssymb} 
\usepackage{amsthm} 
\usepackage{bm}
\usepackage{imakeidx}
\makeindex

\usepackage{tikz}
\usetikzlibrary{positioning}
\usetikzlibrary{decorations.markings}
\usetikzlibrary{arrows.meta}

\usepackage{pgfplots}
\usepgfplotslibrary{fillbetween}
\pgfplotsset{compat=1.12}



\makeatletter

\newcounter{algorithmicH}% New algorithmic-like hyperref counter
\let\oldalgorithmic\algorithmic
\renewcommand{\algorithmic}{%
  \stepcounter{algorithmicH}% Step counter
  \oldalgorithmic}% Do what was always done with algorithmic environment
% \providecommand\theHALG@line{\thealgorithm.\arabic{ALG@line}}
\providecommand\theHALG@line{ALG@line.\thealgorithmicH.\arabic{ALG@line}}
% \renewcommand{\theHALG@line}{ALG@line.\thealgorithmicH.\arabic{ALG@line}}
\makeatother




\newcommand{\myAlgorithm}[4]{
  \begin{algorithm}[H]
\caption{#1: #4}
  \begin{tcolorbox}[title={#1}, colframe=black!10, coltitle=black]
    \begin{algorithmic}[1]
\Require #2
      #3
    \end{algorithmic}
  \end{tcolorbox}
  \end{algorithm}
}
\newcommand{\myNoteHelp}[1]{\marginpar{\textcolor{black}{#1}}}
\newcommand{\myNote}[1]{\myNoteHelp{\textrm{#1}}}

\newcommand{\myDef}[1]{\emph{#1}\myNoteHelp{#1}\index{#1}}
\newcommand{\myDefInformal}[1]{\emph{#1}\myNoteHelp{#1}\index{#1
    (informal)}}

\newcommand{\AlgReturn}[1]{ \textbf{return} #1}
\newcommand{\algo}[1]{ \mathsf{#1}}

\newcommand{\nats}{\mathbb{N}}
\newcommand{\landauO}{\mathcal{O}}
\newcommand{\reals}{\mathbb{R}}
\newcommand{\sgets}{\twoheadleftarrow}
\newcommand{\rats}{\mathbb{Q}}
\newcommand{\concat}{\mid \mid}
\newcommand{\powerset}{\mathcal{P}}
\newcommand{\machine}{\mathsf{M}}
\newcommand{\chan}{\mathcal{C}}
\newcommand{\stegos}{\mathcal{S}}
\newcommand{\DomainHistory}{\left(\Sigma^{n(\kappa)}\right)^{*}}
\DeclareMathOperator{\dom}{dom}
\DeclareMathOperator{\fout}{out}
\DeclareMathOperator{\fin}{in}
\DeclareMathOperator{\supp}{supp}
\DeclareMathOperator{\Exp}{Exp}
\DeclareMathOperator{\img}{img}
\DeclareMathOperator{\query}{query}
\DeclareMathOperator{\rate}{rate}
\DeclareMathOperator{\collision}{Collision}
\DeclareMathOperator{\Coll}{\mathsf{Coll}}
\DeclareMathOperator{\Sig-Forge}{\mathsf{Sig-Forge}}
\DeclareMathOperator{\DDisting}{\mathsf{Dist-Disting}}
\DeclareMathOperator{\SSCHA-Dist}{\mathsf{SS-CHA-Dist}}
\DeclareMathOperator{\SSCCA-Dist}{\mathsf{SS-CCA-Dist}}
\DeclareMathOperator{\SSRCCA-Dist}{\mathsf{SS-RCCA-Dist}}
\DeclareMathOperator{\CPA-Dist}{\mathsf{CPA-Dist}}
\DeclareMathOperator{\CPAD-Dist}{\mathsf{CPA\$-Dist}}
\DeclareMathOperator{\CCA-Dist}{\mathsf{CCA-Dist}}
\DeclareMathOperator{\CCAD-Dist}{\mathsf{CCA\$-Dist}}
\DeclareMathOperator{\negl}{\mathsf{negl}}
\DeclareMathOperator{\poly}{\mathsf{poly}}
\DeclareMathOperator{\lin}{\mathsf{lin}}

\DeclareMathOperator{\Fun}{Fun}
\DeclareMathOperator{\Pics}{Pics}

\DeclareMathOperator{\Gen}{\mathsf{Gen}}
\DeclareMathOperator{\Inv}{\mathsf{Inv}}
\DeclareMathOperator{\algf}{\algo{F}}
\DeclareMathOperator{\algh}{\algo{H}}
\DeclareMathOperator{\Fi}{\mathsf{Fi}}
\DeclareMathOperator{\Dist}{\mathsf{Dist}}
\DeclareMathOperator{\DDist}{\mathsf{DDist}}
\DeclareMathOperator{\Sign}{\mathsf{Sign}}
\DeclareMathOperator{\Vrfy}{\mathsf{Vrfy}}
\DeclareMathOperator{\Forg}{\mathsf{Fo}}
\DeclareMathOperator{\Enc}{\mathsf{Enc}}
\DeclareMathOperator{\Dec}{\mathsf{Dec}}
\DeclareMathOperator{\PKEnc}{\mathsf{PKEnc}}
\DeclareMathOperator{\PKDec}{\mathsf{PKDec}}
\DeclareMathOperator{\SEnc}{\mathsf{SEnc}}
\DeclareMathOperator{\Steg}{\mathsf{S}}
\DeclareMathOperator{\SDec}{\mathsf{SDec}}
\DeclareMathOperator{\Att}{\mathsf{A}}
\DeclareMathOperator{\Ward}{\mathsf{W}}
\DeclareMathOperator{\RejSam}{\mathsf{RejSam}}

\newcommand{\pk}{\textit{pk}}
\newcommand{\sk}{\textit{sk}}







\DeclareMathOperator{\Adv}{\mathbf{Adv}}
\DeclareMathOperator{\InSec}{\mathbf{InSec}}
\DeclareMathOperator{\unrel}{\mathbf{UnRel}}
\DeclareMathOperator{\hash}{hash}
\DeclareMathOperator{\prf}{prf}
\DeclareMathOperator{\sig}{sig}
\DeclareMathOperator{\cpa}{cpa}
\DeclareMathOperator{\dist}{dist}
\DeclareMathOperator{\sscha}{ss-cha}
\DeclareMathOperator{\sscca}{ss-cca}
\DeclareMathOperator{\ssrcca}{ss-rcca}
\DeclareMathOperator{\cpad}{cpa\scalebox{.7}{\$}}
\DeclareMathOperator{\cca}{cca}
\DeclareMathOperator{\ccad}{cca\scalebox{.7}{\$}}

\newcommand{\minent}{H_{\infty}}

\newcommand{\examplesymbol}{\( \diamond \)}


\newtheorem{theorem}{Theorem}
\newtheorem{corollary}[theorem]{Corollary}


\theoremstyle{definition}
\newtheorem*{exampleth}{Example}
\newenvironment{example}{\begin{exampleth}%
  \renewcommand{\qedsymbol}{\examplesymbol}\pushQED{\qed}}%
  {\popQED\end{exampleth}}



\makeatletter
\AtBeginDocument{%
  \renewcommand*{\AC@hyperlink}[2]{%
    \begingroup
      \hypersetup{hidelinks}%
      \hyperlink{#1}{#2}%
    \endgroup
  }%
}
\makeatother



% ****************************************************************************************************
% ********************************************************************
% Last, but not least...
% ********************************************************************
\usepackage{classicthesis} 
\usepackage{cleveref}
\usepackage{epsdice}
% ****************************************************************************************************


% ****************************************************************************************************
% 8. Further adjustments (experimental)
% ****************************************************************************************************
% ********************************************************************
% Changing the text area
% ********************************************************************
%\linespread{1.05} % a bit more for Palatino
%\areaset[current]{312pt}{761pt} % 686 (factor 2.2) + 33 head + 42 head \the\footskip
%\setlength{\marginparwidth}{7em}%
%\setlength{\marginparsep}{2em}%

% ********************************************************************
% Using different fonts
% ********************************************************************
%\usepackage[oldstylenums]{kpfonts} % oldstyle notextcomp
%\usepackage[osf]{libertine}
%\usepackage[light,condensed,math]{iwona}
%\renewcommand{\sfdefault}{iwona}
%\usepackage{lmodern} % <-- no osf support :-(
%\usepackage{cfr-lm} % 
%\usepackage[urw-garamond]{mathdesign} <-- no osf support :-(
%\usepackage[default,osfigures]{opensans} % scale=0.95 
%\usepackage[sfdefault]{FiraSans}
% ****************************************************************************************************

% ****************************************************************************************************  
% If you like the classicthesis, then I would appreciate a postcard. 
% My address can be found in the file ClassicThesis.pdf. A collection 
% of the postcards I received so far is available online at 
% http://postcards.miede.de
% ****************************************************************************************************


% ****************************************************************************************************
% 0. Set the encoding of your files. UTF-8 is the only sensible encoding nowadays. If you can't read
% äöüßáéçèê∂åëæƒÏ€ then change the encoding setting in your editor, not the line below. If your editor
% does not support utf8 use another editor!
% ****************************************************************************************************
\PassOptionsToPackage{utf8}{inputenc}
	\usepackage{inputenc}

% ****************************************************************************************************
% 1. Configure classicthesis for your needs here, e.g., remove "drafting" below 
% in order to deactivate the time-stamp on the pages
% ****************************************************************************************************
\PassOptionsToPackage{eulerchapternumbers,listings,drafting,%
					 pdfspacing,%floatperchapter,%linedheaders,%
					 subfig,beramono,eulermath,
					 }{classicthesis}                                        
% ********************************************************************
% Available options for classicthesis.sty 
% (see ClassicThesis.pdf for more information):
% drafting
% parts nochapters linedheaders
% eulerchapternumbers beramono eulermath pdfspacing minionprospacing
% tocaligned dottedtoc manychapters
% listings floatperchapter subfig
% ********************************************************************


% ****************************************************************************************************
% 2. Personal data and user ad-hoc commands
% ****************************************************************************************************
\newcommand{\myTitle}{Upper and Lower Bounds on Provably Secure Steganography\xspace}
\newcommand{\mySubtitle}{Inauguraldissertation\xspace}
\newcommand{\myDegree}{Doctor Rerum Naturalium (Dr. rer. nat.)\xspace}
\newcommand{\myName}{Sebastian Berndt\xspace}
\newcommand{\myProf}{Maciej Liśkiewicz\xspace}
\newcommand{\myOtherProf}{Put name here\xspace}
\newcommand{\mySupervisor}{Put name here\xspace}
\newcommand{\myFaculty}{Put data here\xspace}
\newcommand{\myDepartment}{Institute for Theoretical Computer Science\xspace}
\newcommand{\myUni}{University of Lübeck\xspace}
\newcommand{\myLocation}{Lübeck\xspace}
\newcommand{\myTime}{March 2016\xspace}
\newcommand{\myVersion}{version 0.1\xspace}

% ********************************************************************
% Setup, finetuning, and useful commands
% ********************************************************************
\newcounter{dummy} % necessary for correct hyperlinks (to index, bib, etc.)
\newlength{\abcd} % for ab..z string length calculation
\providecommand{\mLyX}{L\kern-.1667em\lower.25em\hbox{Y}\kern-.125emX\@}
\newcommand{\ie}{i.\,e.\xspace}
\newcommand{\Ie}{I.\,e.\xspace}
\newcommand{\eg}{e.\,g.\xspace}
\newcommand{\wlogeneral}{w.\,l.\,o.\,g.\xspace}
\newcommand{\Eg}{E.\,g.\xspace} 
% ****************************************************************************************************


% ****************************************************************************************************
% 3. Loading some handy packages
% ****************************************************************************************************
% ******************************************************************** 
% Packages with options that might require adjustments
% ******************************************************************** 
%\PassOptionsToPackage{ngerman,american}{babel}   % change this to your language(s)
% Spanish languages need extra options in order to work with this template
%\PassOptionsToPackage{spanish,es-lcroman}{babel}
	\usepackage{babel}                  

\usepackage{csquotes}
\PassOptionsToPackage{%
    %backend=biber, %instead of bibtex
	backend=bibtex8,bibencoding=ascii,%
	language=auto,%
%	style=numeric-comp,%
    %style=authoryear-comp, % Author 1999, 2010
        style=alphabetic,
    %bibstyle=authoryear,dashed=false, % dashed: substitute rep. author with ---
    sorting=nyt, % name, year, title
    maxbibnames=10, % default: 3, et al.
    %backref=true,%
    natbib=true % natbib compatibility mode (\citep and \citet still work)
}{biblatex}
    \usepackage{biblatex}

\PassOptionsToPackage{fleqn}{amsmath}       % math environments and more by the AMS 
    \usepackage{amsmath}

% ******************************************************************** 
% General useful packages
% ******************************************************************** 
\PassOptionsToPackage{T1}{fontenc} % T2A for cyrillics
    \usepackage{fontenc}     
\usepackage{textcomp} % fix warning with missing font shapes
\usepackage{scrhack} % fix warnings when using KOMA with listings package          
\usepackage{xspace} % to get the spacing after macros right  
\usepackage{mparhack} % get marginpar right
\usepackage{fixltx2e} % fixes some LaTeX stuff --> since 2015 in the LaTeX kernel (see below)
%\usepackage[latest]{latexrelease} % will be used once available in more distributions (ISSUE #107)
\PassOptionsToPackage{printonlyused,smaller}{acronym} 
    \usepackage{acronym} % nice macros for handling all acronyms in the thesis
    %\renewcommand{\bflabel}[1]{{#1}\hfill} % fix the list of acronyms --> no longer working
    %\renewcommand*{\acsfont}[1]{\textsc{#1}} 
    \renewcommand*{\aclabelfont}[1]{\acsfont{#1}}
% ****************************************************************************************************


% ****************************************************************************************************
% 4. Setup floats: tables, (sub)figures, and captions
% ****************************************************************************************************
\usepackage{tabularx} % better tables
    \setlength{\extrarowheight}{3pt} % increase table row height
\newcommand{\tableheadline}[1]{\multicolumn{1}{c}{\spacedlowsmallcaps{#1}}}
\newcommand{\myfloatalign}{\centering} % to be used with each float for alignment
\usepackage{caption}
% Thanks to cgnieder and Claus Lahiri
% http://tex.stackexchange.com/questions/69349/spacedlowsmallcaps-in-caption-label
% [REMOVED DUE TO OTHER PROBLEMS, SEE ISSUE #82]    
%\DeclareCaptionLabelFormat{smallcaps}{\bothIfFirst{#1}{~}\MakeTextLowercase{\textsc{#2}}}
%\captionsetup{font=small,labelformat=smallcaps} % format=hang,
\captionsetup{font=small} % format=hang,
\usepackage{subfig}  
% ****************************************************************************************************


% ****************************************************************************************************
% 5. Setup code listings
% ****************************************************************************************************
% \usepackage{listings} 
% %\lstset{emph={trueIndex,root},emphstyle=\color{BlueViolet}}%\underbar} % for special keywords
% \lstset{language=[LaTeX]Tex,%C++,
%     morekeywords={PassOptionsToPackage,selectlanguage},
%     keywordstyle=\color{RoyalBlue},%\bfseries,
%     basicstyle=\small\ttfamily,
%     %identifierstyle=\color{NavyBlue},
%     commentstyle=\color{Green}\ttfamily,
%     stringstyle=\rmfamily,
%     numbers=none,%left,%
%     numberstyle=\scriptsize,%\tiny
%     stepnumber=5,
%     numbersep=8pt,
%     showstringspaces=false,
%     breaklines=true,
%     %frameround=ftff,
%     %frame=single,
%     belowcaptionskip=.75\baselineskip
%     %frame=L
% } 
\usepackage[plain]{algorithm}

% Remove the Label "Algorithm" from the algorithms
\captionsetup[algorithm]{labelformat=empty}
% Remove the counter from the algorithms
\renewcommand{\thealgorithm}{}

\usepackage{algpseudocode}
\algrenewcommand\algorithmicrequire{\textbf{Input:}}
\algrenewcommand\algorithmicindent{2em}%

\usepackage{tcolorbox}

% ****************************************************************************************************             


% ****************************************************************************************************
% 6. PDFLaTeX, hyperreferences and citation backreferences
% ****************************************************************************************************
% ********************************************************************
% Using PDFLaTeX
% ********************************************************************
\PassOptionsToPackage{pdftex,hyperfootnotes=false,pdfpagelabels}{hyperref}
    \usepackage{hyperref}  % backref linktocpage pagebackref
\pdfcompresslevel=9
\pdfadjustspacing=1 
\PassOptionsToPackage{pdftex}{graphicx}
    \usepackage{graphicx} 
 

% ********************************************************************
% Hyperreferences
% ********************************************************************
\hypersetup{%
    %draft, % = no hyperlinking at all (useful in b/w printouts)
    colorlinks=true, linktocpage=true, pdfstartpage=3, pdfstartview=FitV,%
    % uncomment the following line if you want to have black links (e.g., for printing)
    %colorlinks=false, linktocpage=false, pdfstartpage=3, pdfstartview=FitV, pdfborder={0 0 0},%
    breaklinks=true, pdfpagemode=UseNone, pageanchor=true, pdfpagemode=UseOutlines,%
    plainpages=false, bookmarksnumbered, bookmarksopen=true, bookmarksopenlevel=1,%
    hypertexnames=true, pdfhighlight=/O,%nesting=true,%frenchlinks,%
    urlcolor=webbrown, linkcolor=RoyalBlue, citecolor=webgreen, %pagecolor=RoyalBlue,%
    %urlcolor=Black, linkcolor=Black, citecolor=Black, %pagecolor=Black,%
    pdftitle={\myTitle},%
    pdfauthor={\textcopyright\ \myName, \myUni, \myFaculty},%
    pdfsubject={},%
    pdfkeywords={},%
    pdfcreator={pdfLaTeX},%
    pdfproducer={LaTeX with hyperref and classicthesis},%
}   

% ********************************************************************
% Setup autoreferences
% ********************************************************************
% There are some issues regarding autorefnames
% http://www.ureader.de/msg/136221647.aspx
% http://www.tex.ac.uk/cgi-bin/texfaq2html?label=latexwords
% you have to redefine the makros for the 
% language you use, e.g., american, ngerman
% (as chosen when loading babel/AtBeginDocument)
% ********************************************************************
\makeatletter
\@ifpackageloaded{babel}%
    {%
       \addto\extrasamerican{%
			\renewcommand*{\figureautorefname}{Figure}%
			\renewcommand*{\tableautorefname}{Table}%
			\renewcommand*{\partautorefname}{Part}%
			\renewcommand*{\chapterautorefname}{Chapter}%
			\renewcommand*{\sectionautorefname}{Section}%
			\renewcommand*{\subsectionautorefname}{Section}%
			\renewcommand*{\subsubsectionautorefname}{Section}%     
                }%
       \addto\extrasngerman{% 
			\renewcommand*{\paragraphautorefname}{Absatz}%
			\renewcommand*{\subparagraphautorefname}{Unterabsatz}%
			\renewcommand*{\footnoteautorefname}{Fu\"snote}%
			\renewcommand*{\FancyVerbLineautorefname}{Zeile}%
			\renewcommand*{\theoremautorefname}{Theorem}%
			\renewcommand*{\appendixautorefname}{Anhang}%
			\renewcommand*{\equationautorefname}{Gleichung}%        
			\renewcommand*{\itemautorefname}{Punkt}%
                }%  
            % Fix to getting autorefs for subfigures right (thanks to Belinda Vogt for changing the definition)
            \providecommand{\subfigureautorefname}{\figureautorefname}%             
    }{\relax}
\makeatother


% ****************************************************************************************************
% 7. Last calls before the bar closes
% ****************************************************************************************************
% ********************************************************************
% Development Stuff
% ********************************************************************
\listfiles
%\PassOptionsToPackage{l2tabu,orthodox,abort}{nag}
%   \usepackage{nag}
%\PassOptionsToPackage{warning, all}{onlyamsmath}
%   \usepackage{onlyamsmath}



% ********************************************************************
% Sebastians Stuff
% ********************************************************************
\usepackage{amssymb} 
\usepackage{amsthm} 
\usepackage{bm}
\usepackage{imakeidx}
\makeindex

\usepackage{tikz}
\usetikzlibrary{positioning}
\usetikzlibrary{decorations.markings}
\usetikzlibrary{arrows.meta}

\usepackage{pgfplots}
\usepgfplotslibrary{fillbetween}
\pgfplotsset{compat=1.12}



\makeatletter

\newcounter{algorithmicH}% New algorithmic-like hyperref counter
\let\oldalgorithmic\algorithmic
\renewcommand{\algorithmic}{%
  \stepcounter{algorithmicH}% Step counter
  \oldalgorithmic}% Do what was always done with algorithmic environment
% \providecommand\theHALG@line{\thealgorithm.\arabic{ALG@line}}
\providecommand\theHALG@line{ALG@line.\thealgorithmicH.\arabic{ALG@line}}
% \renewcommand{\theHALG@line}{ALG@line.\thealgorithmicH.\arabic{ALG@line}}
\makeatother




\newcommand{\myAlgorithm}[4]{
  \begin{algorithm}[H]
\caption{#1: #4}
  \begin{tcolorbox}[title={#1}, colframe=black!10, coltitle=black]
    \begin{algorithmic}[1]
\Require #2
      #3
    \end{algorithmic}
  \end{tcolorbox}
  \end{algorithm}
}
\newcommand{\myNoteHelp}[1]{\marginpar{\textcolor{black}{#1}}}
\newcommand{\myNote}[1]{\myNoteHelp{\textrm{#1}}}

\newcommand{\myDef}[1]{\emph{#1}\myNoteHelp{#1}\index{#1}}
\newcommand{\myDefInformal}[1]{\emph{#1}\myNoteHelp{#1}\index{#1
    (informal)}}

\newcommand{\AlgReturn}[1]{ \textbf{return} #1}
\newcommand{\algo}[1]{ \mathsf{#1}}

\newcommand{\nats}{\mathbb{N}}
\newcommand{\landauO}{\mathcal{O}}
\newcommand{\reals}{\mathbb{R}}
\newcommand{\sgets}{\twoheadleftarrow}
\newcommand{\rats}{\mathbb{Q}}
\newcommand{\concat}{\mid \mid}
\newcommand{\powerset}{\mathcal{P}}
\newcommand{\machine}{\mathsf{M}}
\newcommand{\chan}{\mathcal{C}}
\newcommand{\stegos}{\mathcal{S}}
\newcommand{\DomainHistory}{\left(\Sigma^{n(\kappa)}\right)^{*}}
\DeclareMathOperator{\dom}{dom}
\DeclareMathOperator{\fout}{out}
\DeclareMathOperator{\fin}{in}
\DeclareMathOperator{\supp}{supp}
\DeclareMathOperator{\Exp}{Exp}
\DeclareMathOperator{\img}{img}
\DeclareMathOperator{\query}{query}
\DeclareMathOperator{\rate}{rate}
\DeclareMathOperator{\collision}{Collision}
\DeclareMathOperator{\Coll}{\mathsf{Coll}}
\DeclareMathOperator{\Sig-Forge}{\mathsf{Sig-Forge}}
\DeclareMathOperator{\DDisting}{\mathsf{Dist-Disting}}
\DeclareMathOperator{\SSCHA-Dist}{\mathsf{SS-CHA-Dist}}
\DeclareMathOperator{\SSCCA-Dist}{\mathsf{SS-CCA-Dist}}
\DeclareMathOperator{\SSRCCA-Dist}{\mathsf{SS-RCCA-Dist}}
\DeclareMathOperator{\CPA-Dist}{\mathsf{CPA-Dist}}
\DeclareMathOperator{\CPAD-Dist}{\mathsf{CPA\$-Dist}}
\DeclareMathOperator{\CCA-Dist}{\mathsf{CCA-Dist}}
\DeclareMathOperator{\CCAD-Dist}{\mathsf{CCA\$-Dist}}
\DeclareMathOperator{\negl}{\mathsf{negl}}
\DeclareMathOperator{\poly}{\mathsf{poly}}
\DeclareMathOperator{\lin}{\mathsf{lin}}

\DeclareMathOperator{\Fun}{Fun}
\DeclareMathOperator{\Pics}{Pics}

\DeclareMathOperator{\Gen}{\mathsf{Gen}}
\DeclareMathOperator{\Inv}{\mathsf{Inv}}
\DeclareMathOperator{\algf}{\algo{F}}
\DeclareMathOperator{\algh}{\algo{H}}
\DeclareMathOperator{\Fi}{\mathsf{Fi}}
\DeclareMathOperator{\Dist}{\mathsf{Dist}}
\DeclareMathOperator{\DDist}{\mathsf{DDist}}
\DeclareMathOperator{\Sign}{\mathsf{Sign}}
\DeclareMathOperator{\Vrfy}{\mathsf{Vrfy}}
\DeclareMathOperator{\Forg}{\mathsf{Fo}}
\DeclareMathOperator{\Enc}{\mathsf{Enc}}
\DeclareMathOperator{\Dec}{\mathsf{Dec}}
\DeclareMathOperator{\PKEnc}{\mathsf{PKEnc}}
\DeclareMathOperator{\PKDec}{\mathsf{PKDec}}
\DeclareMathOperator{\SEnc}{\mathsf{SEnc}}
\DeclareMathOperator{\Steg}{\mathsf{S}}
\DeclareMathOperator{\SDec}{\mathsf{SDec}}
\DeclareMathOperator{\Att}{\mathsf{A}}
\DeclareMathOperator{\Ward}{\mathsf{W}}
\DeclareMathOperator{\RejSam}{\mathsf{RejSam}}

\newcommand{\pk}{\textit{pk}}
\newcommand{\sk}{\textit{sk}}







\DeclareMathOperator{\Adv}{\mathbf{Adv}}
\DeclareMathOperator{\InSec}{\mathbf{InSec}}
\DeclareMathOperator{\unrel}{\mathbf{UnRel}}
\DeclareMathOperator{\hash}{hash}
\DeclareMathOperator{\prf}{prf}
\DeclareMathOperator{\sig}{sig}
\DeclareMathOperator{\cpa}{cpa}
\DeclareMathOperator{\dist}{dist}
\DeclareMathOperator{\sscha}{ss-cha}
\DeclareMathOperator{\sscca}{ss-cca}
\DeclareMathOperator{\ssrcca}{ss-rcca}
\DeclareMathOperator{\cpad}{cpa\scalebox{.7}{\$}}
\DeclareMathOperator{\cca}{cca}
\DeclareMathOperator{\ccad}{cca\scalebox{.7}{\$}}

\newcommand{\minent}{H_{\infty}}

\newcommand{\examplesymbol}{\( \diamond \)}


\newtheorem{theorem}{Theorem}
\newtheorem{corollary}[theorem]{Corollary}


\theoremstyle{definition}
\newtheorem*{exampleth}{Example}
\newenvironment{example}{\begin{exampleth}%
  \renewcommand{\qedsymbol}{\examplesymbol}\pushQED{\qed}}%
  {\popQED\end{exampleth}}



\makeatletter
\AtBeginDocument{%
  \renewcommand*{\AC@hyperlink}[2]{%
    \begingroup
      \hypersetup{hidelinks}%
      \hyperlink{#1}{#2}%
    \endgroup
  }%
}
\makeatother



% ****************************************************************************************************
% ********************************************************************
% Last, but not least...
% ********************************************************************
\usepackage{classicthesis} 
\usepackage{cleveref}
\usepackage{epsdice}
% ****************************************************************************************************


% ****************************************************************************************************
% 8. Further adjustments (experimental)
% ****************************************************************************************************
% ********************************************************************
% Changing the text area
% ********************************************************************
%\linespread{1.05} % a bit more for Palatino
%\areaset[current]{312pt}{761pt} % 686 (factor 2.2) + 33 head + 42 head \the\footskip
%\setlength{\marginparwidth}{7em}%
%\setlength{\marginparsep}{2em}%

% ********************************************************************
% Using different fonts
% ********************************************************************
%\usepackage[oldstylenums]{kpfonts} % oldstyle notextcomp
%\usepackage[osf]{libertine}
%\usepackage[light,condensed,math]{iwona}
%\renewcommand{\sfdefault}{iwona}
%\usepackage{lmodern} % <-- no osf support :-(
%\usepackage{cfr-lm} % 
%\usepackage[urw-garamond]{mathdesign} <-- no osf support :-(
%\usepackage[default,osfigures]{opensans} % scale=0.95 
%\usepackage[sfdefault]{FiraSans}
% ****************************************************************************************************
